\documentclass[UTF8]{ctexart}
\usepackage{geometry}
\usepackage{listings}
\usepackage{framed}
\usepackage{amsmath}
\usepackage{amssymb}

\geometry{left=2.0cm,right=2.0cm,top=2.5cm,bottom=2.5cm}

\lstset{
 breaklines,
 columns=fixed,       
 numbers=left,                                        % 在左侧显示行号
 frame=none,                                          % 不显示背景边框  
 commentstyle=\it,                % 设置代码注释的格式
 stringstyle=\rmfamily\slshape,   % 设置字符串格式
 showstringspaces=false,                              % 不显示字符串中的空格
 language=C++,
 basicstyle=\ttfamily,
}

\title{poplpr's \ code}
\author{poplpr}
\date{\today}


\begin{document}

\begin{titlepage}
	\vspace*{\fill}
	\begin{center}
		\normalfont
		{\Huge\bfseries poplpr's code}
		
		\bigskip
		{\Large poplpr}
		
		\medskip
		\today
	\end{center}
	\vspace{\stretch{3}}
\end{titlepage}

\tableofcontents
\newpage

\section{数学}
\subsection{高精度计算}

我只会java,这里贴一个java常用板子。

设此时有 BigInteger a,b,那么读入函数是 sc.nextBigInteger()。

int 类型的读入函数是 sc.nextInt()。

a+b:a.add(b)

a-b:a.substract(b)

a*b:a.multiply(b)

a/b:a.divide(b)

a\%b:a.remainder(b)

gcd(a,b):a.gcd(b)

下面是 \ a-b 的高精度运算。

\begin{framed}
\begin{lstlisting}
import java.math.*;
import java.util.*;
import java.io.*;
import java.text.*;

public class Main{
    public static void main(String[] args){
        Scanner sc = new Scanner(System.in);
        BigInteger a,b;
        a = sc.nextBigInteger(); b = sc.nextBigInteger();
        System.out.println(a.subtract(b));
    }
}
\end{lstlisting}
\end{framed}

\subsection{数论}

\subsubsection{素数}
有素数计数函数 $\pi(x) \sim \frac{x}{\ln x}$

一个数的质因数约有 $\frac{\log n}{\log \log n}$ 个

\subsubsection{判断单个数是否为质数}
时间 $O(\sqrt n)$,空间 $O(1)$

\begin{framed}
\begin{lstlisting}
bool isprime(int x) {
    if(x == 1) return false;
    int m = sqrt(x + 0.5);
    for(int i=2;i<=m;i++) {
        if(x % i == 0) return false;
    }
    return true;
}
\end{lstlisting}
\end{framed}

\subsubsection{欧拉筛:质数、欧拉函数、莫比乌斯函数、约数个数、约数个数和}
时间 $O(n)$ ,空间 $O(n)$
变量名称解释:

- tot:质数个数,函数Euler\_sieve(int n)也会返回质数个数。

- pri[i]:意为prime,这个数组里存储小于\ n\ 的质数。

- np[i]:意为not prime。这个bool数组是判断\ i\ 是否为质数的。

- phi[i]:即 $\varphi_i$。即为\ $i$\ 的欧拉函数

- mu[i]:即 $\mu_i$。即为\ $i$\ 的莫比乌斯函数

- di[i]:意为divisor。即为\ $i$\ 的约数个数函数

- si[i]:即 $\sigma_i$。即为\ $i$\ 的约数和函数

输入:n,意义见输出。

输出:小于等于\ n\ 的所有质数以及质数个数。小于等于\ n\ 的所有自然数的欧拉函数、莫比乌斯函数、约数个数函数、约数和函数。
\begin{framed}
\begin{lstlisting}
#include<cstdio>
using namespace std;
#define IL inline 
typedef long long LL;
const int N = 1e6 + 3;
bool np[N];
int pri[N],phi[N],mu[N],di[N],si[N];
IL int Euler_sieve(int n) {
    int tot = 0;
    phi[0] = mu[0] = di[0] = si[0] = 0;
    np[1] = phi[1] = mu[1] = di[1] = si[1] = 1;
    for(int i=2;i<=n;i++) {
        if(!np[i]) {
            pri[++tot] = i;
            phi[i] = i-1;
            mu[i] = -1;
            di[i] = 2;
            si[i] = i + 1;
        }
        for(int j=1;j<=tot&&i*pri[j]<=n;j++) {
            np[i*pri[j]] = true;
            if(i % pri[j] == 0) {
                phi[i*pri[j]] = pri[j] * phi[i];
                mu[i*pri[j]] = 0;
                di[i*pri[j]] = (di[i] << 1) - di[i/pri[j]];
                si[i*pri[j]] = si[i] + pri[j] * (si[i]-si[i/pri[j]]);
                break;
            }
            phi[i*pri[j]] = phi[i] * (pri[j]-1);
            mu[i*pri[j]] = -mu[i];
            di[i*pri[j]] = di[i] << 1;
            si[i*pri[j]] = si[i] * (1+pri[j]);
        }
    }
    return tot;
}
\end{lstlisting}
\end{framed}

\subsubsection{Miller-Rabin 素性测试}
时间复杂度 $O(k \log^3 n)$ ,空间复杂度 $O(1)$

long long 范围内都可以测出来。由于快速幂中使用 x 作为模数,且 long long 的极限为 9e18,若 x 大于 3e9,为避免 TLE 则开 long long,为避免 WA 则开 \_\_int128。

bool isprime(LL x)                      //判断 x 是不是质数。

\begin{framed}
\begin{lstlisting}
LL gcd(LL a,LL b) { return b == 0 ? a : gcd(b,a%b);}
IL LL ksm(LL a,LL b,LL mod) {
    LL res = 1LL;
    while(b) {
        if(b&1LL) res = (__int128)res * a % mod;
        a = (__int128)a * a % mod;
        b >>= 1LL;
    }
    return res;
}

IL bool mr(LL x,LL b) {
    LL d=x-1, p=0;
    while((d&1)==0) d>>=1,++p;
    int i;
    LL cur=ksm(b,d,x);
    if(cur==1 || cur==x-1) return true;
    for(i=1;i<=p;++i) {
        cur=(__int128)cur*cur%x;
        if(cur==x-1) return true;
    }
    return false;
}

IL bool isprime(LL x) {
    if(x == 46856248255981 || x < 2) return false;
    if(x==2 || x==3 || x==7 || x==61 || x==24251) return true;
    return mr(x,2) && mr(x,3) && mr(x,7) && mr(x,61) && mr(x,24251);
}
\end{lstlisting}
\end{framed}

\subsubsection{Pollard-Rho算法随机找一个\ x\ 的因数}
时间复杂度 $O(n^{\frac{1}{4}})$ ,空间复杂度 $O(1)$

LL PR(LL x)                 //输入一个 x ,返回 x 的一个因数。

\begin{framed}
\begin{lstlisting}
IL LL f(LL x,LL c,LL mod) { return ((__int128)x*x+c) % mod;}
IL LL PR(LL x) {
    LL s=0,t=0,c=1LL*rand()%(x-1) + 1;
    LL val = 1LL;
    for(int goal=1;;goal<<=1,s=t,val=1LL) {
        for(int stp=1;stp<=goal;stp++) {
            t = f(t,c,x);
            val = (__int128)val * abs(t-s) % x;
            if(stp%127 == 0) {
                LL d = gcd(val,x);
                if(d > 1) return d;
            }
        }
        LL d = gcd(val,x);
        if(d > 1) return d;
    }
}
\end{lstlisting}
\end{framed}

\subsubsection{结合Miller-Rabin 和 Pollard-Rho 求出一个数的所有质因子}

朴素做法大约是\ $O(q\frac{\sqrt n}{\ln n} + \sqrt n)$,需要打 $O(\sqrt n)$\ 的表。

此做法时间复杂度\ $O(k\log^3 n + n^\frac{1}{4}\frac{\log n}{\log \log n})$,空间复杂度\ $O(1)$

需要使用上面的\ Miller-Rabin\ 和\ Pollard-Rho\ 板子。

变量解释:

- faccnt:质因数个数(包括相同的)。

- fac[N]:存质因数的数组。

输入:x 

输出:x\ 的所有质因数。

注意:x == 1 时最好在函数外面特判一下,不要丢进函数里。

\begin{framed}
\begin{lstlisting}
int faccnt = 0;
LL fac[N];

IL void getfac(LL x) { 
    if(x == 1) return ;
    if(isprime(x)) fac[++faccnt] = x;
    else {
        LL d = x; while(d >= x) d = PR(x);
        getfac(d); getfac(x/d);
    }
}
int gaofac(LL x) {faccnt = 0; getfac(x); return faccnt;}
\end{lstlisting}
\end{framed}

\subsubsection{欧拉函数}
欧拉函数(Euler's totient function),即\ $\varphi(n)$,表示的是小于等于\ $n$\ 和\ $n$\ 互质的数的个数。

比如说\ $\varphi(1) = 1$。

当\ n\ 是质数的时候,显然有\ $\varphi(n) = n - 1$。

计算欧拉函数公式为:$\varphi(n)=n\prod_{i=1}^s (1 - \frac{1}{p_i})$

\subsubsection{欧拉定理与扩展欧拉定理}

欧拉定理:若 $\gcd (a,b)=1$,则 $a^{\varphi(m)} \equiv 1 \pmod{m} $

扩展欧拉定理(欧拉降幂):

请注意 b 与 $\varphi(p)$ 的大小关系。
$$
a^b \bmod p \equiv 
\begin{cases}
    a^{b \bmod \varphi(p)} \bmod p, & \gcd(a,p)=1 \\ 
    a^b \bmod p , & \gcd(a,p) \neq 1,b < \varphi(p) \\ 
    a^{b \bmod \varphi(p)+\varphi(p)} \bmod p , & \gcd(a,p) \neq 1, b \geq \varphi(p)
\end{cases}
$$

\subsubsection{乘法逆元}
设\ $ax \equiv 1 \pmod{b}$,求出\ x。

快速幂:$x = ksm(a,b-2,b)$

扩展欧几里得算法:$exgcd(a,b,d,x,y)$

打表:求出\ 1\~{}n\ 中所有数对于\ $p$\ 的逆元。

\begin{framed}
\begin{lstlisting}
inv[1] = 1;
for(int i=2;i<=n;++i) inv[i] = (long long)(p - p / i) * inv[p % i] % p;
\end{lstlisting}
\end{framed}

线性求任意 n 个数的逆元:设有 n 个数,$1 \leq a_i < p$ ,求出所有 $a_i$ 的逆元。

\begin{framed}
\begin{lstlisting}
s[0] = 1;
for(int i=1;i<=n;++i) s[i] = s[i - 1] * a[i] % p;
sv[n] = qpow(s[n], p - 2,p);
for(int i=n;i>=1;--i) sv[i - 1] = sv[i] * a[i] % p;
for(int i=1;i<=n;++i) inv[i] = sv[i] * s[i - 1] % p;
\end{lstlisting}
\end{framed}

\subsubsection{中国剩余定理(CRT)}
如果模数很大而且不是质数的情况,可以将模数拆分,若模数唯一分解式中每个质因数次数都为\ 1,则可以分别求出表达式在不同质因数下的数值,再使用\ CRT\ 求解。

\subsubsection{卢卡斯定理(Lucas)}
对于质数\ p,有
$$
{n \choose m} \mod p = {\lfloor n/p \rfloor \choose \lfloor m/p \rfloor} \cdot {n \mod p \choose m \mod p} \mod p
$$
如果递归求解的话,p\ 的范围不能太大,一般在\ $10^5$\ 左右。

\begin{framed}
\begin{lstlisting}
IL LL C(LL n,LL m,LL p) {
    if(n<m) return 0;
    return jc[n]*ksm(jc[m],p-2,p)%p*ksm(jc[n-m],p-2,p)%p;
    // return jc[n] * ijc[m] % p * ijc[n-m] % p;
}

IL LL Lucas(LL n,LL m,LL p) {
    if(n<m) return 0; if(!n) return 1; if(!m) return 1;
    return Lucas(n/p,m/p,p) * C(n%p,m%p,p) % p;
}
\end{lstlisting}
\end{framed}

\subsubsection{莫比乌斯函数与狄利克雷卷积}

积性函数定义:若\ $x \bot y$\ 且\ $f(xy)=f(x)f(y)$,则\ $f(n)$\ 为积性函数。($x \bot y \iff gcd(x,y)=1$)

积性函数性质:若\ $f(x)$\ 和\ $g(x)$\ 都为积性函数,则以下函数也为积性函数。
\begin{align*}
h(x)&=f(x^p) \\
h(x)&=f^p(x) \\
h(x)&=f(x)g(x) \\
h(x)&=\sum_{d|x}f(d)g(\frac{x}{d})
\end{align*}
例子:

- 单位函数:$\epsilon(n)=1[n=1]$。

- 恒等函数:$id_k(n)=n^k$。$id_1(n)$\ 通常简记作\ $id(n)$。

- 常数函数:$1(n)=1$。

- 除数函数:$\sigma_k(n)=\sum_{d|n}d^k$。$\sigma_0(n)$\ 通常简记作\ $d(n)$\ 或\ $\tau(n)$,这是约数个数函数。$\sigma_1(n)$\ 通常简记作\ $\sigma(n)$,这是约数和函数。

- 欧拉函数:$\varphi(n)=\sum_{i=1}^n[gcd(i,n)=1]$

- 莫比乌斯函数:
  $$\mu(n)=
  \begin{cases}
  1& \text{$n=1$}\\
  0& \text{$\exists d:d^2|n$}\\
  (-1)^{\omega(n)}& \text{otherwise}
  \end{cases}$$
  其中\ $\omega(n)$\ 表示n的不同质因子个数,是一个加性函数。

狄利克雷卷积定义:定义两个数论函数\ $f,g$\ 的\ Dirichlet\ 卷积为
$$ (f\ast g)(n)=\sum_{d\mid n}f(d)g(\frac{n}{d}) $$
狄利克雷卷积性质:Dirichlet\ 卷积满足以下运算规律:

- 交换律\ $(f * g=g * f)$ 

- 结合律\ $(f * g) * h=f * (g * h)$ 

- 分配律\ $f * (g+h)=f * g+f * h$ 

- $f*\varepsilon=f$,其中\ $\varepsilon$\ 为\ Dirichlet\ 卷积的单位元(任何函数卷\ $\varepsilon$\ 都为其本身)

例子:
\begin{align*} 
\varepsilon=\mu \ast 1&\iff\varepsilon(n)=\sum_{d\mid n}\mu(d)\ \\
d=1 \ast 1&\iff d(n)=\sum_{d\mid n}1\ \\ 
\sigma=\operatorname{id} \ast 1&\iff\sigma(n)=\sum_{d\mid n}d\  \\
\varphi=\mu \ast \operatorname{id}&\iff\varphi(n)=\sum_{d\mid n}d\cdot\mu(\frac{n}{d}) \\
id = \varphi \ast 1 &\iff id(n) = \sum_{d \mid n} \varphi(d)
\end{align*}

莫比乌斯函数定义:$\mu$ 为莫比乌斯函数,定义为
$$ \mu(n)= 
\begin{cases} 
1&n=1\\
0&n\text{ 含有平方因子}\\ 
(-1)^k&k\text{ 为\ }n\text{\ 的本质不同质因子个数}\ 
\end{cases} 
$$

详细解释一下:

令\ $n=\prod_{i=1}^kp_i^{c_i}$,其中\ $p_i$\ 为质因子,$c_i\ge 1$。上述定义表示:

1. $n=1$\ 时,$\mu(n)=1$

2. 对于\ $n\not= 1$\ 时:

   - 1. 当存在\ $i\in [1,k]$,使得\ $c_i > 1$\ 时,$\mu(n)=0$,也就是说只要某个质因子出现的次数超过一次,$\mu(n)$\ 就等于\ $0$;
   
   - 2. 当任意\ $i\in[1,k]$,都有\ $c_i=1$\ 时,$\mu(n)=(-1)^k$ ,也就是说每个质因子都仅仅只出现过一次时,即\ $n=\prod_{i=1}^kp_i$ ,${p_i}_{i=1}^k$\ 中个元素唯一时,$\mu(n)$\ 等于\ $-1$\ 的\ $k$\ 次幂,此处\ $k$\ 指的便是仅仅只出现过一次的质因子的总个数。

莫比乌斯函数性质:莫比乌斯函数不但是积性函数,还有如下性质:

$$ \sum_{d\mid n}\mu(d)= \begin{cases} 1&n=1\\ 0&n\neq 1\ \end{cases} $$

即\ $\sum_{d\mid n}\mu(d)=\varepsilon(n)$,$\mu * 1 =\varepsilon$

\subsubsection{数论分块}
若有可能存在 $k < n$ 时,求出 $\sum_{i=1}^n \lfloor \frac{k}{i} \rfloor$
\begin{framed}
\begin{lstlisting}
LL ans = 0;
for (LL l = 1, r; l <= n;l = r + 1) {  // 此处 l 意同 i,r 意同 j  ,下个计算区间的 l 应为上个区间的 r+1
  if (k / l != 0) r = min(k / (k / l), n);
  else r = n;  // l 大于 k 时
  ans += (k / l) * (r - l + 1); // 这个区间内 k/i 均相等,对 i 求和是等差数列求和
}
\end{lstlisting}
\end{framed}

若不存在 $k < n$\ (这里 k = n || k = m)时,求出 $\sum_{i=1}^{min(n,m)} \lfloor \frac{n}{i} \rfloor \lfloor \frac{m}{i} \rfloor$

此时可将代码中 `r = n/(n/i)` 替换成 `r = min(n/(n/i), m/(m/i))` ,并且不需要 k / l != 0 一句判断。

\subsubsection{莫比乌斯反演}
公式:设 $f(n),g(n)$ 为两个数论函数。

如果有 $f(n)=\sum_{d\mid n}g(d)$ ,那么有 $g(n)=\sum_{d\mid n}\mu(d)f(\frac{n}{d})$ 。

如果有 $f(n)=\sum_{n|d}g(d)$ ,那么有 $g(n)=\sum_{n|d}\mu(\frac{d}{n})f(d)$ 。

莫比乌斯反演的非卷积形式:

对于数论函数 $f,g$ 和完全积性函数 $t$ 且 $t(1)=1$ :
$$
f(n)=\sum_{i=1}^nt(i)g\left(\left\lfloor\frac{n}{i}\right\rfloor\right)\ \iff g(n)=\sum_{i=1}^n\mu(i)t(i)f\left(\left\lfloor\frac{n}{i}\right\rfloor\right)
$$

\subsubsection{杜教筛}
杜教筛被用来处理数论函数的前缀和问题。对于求解一个前缀和,杜教筛可以在低于线性时间的复杂度内求解.

对于数论函数 $f$ ,要求我们计算 $S(n)=\sum_{i=1}^{n}f(i)$ .

我们想办法构造一个 $S(n)$ 关于 $S\left(\left\lfloor\frac{n}{i}\right\rfloor\right)$ 的递推式

对于任意一个数论函数 $g$ ,必满足
$$
\sum_{i=1}^{n}\sum_{d \mid i}f(d)g\left(\frac{i}{d}\right)=\sum_{i=1}^{n}g(i)S\left(\left\lfloor\frac{n}{i}\right\rfloor\right)\ \Leftrightarrow \sum_{i=1}^{n}(f\ast g)(i)=\sum_{i=1}^{n}g(i)S\left(\left\lfloor\frac{n}{i}\right\rfloor\right)
$$
可以得到递推式:
$$
g(1)S(n)=\sum_{i=1}^n(f\ast g)(i)-\sum_{i=2}^ng(i)S\left(\left\lfloor\frac{n}{i}\right\rfloor\right)
$$
那么假如我们可以快速对 $\sum_{i=1}^n(f \ast g)(i)$ 求和,并用数论分块求解 $\sum_{i=2}^ng(i)S\left(\left\lfloor\frac{n}{i}\right\rfloor\right)$ 就可以在较短时间内求得 $g(1)S(n)$ .

\paragraph{问题一} 求 $S_1(n)= \sum_{i=1}^{n} \mu(i)$ 和 $S_2(n)= \sum_{i=1}^{n} \varphi(i)$ 的值, $n\le 2^{31} -1$ 。

\subparagraph{莫比乌斯函数前缀和}

由狄利克雷卷积 ,我们知道:

$\because \epsilon =\mu \ast 1$($\epsilon(n)=[n=1]$)

$\therefore \epsilon (n)=\sum_{d \mid n} \mu(d)$

$S_1(n)=\sum_{i=1}^n \epsilon (i)-\sum_{i=2}^n S_1(\lfloor \frac n i \rfloor)$

$= 1-\sum_{i=2}^n S_1(\lfloor \frac n i \rfloor)$

观察到 $\lfloor \frac n i \rfloor$ 最多只有 $O(\sqrt n)$ 种取值,我们就可以应用整除分块(或称数论分块)来计算每一项的值了。

直接计算的时间复杂度为 $O(n^{\frac 3 4})$ 。考虑先线性筛预处理出前 $n^{\frac 2 3}$ 项,剩余部分的时间复杂度为 $O(\int_{0}^{n^{\frac 1 3}} \sqrt{\frac{n}{x}} ~ dx)=O(n^{\frac 2 3})$

对于较大的值,需要用 map 存下其对应的值,方便以后使用时直接使用之前计算的结果。

\subparagraph{欧拉函数前缀和}

当然也可以用杜教筛求出 $\varphi (x)$ 的前缀和,但是更好的方法是应用莫比乌斯反演:

$$\sum_{i=1}^n \sum_{j=1}^n 1[\gcd(i,j)=1]=\sum_{i=1}^n \sum_{j=1}^n \sum_{d \mid i,d \mid j} \mu(d)=\sum_{d=1}^n \mu(d) {\lfloor \frac n d \rfloor}^2$$

由于题目所求的是 $\sum_{i=1}^n \sum_{j=1}^i 1[\gcd(i,j)=1]$ ,所以我们排除掉 $i=1,j=1$ 的情况,并将结果除以 $2$ 即可。

观察到,只需求出莫比乌斯函数的前缀和,就可以快速计算出欧拉函数的前缀和了。时间复杂度 $O(n^{\frac 2 3})$ 。

\subparagraph{使用杜教筛求解}

求 $S(i)=\sum_{i=1}^n\varphi(i)$ .

同样的, $\varphi\ast 1=ID$

\begin{align*}
&\sum_{i=1}^n(\varphi\ast 1)(i) = \sum_{i=1}^n1\cdot S\left(\left\lfloor\frac{n}{i}\right\rfloor\right)\\
&\sum_{i=1}^nID(i) = \sum_{i=1}^n1\cdot S\left(\left\lfloor\frac{n}{i}\right\rfloor\right)\\
&\frac{1}{2}n(n+1)=\sum_{i=1}^nS\left(\left\lfloor\frac{n}{i}\right\rfloor\right)\\
&S(n) = \frac{1}{2}n(n+1)-\sum_{i=2}^nS\left(\left\lfloor\frac{n}{i}\right\rfloor\right)\ 
\end{align*}

\begin{framed}
\begin{lstlisting}
#include <algorithm>
#include <cstdio>
#include <cstring>
#include <map>
using namespace std;
const int maxn = 2000010;
typedef long long ll;
ll T, n, pri[maxn], cur, mu[maxn], sum_mu[maxn];
bool vis[maxn];
map<ll, ll> mp_mu;
ll S_mu(ll x) {
  if (x < maxn) return sum_mu[x];
  if (mp_mu[x]) return mp_mu[x];
  ll ret = 1ll;
  for (ll i = 2, j; i <= x; i = j + 1) {
    j = x / (x / i);
    ret -= S_mu(x / i) * (j - i + 1);
  }
  return mp_mu[x] = ret;
}
ll S_phi(ll x) {
  ll ret = 0ll;
  for (ll i = 1, j; i <= x; i = j + 1) {
    j = x / (x / i);
    ret += (S_mu(j) - S_mu(i - 1)) * (x / i) * (x / i);
  }
  return ((ret - 1) >> 1) + 1;
}
int main() {
  scanf("%lld", &T);
  mu[1] = 1;
  for (int i = 2; i < maxn; i++) {
    if (!vis[i]) {
      pri[++cur] = i;
      mu[i] = -1;
    }
    for (int j = 1; j <= cur && i * pri[j] < maxn; j++) {
      vis[i * pri[j]] = true;
      if (i % pri[j])
        mu[i * pri[j]] = -mu[i];
      else {
        mu[i * pri[j]] = 0;
        break;
      }
    }
  }
  for (int i = 1; i < maxn; i++) sum_mu[i] = sum_mu[i - 1] + mu[i];
  while (T--) {
    scanf("%lld", &n);
    printf("%lld %lld\n", S_phi(n), S_mu(n));
  }
  return 0;
}
\end{lstlisting}
\end{framed}

\subsubsection{Messiel-Lehmer 算法求质数幂次前缀和}
快速求1e11以内质数个数

这段代码预处理部分就需要 128 MB 内存,大概预处理前 $10^7$ 的数,然后压位。时间复杂度大概是 $O(\frac{n^{\frac{2}{3}}}{\log n})$

如果空间只给了 64 MB ,那么把 MAXM 改为 50010,MAXP 改为 66666,MAX 改为 1000010 即可卡过。

\begin{framed}
\begin{lstlisting}
#include <bits/stdtr1c++.h>

#define MAXN 100    // pre-calc max n for phi(m, n)
#define MAXM 100010 // pre-calc max m for phi(m, n)
#define MAXP 666666 // max primes counter
#define MAX 10000010    // max prime
#define clr(ar) memset(ar, 0, sizeof(ar))
#define read() freopen("lol.txt", "r", stdin)
#define dbg(x) cout << #x << " = " << x << endl
// compressed bool flag for sieve prime. (i >> 1) because even numbers are omitted. 
#define setbit(ar, i) (((ar[(i) >> 6]) |= (1 << (((i) >> 1) & 31)))) 
#define chkbit(ar, i) (((ar[(i) >> 6]) & (1 << (((i) >> 1) & 31))))
#define isprime(x) (( (x) && ((x)&1) && (!chkbit(ar, (x)))) || ((x) == 2))

using namespace std;

namespace pcf{ // prime counting function
    long long dp[MAXN][MAXM];
    unsigned int ar[(MAX >> 6) + 5] = {0};
    int len = 0, primes[MAXP], counter[MAX];

    void Sieve(){
        setbit(ar, 0), setbit(ar, 1);
        for (int i = 3; (i * i) < MAX; i++, i++){
            if (!chkbit(ar, i)){
                int k = i << 1;
                for (int j = (i * i); j < MAX; j += k) setbit(ar, j);
            }
        }

        for (int i = 1; i < MAX; i++){
            counter[i] = counter[i - 1];
            if (isprime(i)) primes[len++] = i, counter[i]++;
        }
    }

    void init(){
        Sieve();
        for (int n = 0; n < MAXN; n++){
            for (int m = 0; m < MAXM; m++){
                if (!n) dp[n][m] = m;
                else dp[n][m] = dp[n - 1][m] - dp[n - 1][m / primes[n - 1]];
            }
        }
    }

    long long phi(long long m, int n){
        if (n == 0) return m;
        if (primes[n - 1] >= m) return 1;
        if (m < MAXM && n < MAXN) return dp[n][m];
        return phi(m, n - 1) - phi(m / primes[n - 1], n - 1);
    }

    long long Lehmer(long long m){
        if (m < MAX) return counter[m];

        long long w, res = 0;
        int i, a, s, c, x, y;
        s = sqrt(0.9 + m), y = c = cbrt(0.9 + m);
        a = counter[y], res = phi(m, a) + a - 1;
        for(i=a;primes[i]<=s;i++) res = res - Lehmer(m / primes[i]) + 
                                         Lehmer(primes[i]) - 1;
        return res;
    }
}

int main(){
    pcf::init();
    long long n, res;

    while (scanf("%lld", &n) != EOF){
        res = pcf::Lehmer(n);
        printf("%lld\n", res);
    }
    return 0;
}
\end{lstlisting}
\end{framed}

\subsection{线性代数}
\subsubsection{线性基}
\paragraph{查询异或和最大值}
给定 n 个整数(数字可能重复),求在这些数中选取任意个,使得他们的异或和最大。

void insert(LL x)                       // 插入一个数 x

LL qmax()                                 // 查询最大的异或和

查询原集合内任意几个元素 xor 的最大值,就可以用线性基解决。

将线性基从高位向低位扫,若 xor 上当前扫到的 $a_x$ 答案变大,就把答案异或上 $a_x$。

为什么能行呢?因为从高往低位扫,若当前扫到第 $i$ 位,意味着可以保证答案的第 $i$ 位为 1,且后面没有机会改变第 $i$ 位。

\begin{framed}
\begin{lstlisting}
#include<cstdio>
#include<cstring>
#include<algorithm>
using namespace std;
typedef long long LL;
#define ri register int 
#define IL inline
const int N = 100;
int n;
LL b;
LL a[N];
void insert(LL x) {
    for(ri i=60;~i;i--) {
        if(x&(1LL<<i)) {
            if(!a[i]) {a[i]=x; return;}
            else x ^= a[i];
        }
    }
}
LL qmax() {
    LL ans = 0;
    for(ri i=60;~i;i--) {
        ans = max(ans,ans^a[i]);
    }
    return ans;
}

int main() {
    scanf("%d",&n);
    for(ri i=1;i<=n;i++) {
        scanf("%lld",&b);
        insert(b);
    }
    printf("%lld\n",qmax());
    return 0;
}
\end{lstlisting}
\end{framed}

\paragraph{查询异或和最小值}
查询原集合内任意几个元素 xor 的最小值,就是线性基集合所有元素中最小的那个。

\paragraph{查询某个数是否能被异或出来}
查询某个数是否能被异或出来,类似于插入,如果最后插入的数 $p$ 被异或成了 0,则能被异或出来。

\subsection{多项式}
\subsubsection{lagrange 插值}
给出 $n+1$ 个点,下标为 x[0]\~{}x[n],求出横坐标在 $xi$ 处的纵坐标。

\paragraph{横坐标连续的lagrange插值}
时间复杂度 $O(n)$
$$ pre_i=\prod_{j=0}^i k-j $$
$$ suf_i = \prod_{j=i}^n k-j $$
$$ f(k) = \sum_{i=0}^n y_i \prod_{i\not=j} \frac{k-j}{i-j} = \sum_{i=0}^n y_i \frac{pre_{i-1}\cdot suf_{i+1}}{i! \cdot (n-i)!}$$

\begin{framed}
\begin{lstlisting}
//(x0,y0),(x1,y1),..,(xn,yn) -> (xi,y_xi)
LL pre[N],suf[N];
LL lagrange(int n,int *x,LL *y,LL xi) {
    pre[0] = (xi-x[0]) % mod; suf[n+1] = 1;
    for(int i=1;i<=n;i++) pre[i] = (xi-x[i])%mod*pre[i-1]%mod;
    for(int i=n;i>=0;i--) suf[i] = (xi-x[i])%mod*suf[i+1]%mod;
    LL ans = y[0]%mod*suf[1]%mod*ijc[n]%mod;
    if(n&1) ans = mod-ans; 
    for(int i=1;i<=n;i++) {
        LL now = y[i]*pre[i-1]%mod*suf[i+1]%mod*ijc[i]%mod*ijc[n-i]%mod;
        if((n-i)&1) ans = (ans - now + mod) % mod;
        else ans = (ans + now) % mod;
    }
    return ans;
}
\end{lstlisting}
\end{framed}

\paragraph{横坐标不连续的lagrange插值}
时间复杂度 $O(n^2)$
$$ f(k) = \sum_{i=0}^n y_i \prod_{i=0,i\not=j}^n \frac{k-x_j}{x_i-x_j} $$

\begin{framed}
\begin{lstlisting}
#include<cstdio>
#include<cmath>
#include<cstring>
#include<algorithm>
using namespace std;
#define IL inline
#define ri register int 
const int mod = 998244353;

int ksm(int a,int b,int p) {
    int res = 1;
    while(b) {
        if(b&1) res = 1LL * res * a % p;
        a = 1LL * a * a % p;
        b >>= 1;
    }
    return res;
}

int lagrange(int n,int *x,int *y,int xi) {
    int ans = 0;
    for(int i=0;i<=n;i++) {
        int fz = 1, fm = 1;
        for(int j=0;j<=n;j++) if(i != j) {
            fz = 1LL * fz * (xi - x[j]) % mod;
            fm = 1LL * fm * (x[i]-x[j]) % mod;
        }
        ans = (ans + 1LL*y[i]*fz%mod*ksm(fm,mod-2,mod) % mod) % mod;
    }
    return (ans + mod) % mod;
}

const int N = 2e3 + 3;

int n,m;
int a[N],b[N];

int main() {
    scanf("%d%d",&n,&m);
    for(int i=0;i<n;i++) scanf("%d%d",&a[i],&b[i]);
    printf("%d\n",lagrange(n-1,a,b,m));
    return 0;
}
\end{lstlisting}
\end{framed}

\subsubsection{快速傅里叶变换(FFT)}
带共轭优化
\begin{framed}
\begin{lstlisting}
#include<bits/stdc++.h>
using namespace std;

typedef long long ll;
typedef double db;
typedef long double ld;
#define IL inline
#define dbg1(x) cerr << #x << " = " << x << ", "
#define dbg2(x) cerr << #x << " = " << x << endl

template<typename Tp> IL void read(Tp& x) {
    x=0; int f=1; char ch=getchar();
    while(!isdigit(ch)) {if(ch == '-') f=-1; ch=getchar();}
    while(isdigit(ch)) { x = x*10+ch-'0'; ch=getchar();}
    x *= f;
}
int buf[22];
template<typename Tp> IL void write(Tp x) {
    int p = 0;
    if(x < 0) {putchar('-'); x=-x;}
    if(x == 0) buf[++p] = 0;
    else while(x) {
        buf[++p] = x % 10;
        x /= 10;
    }
    for(int i=p;i;i--) putchar('0' + buf[i]);
}

struct cp {
    db x, y;
    cp(db x=0.0, db y=0.0):x(x), y(y) {}
    IL cp operator + (const cp& o) const { return cp(x+o.x, y+o.y);}
    IL cp operator - (const cp& o) const { return cp(x-o.x, y-o.y);}
    IL cp operator * (const cp& o) const { return cp(x*o.x-y*o.y, x*o.y+y*o.x);}
    IL cp operator * (const db& p) const { return cp(x*p, y*p);}
    IL cp operator / (const db& p) const { return cp(x/p, y/p);}
    IL cp operator ! () const { return cp(x, -y);}
};
IL cp polar(const db& rho, const db& theta) {return cp(rho*cos(theta), rho*sin(theta));}

const int N = 1048576;
const db PI = acos(-1.0);

int rev[N];
cp eps[N], ieps[N];
IL void initeps() {
    for(int i=0;i<N;i++) eps[i] = polar(1.0, 2*PI*i/N);
    ieps[0] = eps[0] = 1;
    for(int i=1;i<N;i++) ieps[i] = eps[N-i];
}
IL void cal_rev(int degA, int degB, int& lim, int &p) {
    lim = 1; p = 0;
    while(lim <=  ((degA+degB) >> 1)) {lim <<= 1; p++;}
    for(int i=0;i<lim;i++) rev[i] = (rev[i>>1]>>1) | ((i&1) << (p-1));
}
IL void trans(cp *x, cp *w, int n) {
    for(int i=0;i<n;i++) if(i < rev[i]) swap(x[i], x[rev[i]]);
    for(int i=2;i<=n;i<<=1) {
        int l = i >> 1, d = N / i;
        for(int j=0;j<n;j+=i) {
            for(int k=0;k<l;k++) {
                cp t = x[j+k+l] * w[d*k];
                x[j+k+l] = x[j+k] - t;
                x[j+k] = x[j+k] + t;
            }
        }
    }
}
IL void dft(cp *x, int n) {trans(x, eps, n);}
IL void idft(cp *x, int n) {trans(x, ieps, n); for(int i=0;i<n;i++) x[i] = x[i] / n;}

int n, m, P;

int f[N], g[N], ans[N<<1];
cp fft_a[N], fft_b[N], fft_c[N];

IL int polymul(int *C, int *A, int *B, int degA, int degB) {
    int lim, lglim;
    cal_rev(degA, degB, lim, lglim);
    for(int i=0;i<=degA;i++) (i & 1 ? fft_a[i>>1].y : fft_a[i>>1].x) = A[i];
    for(int i=degA+1;i<lim;i++) (i & 1 ? fft_a[i>>1].y : fft_a[i>>1].x) = 0;
    for(int i=0;i<=degB;i++) (i & 1 ? fft_b[i>>1].y : fft_b[i>>1].x) = B[i]; 
    for(int i=degB+1;i<lim;i++) (i & 1 ? fft_b[i>>1].y : fft_b[i>>1].x) = 0;
    dft(fft_a, lim); dft(fft_b, lim);
    int d = N / lim;
    for(int i=0;i<lim;i++) {
        int j = (lim-1) & (lim-i);
        fft_c[i] = (fft_a[i] * fft_b[i] * 4 - (fft_a[i]-!fft_a[j]) * (fft_b[i]-!fft_b[j]) * (eps[d*i]+cp(1,0))) * 0.25;
    }
    idft(fft_c, lim);
    for(int i=0;i<=degA+degB;i++) C[i] = (i & 1 ? fft_c[i>>1].y : fft_c[i>>1].x ) + 0.5;
    return degA + degB;
}

int main() {
    initeps();
    read(n); read(m);
    for(int i=0;i<=n;i++) read(f[i]);
    for(int i=0;i<=m;i++) read(g[i]);
    polymul(ans, f, g, n, m);
    for(int i=0;i<=n+m;i++) {write(ans[i]); putchar(" \n"[i==n+m]);}
    return 0;
}
\end{lstlisting}
\end{framed}

\subsubsection{快速数论变换(NTT)}
\begin{framed}
\begin{lstlisting}
#include<cstdio>
#include<cstring>
#include<algorithm>
#include<cmath>
using namespace std;
typedef double db;
#define IL inline

const int MOD = 998244353;
const int N = 2097152;
const int G = 3;

IL int ksm(int a,int b,int m) {
	int res = 1;
	while(b) {
		if(b&1) res = 1LL * res * a % m;
		a = 1LL * a * a % m;
		b >>= 1;
	}
	return res;
}
IL int inv(int x) { return ksm(x,MOD-2,MOD);}

int rev[N];
int eps[N], ieps[N];
IL void initeps() {
	int g = ksm(G, (MOD-1) / N, MOD), ig = inv(g);
	ieps[0] = eps[0] = 1;
	for(int i=1;i<N;i++) eps[i] = 1LL * eps[i-1] * g % MOD;
	for(int i=1;i<N;i++) ieps[i] = 1LL * ieps[i-1] * ig % MOD;
}
IL void cal_rev(int degA, int degB, int& lim, int& p) {
	lim = 1; p = 0;
	while(lim <= (degA+degB)) {lim <<= 1; ++p;}
	for(int i=0;i<lim;i++) rev[i] = (rev[i>>1]>>1) | ((i&1)<<(p-1));
}
IL void trans(int *x, int* w, int n) {
	for(int i=0; i<n; i++) if(i < rev[i]) swap(x[i],x[rev[i]]);
	for(int i=2;i<=n;i<<=1) {
		int l = i>>1, d = N / i;
		for(int j=0;j<n;j+=i) {
			for(int k=0;k<l;k++) {
				int t = 1LL * x[j+k+l] * w[d*k] % MOD;
				x[j+k+l] = (1LL * x[j+k] - t + MOD) % MOD;
				x[j+k] = (x[j+k] + t) % MOD;
			}
		}
	}
}
IL void dft(int* x, int n) { trans(x,eps,n);}
IL void idft(int* x, int n) { 
	trans(x,ieps,n); 
	int in = inv(n);
	for(int i=0;i<n;i++) x[i] = 1LL * x[i] * in % MOD;
}

int ntt_a[N], ntt_b[N];
IL int mul_normal(int *C, int *A, int *B, int degA, int degB) {
	int lim, p;
	cal_rev(degA, degB, lim, p); // if length is the same, u can write it in the main. 
	for(int i=0;i<=degA;i++) ntt_a[i] = A[i];
    for(int i=degA+1;i<lim;i++) ntt_a[i] = 0;
	for(int i=0;i<=degB;i++) ntt_b[i] = B[i];
    for(int i=degB+1;i<lim;i++) ntt_b[i] = 0;
	dft(ntt_a, lim); dft(ntt_b, lim);
	for(int i=0;i<lim;i++) ntt_a[i] = 1LL * ntt_a[i] * ntt_b[i] % MOD;
	idft(ntt_a, lim);
	for(int i=0;i<=degA+degB;i++) C[i] = ntt_a[i];
	return degA + degB; // return length of the poly.
}

int f[N], g[N], ans[N];

int main() {
	initeps();
	int n,m; scanf("%d%d",&n,&m);
	for(int i=0;i<=n;i++) scanf("%d",&f[i]);
	for(int i=0;i<=m;i++) scanf("%d",&g[i]);
	mul_normal(ans,f,g,n,m);
	for(int i=0;i<=n+m;i++) printf("%d%c",ans[i]," \n"[i==n+m]);
	return 0;
} 
\end{lstlisting}
\end{framed}

\subsubsection{快速沃尔什变换(FWT)}

给定长度为 $2^n$ 两个序列 A,B,设

$$
C_i=\sum_{j\oplus k = i}A_j \times B_k
$$

分别当 $\oplus$ 是 or,and,xor 时求出 C

输入

第一行一个数 $n(n \leq 17)$。 第二行 $2^n$ 个数 $A_0..A_{2^n-1}$​ 第三行 $2^n$ 个数 $B_0..B_{2^n-1}$​

输出

三行每行 $2^n$ 个数,分别代表 $\oplus$ 是 or,and,xor 时$C_0..C_{2^n-1}$ ​的值 $\bmod 998244353$

\begin{framed}
\begin{lstlisting}
#include<cstdio>
#include<cstring>
#include<algorithm>
using namespace std;
typedef long long LL;
#define IL inline
#define ri register int 
#define mk make_pair

const int N = 19;
const int MOD = 998244353;
const int inv2 = 499122177;

int n;
int a[1<<N],b[1<<N];
int ora[1<<N],orb[1<<N],orc[1<<N];
int anda[1<<N],andb[1<<N],andc[1<<N];
int xora[1<<N],xorb[1<<N],xorc[1<<N];

IL void OR(int *A,int n,int x) {
    for(ri i=1;i<(1<<n);i<<=1) {
        for(ri p=i<<1,j=0;j<(1<<n);j+=p) {
            for(ri k=0;k<i;k++) {
                A[i+j+k] = (1LL * A[i+j+k] + A[j+k] * x + MOD) % MOD;
            }
        }
    }
} 

IL void AND(int *A,int n,int x) {
    for(ri i=1;i<(1<<n);i<<=1) {
        for(ri p=i<<1,j=0;j<(1<<n);j+=p) {
            for(ri k=0;k<i;k++) {
                A[j+k] = (1LL * A[j+k] + A[i+j+k]*x + MOD) % MOD;
            }
        }
    }
}

IL void XOR(int *A,int n,int t) {
    for(ri i=1;i<(1<<n);i<<=1) {
        for(ri p=i<<1,j=0;j<(1<<n);j+=p) {
            for(ri k=0;k<i;k++) {
                int x = A[j+k], y = A[i+j+k];
                A[j+k] = (1LL*x+y) % MOD;
                A[i+j+k] = (1LL*x-y+MOD) % MOD;
                if(t == -1) {
                    A[j+k] = 1LL * A[j+k] * inv2 % MOD;
                    A[i+j+k] = 1LL * A[i+j+k] * inv2 % MOD;
                }
            }
        }
    }
}

int main() {
    scanf("%d",&n);
    for(ri i=0;i<(1<<n);i++) scanf("%d",a+i);
    for(ri i=0;i<(1<<n);i++) scanf("%d",b+i);
    memcpy(ora,a,sizeof(int)*(1<<n));
    memcpy(orb,b,sizeof(int)*(1<<n));
    memcpy(anda,a,sizeof(int)*(1<<n));
    memcpy(andb,b,sizeof(int)*(1<<n));
    memcpy(xora,a,sizeof(int)*(1<<n));
    memcpy(xorb,b,sizeof(int)*(1<<n));
    OR(ora,n,1); OR(orb,n,1);
    for(ri i=0;i<(1<<n);i++) orc[i] = 1LL*ora[i]*orb[i]%MOD;
    OR(orc,n,-1);
    AND(anda,n,1); AND(andb,n,1);
    for(ri i=0;i<(1<<n);i++) andc[i] = 1LL*anda[i]*andb[i]%MOD;
    AND(andc,n,-1);
    XOR(xora,n,1); XOR(xorb,n,1);
    for(ri i=0;i<(1<<n);i++) xorc[i] = 1LL*xora[i]*xorb[i]%MOD;
    XOR(xorc,n,-1);
    for(ri i=0;i<(1<<n);i++) printf("%d%c",orc[i]," \n"[i==(1<<n)-1]);
    for(ri i=0;i<(1<<n);i++) printf("%d%c",andc[i]," \n"[i==(1<<n)-1]);
    for(ri i=0;i<(1<<n);i++) printf("%d%c",xorc[i]," \n"[i==(1<<n)-1]);
    return 0;
}
\end{lstlisting}
\end{framed}

\subsubsection{全位运算 FWT}

每一位都是不同运算法则的 FWT.

分别是 c(0, 0), c(0, 1), c(1, 0), c(1, 1)

\paragraph{算法一}

n = 17 (没有取模,可能爆longlong) 洛谷800ms

\begin{framed}
\begin{lstlisting}
#include <bits/stdc++.h>
#define ll long long
using namespace std;
char s[101][101];
int num[101];
vector<ll> gao(vector<ll> &a, vector<ll> & b, int cur) {
    int idx = num[cur];
    vector<ll> aa, bb, ret;
    int N = (1 << cur);
    for (int i = 0; i < N * 2; i ++) ret.push_back(0);
    if (cur == 0) {
        for (int i = 0; i < 2; i ++)
            for (int j = 0; j < 2; j ++)
                ret[s[cur][i * 2 + j] - '0'] += a[i] * b[j];
        return ret;
    }
    bool flg = false;
    for (int i = 0; i < N; i ++) {
        aa.push_back(a[i] + a[i + N]);
        bb.push_back(b[i] + b[i + N]);
    }
    vector<ll> tmp = gao(aa, bb, cur - 1);
 
    if (idx >= 8) idx = 15 - idx, flg = true;
    if (idx == 0) { // count except highest bit
        for (int i = 0; i < N; i ++) ret[i] = tmp[i];
    }
    else if (idx == 1) { 
        for (int i = 0; i < N; i ++)
            aa[i] = a[i], bb[i] = b[i];
        vector<ll> tmp2 = gao(aa, bb, cur - 1);
        for (int i = 0; i < N; i ++) {
            ret[i + N] = tmp2[i];
            ret[i] = tmp[i] - tmp2[i];
        }
    }
    else if (idx == 2) {
        for (int i = 0; i < N; i ++)
            aa[i] = a[i], bb[i] = b[i + N];
        vector<ll> tmp2 = gao(aa, bb, cur - 1);
        for (int i = 0; i < N; i ++) {
            ret[i] = tmp[i] - tmp2[i];
            ret[i + N] = tmp2[i];
        }
    }
    else if (idx == 3) {
        for (int i = 0; i < N; i ++)
            aa[i] = a[i];
        vector<ll> tmp2 = gao(aa, bb, cur - 1);
        for (int i = 0; i < N; i ++) {
            ret[i] = tmp[i] - tmp2[i];
            ret[i + N] = tmp2[i];
        }
    }
    else if (idx == 4) {
        for (int i = 0; i < N; i ++)
            aa[i] = a[i + N], bb[i] = b[i];
        vector<ll> tmp2 = gao(aa, bb, cur - 1);
        for (int i = 0; i < N; i ++) {
            ret[i] = tmp[i] - tmp2[i];
            ret[i + N] = tmp2[i];
        }
    }
    else if (idx == 5) {
        for (int i = 0; i < N; i ++)
            bb[i] = b[i];
        vector<ll> tmp2 = gao(aa, bb, cur - 1);
        for (int i = 0; i < N; i ++) {
            ret[i] = tmp[i] - tmp2[i];
            ret[i + N] = tmp2[i];
        }
    }
    else if (idx == 6) {
        for (int i = 0; i < N; i ++)
            aa[i] = a[i] - a[i + N], bb[i] = b[i] - b[i + N];
        vector<ll> tmp2 = gao(aa, bb, cur - 1);
        for (int i = 0; i < N; i ++) {
            ret[i] = (tmp[i] + tmp2[i]) / 2;
            ret[i + N] = (tmp[i] - tmp2[i]) / 2;
        }
    }
    else if (idx == 7) {
        for (int i = 0; i < N; i ++)
            aa[i] = a[i + N], bb[i] = b[i + N];
        vector<ll> tmp2 = gao(aa, bb, cur - 1);
 
        for (int i = 0; i < N; i ++) {
            ret[i] = tmp2[i];
            ret[i + N] = (tmp[i] - tmp2[i]);
        }
    }
    if (flg == true) {
        for (int i = 0; i < N; i ++)
            swap(ret[i], ret[i + N]);
    }
    return ret;
}
 
int main( ) {
    int n;
    long long d;
    scanf("%d", &n);
    for (int i = 0; i < n; i ++) {
        scanf("%s", s[i]);
        for (int j = 0; j < 4; j ++)
            num[i] += ((s[i][j] - '0') << j);
    }
    //c(0,0), c(0,1), c(1,0), c(1,1)
    vector<ll> a, b;
    for (int i = 0; i < (1 << n); i ++) {
        scanf("%lld", &d);
        a.push_back(d);
    }
    for (int i = 0; i < (1 << n); i ++) {
        scanf("%lld", &d);
        b.push_back(d);
    }
    vector<ll> ret = gao(a, b, n - 1);
    for (int i = 0; i < (1 << n); i ++)
        printf("%lld ", ret[i]);
    puts("");
    return 0;
}
\end{lstlisting}
\end{framed}

\paragraph{算法二}

130ms,但是不保证取模时是否会出锅

\begin{framed}
\begin{lstlisting}
#include<bits/stdc++.h>
using namespace std;
typedef long long ll;
const int mod = 998244353;
int id[16][3] = {
    {0, 0, 61},
    {1, 1, 77},
    {1, 3, 77},
    {1, 0, 77},
    {3, 1, 77},
    {0, 1, 77},
    {2, 2, 78},
    {3, 3, 79},
    {3, 3, 77},
    {2, 2, 74},
    {0, 1, 79},
    {3, 1, 79},
    {1, 0, 79},
    {1, 3, 79},
    {1, 1, 79},
    {0, 0, 23}
};
int fun[16][2][2], mt[81][2][2], li[100];
int chk(int x) {
    return mt[x][0][0] * mt[x][1][1] - mt[x][0][1] * mt[x][1][0];   
}
int getinv(int dst[][2], int src[][2]) {
    dst[0][0] = src[1][1], dst[1][1] = src[0][0];
    dst[0][1] = -src[0][1], dst[1][0] = -src[1][0];
    return src[0][0] * src[1][1] - src[0][1] * src[1][0];
}
long long km(long long now,long long k) {
	if(k == 0) return 1;
	long long ret = km(now,k/2);
	ret = ret * ret % mod;
	if(k % 2 == 1) {
		ret = ret * now % mod;
	}
	return ret;
}

int getinv(long long now) {
	if(now <= mod) now += mod;
	return km(now,mod-2)%mod;
}
void FWT(ll *a, int len, int op) {
    ll x, y, D = 1;
    for(int h = 1, b = 0; h < len; h <<= 1, ++b) {
        auto mat = mt[id[li[b]][op]];
        for(int i = 0; i < len; i += (h << 1)) {
            for(int j = 0; j < h; ++j) {
                x = a[i+j], y = a[i+j+h];
                if(op <= 1) {
                    a[i+j] = (mat[0][0] * x % mod + mat[0][1] * y)%mod;
                    a[i+j+h] = (mat[1][0] * x % mod + mat[1][1] * y)%mod;
                } else {
                    a[i+j] = ((mat[1][1] * x % mod) - mat[0][1] * y)%mod;
                    a[i+j+h] = ((-mat[1][0] * x % mod) + mat[0][0] * y)%mod;
                }
            }
        }
        D *= (mat[0][0] * mat[1][1] % mod - mat[0][1] * mat[1][0])%mod;
        D %=mod;
    }
    int invD =  getinv(D);
    //过了洛谷的 fwt 模板题,暂时不考虑 D 可能为 mod 倍数的情况 
    if(op == 2) {
    	//如果不取模,此处用 /D , D 不为 0 
        for(int i = 0; i < len; ++i) a[i] = a[i] * invD % mod;
    }
}

ll A[1<<19], B[1<<19], C[1<<19];
inline int pmod(long long now) {
	return (now+mod)%mod;
}
int main() {
    for(int i = 0; i < 16; ++i) {
        for(int j = 0; j < 4; ++j) {
            fun[i][j/2][j%2]= ((i>>j)&1);
        }
    }
    int tot = 0;
    for(int a = -1; a <= 1; ++a) 
        for(int b = -1; b <= 1; ++b)
            for(int c = -1; c <= 1; ++c)
                for(int d = -1; d <= 1; ++d) {
                    mt[tot][0][0] = a;
                    mt[tot][0][1] = b;
                    mt[tot][1][0] = c;
                    mt[tot][1][1] = d;
                    ++tot;
                }
    int n, len;
    cin >> n; len = (1<<n);
    string s;
    for(int i = 0; i < len; ++i){
		scanf("%lld", &A[i]);
		A[i] %= mod;
	}
    for(int i = 0; i < len; ++i){
		scanf("%lld", &B[i]);
		B[i] %= mod;
	}
    //c(0,0) c(0,1) c(1,0) c(1,1)  
    for(int i = 0; i < n; ++i) {
    	s = "0111";
        int x = 0;
        for(int j = 0; j < 4; ++j) {
            if(s[j]-'0') x |= (1<<j);
        }
        li[i] = x;
    }
    FWT(A, len, 0); FWT(B, len, 1);
    for(int i = 0; i < len; ++i) C[i] = A[i] * B[i];
    FWT(C, len, 2);
    for(int i = 0; i < len; ++i) printf("%d ", pmod(C[i]));
    printf("\n");
}
\end{lstlisting}
\end{framed}

\subsubsection{分治FFT}
给定序列 $g_{1...n-1}$,求序列 $f_{0...n-1}$

其中 $f_i = \sum_{j=1}^i f_{i-j}g_j$,边界为 $f_0=1$,答案对 $998244353$ 取模。

分治做法:
\begin{framed}
\begin{lstlisting}
#include <bits/stdc++.h>
using namespace std;

#define IL inline
#define ri register int 
#define dbg1(x) cout << #x << " = " << x << ", "
#define dbg2(x) cout << #x << " = " << x << endl

template<typename Tp> IL void read(Tp &x) {
    x=0; int f=1; char ch=getchar();
    while(!isdigit(ch)) {if(ch == '-') f=-1; ch=getchar();}
    while(isdigit(ch)) {x =  x*10+ch-'0'; ch=getchar();}
    x *= f;
}
int buf[22];
template<typename Tp> IL void write(Tp x) {
    int p = 0;
    if(x < 0) {putchar('-'); x=-x;}
    if(x == 0) buf[++p] = 0;
    else while(x) {
        buf[++p] = x % 10;
        x /= 10;
    }
    for(int i=p;i;i--) putchar('0' + buf[i]);
}

const int mod = 998244353;
const int G = 3;
const int N = 262144;

IL int ksm(int a, int b, int m) {
    int ret = 1;
    while(b) {
        if(b&1) ret = 1ll * ret * a % m;
        a = 1ll * a * a % m;
        b >>= 1;
    }
    return ret;
}
IL int inv(int x) {return ksm(x, mod-2, mod);}
const int inv2 = inv(2);

int rev[N], eps[N], ieps[N];
IL void initeps() {
    int g = ksm(G, (mod-1) / N, mod);
    ieps[0] = eps[0] = 1;
    for(int i=1;i<N;i++) eps[i] = 1ll * eps[i-1] * g % mod;
    for(int i=1;i<N;i++) ieps[i] = eps[N-i];
}

IL void trans(int *x, int *w, int n) {
    for(int i=0;i<n;i++) if(i < rev[i]) swap(x[i], x[rev[i]]);
    for(int i=2;i<=n;i<<=1) {
        int l = i >> 1, d = N / i;
        for(int j=0;j<n;j+=i) {
            for(int k=0;k<l;k++) {
                int t = 1ll * x[j+k+l] * w[d*k] % mod;
                x[j+k+l] = (x[j+k] - t + mod) % mod;
                x[j+k] = (x[j+k] + t) % mod;
            }
        }
    }
}

IL void dft(int *x, int n) { trans(x, eps, n);}
IL void idft(int *x, int n) { 
    trans(x, ieps, n);
    int in = inv(n);
    for(int i=0;i<n;i++) x[i] = 1ll * x[i] * in % mod;
}

int ntt_a[N], ntt_b[N], ntt_c[N];

IL void CDQ_NTT(int *B, int *A, int L, int R) {
    if(L == R) { if(L == 0) B[0] = 1; return;}
    int M = L+R >> 1;
    CDQ_NTT(B, A, L, M);
    int lim = 1, lglim = 0;
    while(lim <= (R-L+1)) {lim <<= 1; lglim++;}
    for(int i=0;i<lim;i++) rev[i] = (rev[i>>1] >> 1) | ((i&1) << (lglim-1));
    for(int i=0;i+L<=M;i++) ntt_a[i] = B[i+L];
    for(int i=M-L+1;i<lim;i++) ntt_a[i] = 0;
    for(int i=0;i+L<=R;i++) ntt_b[i] = A[i];
    for(int i=R-L+1;i<lim;i++) ntt_b[i] = 0;
    dft(ntt_a, lim); dft(ntt_b, lim);
    for(int i=0;i<lim;i++) ntt_a[i] = 1ll * ntt_a[i] * ntt_b[i] % mod;
    idft(ntt_a, lim);
    for(int i=M+1;i<=R;i++) B[i] = (B[i] + ntt_a[i-L]) % mod;
    CDQ_NTT(B, A, M+1, R);
}

int n;
int f[N], ans[N];

int main() {
    initeps();
    read(n); n--;
    for(int i=1;i<=n;i++) read(f[i]);
    CDQ_NTT(ans, f, 0, n);
    for(int i=0;i<=n;i++) {write(ans[i]); putchar(" \n"[i==n]);}
    return 0;
}
\end{lstlisting}
\end{framed}

\subsubsection{任意模数多项式乘法}
3.5 次 FFT

\begin{framed}
\begin{lstlisting}
#include<bits/stdc++.h>
using namespace std;

typedef long long ll;
typedef double db;
typedef long double ld;
#define IL inline
#define dbg1(x) cerr << #x << " = " << x << ", "
#define dbg2(x) cerr << #x << " = " << x << endl

template<typename Tp> IL void read(Tp& x) {
    x=0; int f=1; char ch=getchar();
    while(!isdigit(ch)) {if(ch == '-') f=-1; ch=getchar();}
    while(isdigit(ch)) { x = x*10+ch-'0'; ch=getchar();}
    x *= f;
}
int buf[22];
template<typename Tp> IL void write(Tp x) {
    int p = 0;
    if(x < 0) {putchar('-'); x=-x;}
    if(x == 0) buf[++p] = 0;
    else while(x) {
        buf[++p] = x % 10;
        x /= 10;
    }
    for(int i=p;i;i--) putchar('0' + buf[i]);
}

struct cp {
    db x, y;
    cp(db x=0.0, db y=0.0):x(x), y(y) {}
    IL cp operator + (const cp& o) const { return cp(x+o.x, y+o.y);}
    IL cp operator - (const cp& o) const { return cp(x-o.x, y-o.y);}
    IL cp operator * (const cp& o) const { return cp(x*o.x-y*o.y, x*o.y+y*o.x);}
    IL cp operator * (const db& p) const { return cp(x*p, y*p);}
    IL cp operator / (const db& p) const { return cp(x/p, y/p);}
    IL cp operator ! () const { return cp(x, -y);}
};
IL cp polar(const db& rho, const db& theta) {return cp(rho*cos(theta), rho*sin(theta));}

const int N = 131072;
const db PI = acos(-1.0);

int rev[N];
cp eps[N], ieps[N];
IL void initeps() {
    for(int i=0;i<N;i++) eps[i] = polar(1.0, 2*PI*i/N);
    ieps[0] = eps[0] = 1;
    for(int i=1;i<N;i++) ieps[i] = eps[N-i];
}
IL void cal_rev(int degA, int degB, int& lim, int &p) {
    lim = 1; p = 0;
    while(lim <=  ((degA+degB) >> 1)) {lim <<= 1; p++;}
    for(int i=0;i<lim;i++) rev[i] = (rev[i>>1]>>1) | ((i&1) << (p-1));
}
IL void trans(cp *x, cp *w, int n) {
    for(int i=0;i<n;i++) if(i < rev[i]) swap(x[i], x[rev[i]]);
    for(int i=2;i<=n;i<<=1) {
        int l = i >> 1, d = N / i;
        for(int j=0;j<n;j+=i) {
            for(int k=0;k<l;k++) {
                cp t = x[j+k+l] * w[d*k];
                x[j+k+l] = x[j+k] - t;
                x[j+k] = x[j+k] + t;
            }
        }
    }
}
IL void dft(cp *x, int n) {trans(x, eps, n);}
IL void idft(cp *x, int n) {trans(x, ieps, n); for(int i=0;i<n;i++) x[i] = x[i] / n;}

int n, m, P;

int f[N], g[N], ans[N<<1];
cp f1[N], f2[N], g1[N], g2[N];
cp h1[N], h2[N], h3[N];

IL int solve(int *C, int *A, int *B, int degA, int degB, int mod) {
    int p = 31624; // sqrt(1e9 + 7) = 31622.77
    for(int i=0;i<=degA;i++) { 
        (i & 1 ? f1[i>>1].y : f1[i>>1].x) = A[i] / p; 
        (i & 1 ? f2[i>>1].y : f2[i>>1].x) = A[i] % p;
    }
    for(int i=0;i<=degB;i++) { 
        (i & 1 ? g1[i>>1].y : g1[i>>1].x) = B[i] / p; 
        (i & 1 ? g2[i>>1].y : g2[i>>1].x) = B[i] % p;
    }
    int lim, lglim;
    cal_rev(degA, degB, lim, lglim);
    for(int i=degA+1;i<(lim << 1);i++) {
        (i & 1 ? f1[i>>1].y : f1[i>>1].x) = 0; 
        (i & 1 ? f2[i>>1].y : f2[i>>1].x) = 0;
    }
    for(int i=degB+1;i<(lim << 1);i++) {
        (i & 1 ? g1[i>>1].y : g1[i>>1].x) = 0; 
        (i & 1 ? g2[i>>1].y : g2[i>>1].x) = 0;
    }
    dft(f1, lim); dft(f2, lim); dft(g1, lim); dft(g2, lim);
    // X = f1 * p + f2
    // Y = g1 * p + g2
    // Z = (f1*g1)p^2 + (f1*g2+f2*g1)p + f2*g2
    int d = N / lim;
    for(int i=0;i<lim;i++) {
        int j = (lim-1) & (lim-i);
        h1[i] = (f1[i] * g1[i] * 4 - (f1[i]-!f1[j]) * 
                (g1[i]-!g1[j]) * (eps[d*i] + cp(1,0))) * 0.25;
        h2[i] = (f1[i] * g2[i] * 4 - (f1[i]-!f1[j]) * 
                (g2[i]-!g2[j]) * (eps[d*i] + cp(1,0))) * 0.25
               +(f2[i] * g1[i] * 4 - (f2[i]-!f2[j]) *
                (g1[i]-!g1[j]) * (eps[d*i] + cp(1,0))) * 0.25;
        h3[i] = (f2[i] * g2[i] * 4 - (f2[i]-!f2[j]) *
                (g2[i]-!g2[j]) * (eps[d*i] + cp(1,0))) * 0.25;
    }
    idft(h1, lim); idft(h2, lim); idft(h3, lim);
    for(int i=0;i<=degA+degB;i++) {
        ll h1v = ((i&1) ? h1[i>>1].y : h1[i>>1].x) + 0.5;
        ll h2v = ((i&1) ? h2[i>>1].y : h2[i>>1].x) + 0.5;
        ll h3v = ((i&1) ? h3[i>>1].y : h3[i>>1].x) + 0.5;
        h1v %= mod; h2v %= mod; h3v %= mod;
        // dbg1(i); dbg1(h1v); dbg1(h2v); dbg2(h3v);
        C[i] = (h1v*p%mod*p%mod + h2v*p%mod + h3v) % mod;
    }
    return degA + degB;
}

int main() {
    initeps();
    read(n); read(m); read(P);
    for(int i=0;i<=n;i++) read(f[i]);
    for(int i=0;i<=m;i++) read(g[i]);
    solve(ans, f, g, n, m, P);
    for(int i=0;i<=n+m;i++) {write(ans[i]); putchar(" \n"[i==n+m]);}
    return 0;
}
\end{lstlisting}
\end{framed}

\subsubsection{多项式乘法逆}

\begin{framed}
\begin{lstlisting}
#include <bits/stdc++.h>
using namespace std;

#define IL inline
#define ri register int 
#define dbg1(x) cout << #x << " = " << x << ", "
#define dbg2(x) cout << #x << " = " << x << endl

template<typename Tp> IL void read(Tp &x) {
    x=0; int f=1; char ch=getchar();
    while(!isdigit(ch)) {if(ch == '-') f=-1; ch=getchar();}
    while(isdigit(ch)) {x =  x*10+ch-'0'; ch=getchar();}
    x *= f;
}
int buf[22];
template<typename Tp> IL void write(Tp x) {
    int p = 0;
    if(x < 0) {putchar('-'); x=-x;}
    if(x == 0) buf[++p] = 0;
    else while(x) {
        buf[++p] = x % 10;
        x /= 10;
    }
    for(int i=p;i;i--) putchar('0' + buf[i]);
}

const int mod = 998244353;
const int G = 3;
const int N = 262144;

IL int ksm(int a, int b, int m) {
    int ret = 1;
    while(b) {
        if(b&1) ret = 1ll * ret * a % m;
        a = 1ll * a * a % m;
        b >>= 1;
    }
    return ret;
}
IL int inv(int x) {return ksm(x, mod-2, mod);}

int rev[N], eps[N], ieps[N];
IL void initeps() {
    int g = ksm(G, (mod-1) / N, mod);
    ieps[0] = eps[0] = 1;
    for(int i=1;i<N;i++) eps[i] = 1ll * eps[i-1] * g % mod;
    for(int i=1;i<N;i++) ieps[i] = eps[N-i];
}

IL void trans(int *x, int *w, int n) {
    for(int i=0;i<n;i++) if(i < rev[i]) swap(x[i], x[rev[i]]);
    for(int i=2;i<=n;i<<=1) {
        int l = i >> 1, d = N / i;
        for(int j=0;j<n;j+=i) {
            for(int k=0;k<l;k++) {
                int t = 1ll * x[j+k+l] * w[d*k] % mod;
                x[j+k+l] = (x[j+k] - t + mod) % mod;
                x[j+k] = (x[j+k] + t) % mod;
            }
        }
    }
}

IL void dft(int *x, int n) { trans(x, eps, n);}
IL void idft(int *x, int n) { 
    trans(x, ieps, n);
    int in = inv(n);
    for(int i=0;i<n;i++) x[i] = 1ll * x[i] * in % mod;
}

int ntt_a[N];
IL void polyinv(int *B, int *A, int degA) {
    if(degA == 0) {B[0] = inv(A[0]); return;}
    polyinv(B, A, degA >> 1);
    int lim = 1, lglim = 0;
    while(lim <= (degA<<1)) { lim <<= 1; lglim++;}
    for(int i=0;i<lim;i++) rev[i] = (rev[i>>1]>>1) | ((i&1) << (lglim-1));
    for(int i=0;i<=degA;i++) ntt_a[i] = A[i];
    for(int i=degA+1;i<lim;i++) ntt_a[i] = 0;
    dft(ntt_a, lim); dft(B, lim);
    for(int i=0;i<lim;i++) B[i] = 1ll*(2-1ll*ntt_a[i]*B[i]%mod+mod)%mod*B[i]%mod;
    idft(B, lim);
    for(int i=degA+1;i<lim;i++) B[i] = 0;
}

int n;
int f[N], ans[N];

int main() {
    initeps();
    read(n); n--;
    for(int i=0;i<=n;i++) read(f[i]);
    polyinv(ans, f, n);
    for(int i=0;i<=n;i++) {write(ans[i]); putchar(" \n"[i==n]);}
    return 0;
}
\end{lstlisting}
\end{framed}

\subsubsection{多项式对数函数}
\begin{framed}
\begin{lstlisting}
#include <bits/stdc++.h>
using namespace std;

#define IL inline
#define ri register int 
#define dbg1(x) cout << #x << " = " << x << ", "
#define dbg2(x) cout << #x << " = " << x << endl

template<typename Tp> IL void read(Tp &x) {
    x=0; int f=1; char ch=getchar();
    while(!isdigit(ch)) {if(ch == '-') f=-1; ch=getchar();}
    while(isdigit(ch)) {x =  x*10+ch-'0'; ch=getchar();}
    x *= f;
}
int buf[22];
template<typename Tp> IL void write(Tp x) {
    int p = 0;
    if(x < 0) {putchar('-'); x=-x;}
    if(x == 0) buf[++p] = 0;
    else while(x) {
        buf[++p] = x % 10;
        x /= 10;
    }
    for(int i=p;i;i--) putchar('0' + buf[i]);
}

const int mod = 998244353;
const int G = 3;
const int N = 262144;

IL int ksm(int a, int b, int m) {
    int ret = 1;
    while(b) {
        if(b&1) ret = 1ll * ret * a % m;
        a = 1ll * a * a % m;
        b >>= 1;
    }
    return ret;
}
IL int inv(int x) {return ksm(x, mod-2, mod);}

int rev[N], eps[N], ieps[N];
IL void initeps() {
    int g = ksm(G, (mod-1) / N, mod);
    ieps[0] = eps[0] = 1;
    for(int i=1;i<N;i++) eps[i] = 1ll * eps[i-1] * g % mod;
    for(int i=1;i<N;i++) ieps[i] = eps[N-i];
}

IL void trans(int *x, int *w, int n) {
    for(int i=0;i<n;i++) if(i < rev[i]) swap(x[i], x[rev[i]]);
    for(int i=2;i<=n;i<<=1) {
        int l = i >> 1, d = N / i;
        for(int j=0;j<n;j+=i) {
            for(int k=0;k<l;k++) {
                int t = 1ll * x[j+k+l] * w[d*k] % mod;
                x[j+k+l] = (x[j+k] - t + mod) % mod;
                x[j+k] = (x[j+k] + t) % mod;
            }
        }
    }
}

IL void dft(int *x, int n) { trans(x, eps, n);}
IL void idft(int *x, int n) { 
    trans(x, ieps, n);
    int in = inv(n);
    for(int i=0;i<n;i++) x[i] = 1ll * x[i] * in % mod;
}

int ntt_a[N], ntt_b[N], ntt_c[N];
IL void polyinv(int *B, int *A, int degA) {
    if(degA == 0) {B[0] = inv(A[0]); return;}
    polyinv(B, A, degA >> 1);
    int lim = 1, lglim = 0;
    while(lim <= (degA<<1)) { lim <<= 1; lglim++;}
    for(int i=0;i<lim;i++) rev[i] = (rev[i>>1]>>1) | ((i&1) << (lglim-1));
    for(int i=0;i<=degA;i++) ntt_a[i] = A[i];
    for(int i=degA+1;i<lim;i++) ntt_a[i] = 0;
    dft(ntt_a, lim); dft(B, lim);
    for(int i=0;i<lim;i++) B[i] = 1ll*(2-1ll*ntt_a[i]*B[i]%mod+mod)%mod*B[i]%mod;
    idft(B, lim);
    for(int i=degA+1;i<lim;i++) B[i] = 0;
}

IL void polyDao(int *B, int *A, int degA) {
    for(int i=0;i<degA;i++) B[i] = 1ll * (i+1) * A[i+1] % mod; B[degA] = 0;
}
IL void polyJifen(int *B, int *A, int degA) {
    for(int i=0;i<degA;i++) B[i+1] = 1ll * A[i] * inv(i+1) % mod; B[0] = 0;
}

IL void polyln(int *B, int *A, int degA) {
    polyDao(ntt_b, A, degA); polyinv(ntt_c, A, degA);
    int lim = 1, lglim = 0;
    while(lim <= (degA << 1)) {lim <<= 1; lglim++;}
    for(int i=0;i<lim;i++) rev[i] = (rev[i>>1] >> 1) | ((i&1) << (lglim-1));
    for(int i=degA+1;i<lim;i++) ntt_b[i] = 0;
    for(int i=degA+1;i<lim;i++) ntt_c[i] = 0;
    dft(ntt_b, lim); dft(ntt_c, lim);
    for(int i=0;i<lim;i++) ntt_b[i] = 1ll * ntt_b[i] * ntt_c[i] % mod;
    idft(ntt_b, lim);
    polyJifen(B, ntt_b, degA);
}

int n;
int f[N], ans[N];

int main() {
    initeps();
    read(n); n--;
    for(int i=0;i<=n;i++) read(f[i]);
    polyln(ans, f, n);
    for(int i=0;i<=n;i++) {write(ans[i]); putchar(" \n"[i==n]);}
    return 0;
}
\end{lstlisting}
\end{framed}

\subsubsection{多项式指数函数}
\begin{framed}
\begin{lstlisting}
#include <bits/stdc++.h>
using namespace std;

#define IL inline
#define ri register int 
#define dbg1(x) cout << #x << " = " << x << ", "
#define dbg2(x) cout << #x << " = " << x << endl

template<typename Tp> IL void read(Tp &x) {
    x=0; int f=1; char ch=getchar();
    while(!isdigit(ch)) {if(ch == '-') f=-1; ch=getchar();}
    while(isdigit(ch)) {x =  x*10+ch-'0'; ch=getchar();}
    x *= f;
}
int buf[22];
template<typename Tp> IL void write(Tp x) {
    int p = 0;
    if(x < 0) {putchar('-'); x=-x;}
    if(x == 0) buf[++p] = 0;
    else while(x) {
        buf[++p] = x % 10;
        x /= 10;
    }
    for(int i=p;i;i--) putchar('0' + buf[i]);
}

const int mod = 998244353;
const int G = 3;
const int N = 262144;

IL int ksm(int a, int b, int m) {
    int ret = 1;
    while(b) {
        if(b&1) ret = 1ll * ret * a % m;
        a = 1ll * a * a % m;
        b >>= 1;
    }
    return ret;
}
IL int inv(int x) {return ksm(x, mod-2, mod);}

int rev[N], eps[N], ieps[N];
IL void initeps() {
    int g = ksm(G, (mod-1) / N, mod);
    ieps[0] = eps[0] = 1;
    for(int i=1;i<N;i++) eps[i] = 1ll * eps[i-1] * g % mod;
    for(int i=1;i<N;i++) ieps[i] = eps[N-i];
}

IL void trans(int *x, int *w, int n) {
    for(int i=0;i<n;i++) if(i < rev[i]) swap(x[i], x[rev[i]]);
    for(int i=2;i<=n;i<<=1) {
        int l = i >> 1, d = N / i;
        for(int j=0;j<n;j+=i) {
            for(int k=0;k<l;k++) {
                int t = 1ll * x[j+k+l] * w[d*k] % mod;
                x[j+k+l] = (x[j+k] - t + mod) % mod;
                x[j+k] = (x[j+k] + t) % mod;
            }
        }
    }
}

IL void dft(int *x, int n) { trans(x, eps, n);}
IL void idft(int *x, int n) { 
    trans(x, ieps, n);
    int in = inv(n);
    for(int i=0;i<n;i++) x[i] = 1ll * x[i] * in % mod;
}

int ntt_a[N], ntt_b[N], ntt_c[N];
IL void polyinv(int *B, int *A, int degA) { // use ntt_a
    if(degA == 0) {B[0] = inv(A[0]); B[1] = 0; return;}
    polyinv(B, A, degA >> 1);
    int lim = 1, lglim = 0;
    while(lim <= (degA<<1)) { lim <<= 1; lglim++;}
    for(int i=0;i<lim;i++) rev[i] = (rev[i>>1]>>1) | ((i&1) << (lglim-1));
    for(int i=0;i<=degA;i++) ntt_a[i] = A[i];
    for(int i=degA+1;i<lim;i++) ntt_a[i] = B[i] = 0;
    dft(ntt_a, lim); dft(B, lim);
    for(int i=0;i<lim;i++) B[i] = 1ll*(2-1ll*ntt_a[i]*B[i]%mod+mod)%mod*B[i]%mod;
    idft(B, lim);
    for(int i=degA+1;i<lim;i++) B[i] = 0;
}

IL void polyDao(int *B, int *A, int degA) {
    for(int i=0;i<degA;i++) B[i] = 1ll * (i+1) * A[i+1] % mod; B[degA] = 0;
}
IL void polyJifen(int *B, int *A, int degA) {
    for(int i=0;i<degA;i++) B[i+1] = 1ll * A[i] * inv(i+1) % mod; B[0] = 0;
}

IL void polyln(int *B, int *A, int degA) { // use ntt_a, ntt_b, ntt_c
    polyDao(ntt_b, A, degA); polyinv(ntt_c, A, degA);
    int lim = 1, lglim = 0;
    while(lim <= (degA << 1)) {lim <<= 1; lglim++;}
    for(int i=0;i<lim;i++) rev[i] = (rev[i>>1] >> 1) | ((i&1) << (lglim-1));
    for(int i=degA+1;i<lim;i++) ntt_b[i] = 0;
    for(int i=degA+1;i<lim;i++) ntt_c[i] = 0;
    dft(ntt_b, lim); dft(ntt_c, lim);
    for(int i=0;i<lim;i++) ntt_b[i] = 1ll * ntt_b[i] * ntt_c[i] % mod;
    idft(ntt_b, lim);
    polyJifen(B, ntt_b, degA);
}

int ntt_d[N];
IL void polyexp(int *B, int *A, int degA) {
    if(degA == 0) { B[0] = 1; B[1] = 0; return;}
    polyexp(B, A, degA >> 1);
    polyln(ntt_d, B, degA); // ntt_d = ln(B);
    int lim = 1, lglim = 0;
    while(lim <= (degA << 1)) { lim <<= 1; lglim++;}
    for(int i=0;i<lim;i++) rev[i] = (rev[i >> 1] >> 1) | ((i & 1) << (lglim - 1));
    for(int i=0;i<=degA;i++) ntt_d[i] = A[i] >= ntt_d[i] ? A[i] - ntt_d[i] : A[i] - ntt_d[i] + mod;
    ntt_d[0] = ntt_d[0] + 1 >= mod ? 0 : ntt_d[0] + 1;
    for(int i=degA+1;i<lim;i++) ntt_d[i] = B[i] = 0;
    dft(ntt_d, lim); dft(B, lim);
    for(int i=0;i<lim;i++) B[i] = 1ll * ntt_d[i] * B[i] % mod;
    idft(B, lim);
    for(int i=degA+1;i<lim;i++) B[i] = 0;
}

int n;
int f[N], ans[N];

int main() {
    initeps();
    read(n); n--;
    for(int i=0;i<=n;i++) read(f[i]);
    polyexp(ans, f, n);
    for(int i=0;i<=n;i++) {write(ans[i]); putchar(" \n"[i==n]);}
    return 0;
}
\end{lstlisting}
\end{framed}

\subsubsection{多项式快速幂}
没有限制的多项式快速幂。

\begin{framed}
\begin{lstlisting}
#include <bits/stdc++.h>
using namespace std;

#define IL inline
#define ri register int 
#define dbg1(x) cout << #x << " = " << x << ", "
#define dbg2(x) cout << #x << " = " << x << endl

template<typename Tp> IL void read(Tp &x) {
    x=0; int f=1; char ch=getchar();
    while(!isdigit(ch)) {if(ch == '-') f=-1; ch=getchar();}
    while(isdigit(ch)) {x =  x*10+ch-'0'; ch=getchar();}
    x *= f;
}
int buf[22];
template<typename Tp> IL void write(Tp x) {
    int p = 0;
    if(x < 0) {putchar('-'); x=-x;}
    if(x == 0) buf[++p] = 0;
    else while(x) {
        buf[++p] = x % 10;
        x /= 10;
    }
    for(int i=p;i;i--) putchar('0' + buf[i]);
}

const int mod = 998244353;
const int G = 3;
const int N = 262144;

IL int ksm(int a, int b, int m) {
    int ret = 1;
    while(b) {
        if(b&1) ret = 1ll * ret * a % m;
        a = 1ll * a * a % m;
        b >>= 1;
    }
    return ret;
}
IL int inv(int x) {return ksm(x, mod-2, mod);}

int rev[N], eps[N], ieps[N];
IL void initeps() {
    int g = ksm(G, (mod-1) / N, mod);
    ieps[0] = eps[0] = 1;
    for(int i=1;i<N;i++) eps[i] = 1ll * eps[i-1] * g % mod;
    for(int i=1;i<N;i++) ieps[i] = eps[N-i];
}

IL void trans(int *x, int *w, int n) {
    for(int i=0;i<n;i++) if(i < rev[i]) swap(x[i], x[rev[i]]);
    for(int i=2;i<=n;i<<=1) {
        int l = i >> 1, d = N / i;
        for(int j=0;j<n;j+=i) {
            for(int k=0;k<l;k++) {
                int t = 1ll * x[j+k+l] * w[d*k] % mod;
                x[j+k+l] = (x[j+k] - t + mod) % mod;
                x[j+k] = (x[j+k] + t) % mod;
            }
        }
    }
}

IL void dft(int *x, int n) { trans(x, eps, n);}
IL void idft(int *x, int n) { 
    trans(x, ieps, n);
    int in = inv(n);
    for(int i=0;i<n;i++) x[i] = 1ll * x[i] * in % mod;
}

int ntt_a[N], ntt_b[N], ntt_c[N];
IL void polyinv(int *B, int *A, int degA) { // use ntt_a
    if(degA == 0) {B[0] = inv(A[0]); B[1] = 0; return;}
    polyinv(B, A, degA >> 1);
    int lim = 1, lglim = 0;
    while(lim <= (degA<<1)) { lim <<= 1; lglim++;}
    for(int i=0;i<lim;i++) rev[i] = (rev[i>>1]>>1) | ((i&1) << (lglim-1));
    for(int i=0;i<=degA;i++) ntt_a[i] = A[i];
    for(int i=degA+1;i<lim;i++) ntt_a[i] = B[i] = 0;
    dft(ntt_a, lim); dft(B, lim);
    for(int i=0;i<lim;i++) B[i] = 1ll*(2-1ll*ntt_a[i]*B[i]%mod+mod)%mod*B[i]%mod;
    idft(B, lim);
    for(int i=degA+1;i<lim;i++) B[i] = 0;
}

IL void polyDao(int *B, int *A, int degA) {
    for(int i=0;i<degA;i++) B[i] = 1ll * (i+1) * A[i+1] % mod; B[degA] = 0;
}
IL void polyJifen(int *B, int *A, int degA) {
    for(int i=0;i<degA;i++) B[i+1] = 1ll * A[i] * inv(i+1) % mod; B[0] = 0;
}

IL void polyln(int *B, int *A, int degA) { // use ntt_a, ntt_b, ntt_c
    polyDao(ntt_b, A, degA); polyinv(ntt_c, A, degA);
    int lim = 1, lglim = 0;
    while(lim <= (degA << 1)) {lim <<= 1; lglim++;}
    for(int i=0;i<lim;i++) rev[i] = (rev[i>>1] >> 1) | ((i&1) << (lglim-1));
    for(int i=degA+1;i<lim;i++) ntt_b[i] = 0;
    for(int i=degA+1;i<lim;i++) ntt_c[i] = 0;
    dft(ntt_b, lim); dft(ntt_c, lim);
    for(int i=0;i<lim;i++) ntt_b[i] = 1ll * ntt_b[i] * ntt_c[i] % mod;
    idft(ntt_b, lim);
    polyJifen(B, ntt_b, degA);
}

int ntt_d[N];
IL void polyexp(int *B, int *A, int degA) { // use ntt_d
    if(degA == 0) { B[0] = 1; B[1] = 0; return;}
    polyexp(B, A, degA >> 1);
    polyln(ntt_d, B, degA); // ntt_d = ln(B);
    int lim = 1, lglim = 0;
    while(lim <= (degA << 1)) { lim <<= 1; lglim++;}
    for(int i=0;i<lim;i++) rev[i] = (rev[i >> 1] >> 1) | ((i & 1) << (lglim - 1));
    for(int i=0;i<=degA;i++) ntt_d[i] = A[i] >= ntt_d[i] ? A[i] - ntt_d[i] : A[i] - ntt_d[i] + mod;
    ntt_d[0] = ntt_d[0] + 1 >= mod ? 0 : ntt_d[0] + 1;
    for(int i=degA+1;i<lim;i++) ntt_d[i] = B[i] = 0;
    dft(ntt_d, lim); dft(B, lim);
    for(int i=0;i<lim;i++) B[i] = 1ll * ntt_d[i] * B[i] % mod;
    idft(B, lim);
    for(int i=degA+1;i<lim;i++) B[i] = 0;
}

int ntt_e[N], ntt_f[N];
IL void polyksm(int *B, int *A, int degA, int b, int b_mod_phi, char *ori_b) {
    int shift = 0; for(int i=0;i<=degA&&A[i]==0;i++) { shift++;}
    if(1ll * shift * b > degA || (shift && strlen(ori_b) >= 6)) { for(int i=0;i<=degA;i++) B[i] = 0; return;}
    // 100000 -> strlen = 6.

    int u = inv(A[shift]), v = ksm(A[shift], b_mod_phi, mod);
    for(int i=0;i+shift<=degA;i++) ntt_e[i] = 1ll * A[i+shift] * u % mod;
    for(int i=degA-shift+1;i<=degA;i++) ntt_e[i] = 0;

    polyln(ntt_f, ntt_e, degA);
    for(int i=0;i<=degA;i++) ntt_f[i] = (1ll * b * ntt_f[i]) % mod;
    polyexp(ntt_e, ntt_f, degA);

    shift *= b;
    for(int i=0;i<shift;i++) B[i] = 0;
    for(int i=shift;i<=degA;i++) B[i] = 1ll * ntt_e[i-shift] * v % mod;
}

int len;
char num[N];
template<typename Tp> IL void readm(Tp& x, Tp &y) {
    x = 0; y = 0;
    for(int i=0;i<len;i++) {
        x = (x * 10ll + num[i] - '0') % mod;
        y = (y * 10ll + num[i] - '0') % (mod - 1);
    }
}

int n, k, kk;
int f[N], ans[N];

int main() {
    initeps();
    read(n); n--; scanf("%s", num); len = strlen(num);
    readm(k, kk);
    for(int i=0;i<=n;i++) read(f[i]);
    polyksm(ans, f, n, k, kk, num);
    for(int i=0;i<=n;i++) {write(ans[i]); putchar(" \n"[i==n]);}
    return 0;
}
\end{lstlisting}
\end{framed}

\subsubsection{多项式开根}
结果返回的是常数项更小的结果。

\begin{framed}
\begin{lstlisting}
#include <bits/stdc++.h>
using namespace std;

#define IL inline
#define ri register int 
#define dbg1(x) cout << #x << " = " << x << ", "
#define dbg2(x) cout << #x << " = " << x << endl

template<typename Tp> IL void read(Tp &x) {
    x=0; int f=1; char ch=getchar();
    while(!isdigit(ch)) {if(ch == '-') f=-1; ch=getchar();}
    while(isdigit(ch)) {x =  x*10+ch-'0'; ch=getchar();}
    x *= f;
}
int buf[22];
template<typename Tp> IL void write(Tp x) {
    int p = 0;
    if(x < 0) {putchar('-'); x=-x;}
    if(x == 0) buf[++p] = 0;
    else while(x) {
        buf[++p] = x % 10;
        x /= 10;
    }
    for(int i=p;i;i--) putchar('0' + buf[i]);
}

const int mod = 998244353;
const int G = 3;
const int N = 262144;

IL int ksm(int a, int b, int m) {
    int ret = 1;
    while(b) {
        if(b&1) ret = 1ll * ret * a % m;
        a = 1ll * a * a % m;
        b >>= 1;
    }
    return ret;
}
IL int inv(int x) {return ksm(x, mod-2, mod);}
const int inv2 = inv(2);

int rev[N], eps[N], ieps[N];
IL void initeps() {
    int g = ksm(G, (mod-1) / N, mod);
    ieps[0] = eps[0] = 1;
    for(int i=1;i<N;i++) eps[i] = 1ll * eps[i-1] * g % mod;
    for(int i=1;i<N;i++) ieps[i] = eps[N-i];
}

IL void trans(int *x, int *w, int n) {
    for(int i=0;i<n;i++) if(i < rev[i]) swap(x[i], x[rev[i]]);
    for(int i=2;i<=n;i<<=1) {
        int l = i >> 1, d = N / i;
        for(int j=0;j<n;j+=i) {
            for(int k=0;k<l;k++) {
                int t = 1ll * x[j+k+l] * w[d*k] % mod;
                x[j+k+l] = (x[j+k] - t + mod) % mod;
                x[j+k] = (x[j+k] + t) % mod;
            }
        }
    }
}

IL void dft(int *x, int n) { trans(x, eps, n);}
IL void idft(int *x, int n) { 
    trans(x, ieps, n);
    int in = inv(n);
    for(int i=0;i<n;i++) x[i] = 1ll * x[i] * in % mod;
}

int ntt_a[N], ntt_b[N], ntt_c[N];
IL void polyinv(int *B, int *A, int degA) { // use ntt_a
    if(degA == 0) {B[0] = inv(A[0]); B[1] = 0; return;}
    polyinv(B, A, degA >> 1);
    int lim = 1, lglim = 0;
    while(lim <= (degA<<1)) { lim <<= 1; lglim++;}
    for(int i=0;i<lim;i++) rev[i] = (rev[i>>1]>>1) | ((i&1) << (lglim-1));
    for(int i=0;i<=degA;i++) ntt_a[i] = A[i];
    for(int i=degA+1;i<lim;i++) ntt_a[i] = B[i] = 0;
    dft(ntt_a, lim); dft(B, lim);
    for(int i=0;i<lim;i++) B[i] = 1ll*(2-1ll*ntt_a[i]*B[i]%mod+mod)%mod*B[i]%mod;
    idft(B, lim);
    for(int i=degA+1;i<lim;i++) B[i] = 0;
}

int cp2i2;
struct cp2 {
    int x, y;
    cp2(int x=0, int y=0):x(x), y(y) {}
};
IL cp2 mul(cp2 a, cp2 b, int p) {
    cp2 ret(0,0);
    ret.x = ((1ll * a.x * b.x % p + 1ll * a.y * b.y % p * cp2i2 % p) % p + p) % p;;
    ret.y = (1ll * a.x * b.y % p + 1ll * a.y * b.x % p) % p;
    return ret;
}
IL int cp2ksm(cp2 a, int b, int p) {
    cp2 ret(1,0);
    while(b) {
        if(b&1) ret = mul(ret, a, p);
        a = mul(a, a, p);
        b >>= 1;
    }
    return ret.x;
}

IL int cipolla(int n, int p, int &x1, int &x2) {
    n %= p;
    if(p == 2) return n;
    if(n == 0) {x1 = x2 = 0; return 0;}
    if(ksm(n, (p-1) / 2, p) == p - 1) return -1;
    int a;
    while(1) {
        a = ((rand() << 15) | rand()) % p;
        cp2i2 = (1ll * a * a % p - n + p) % p;
        if(ksm(cp2i2, (p-1) >> 1, p) == p - 1) break;
    }
    cp2 x(a, 1);
    x1 = cp2ksm(x, (p+1) >> 1, p);
    x2 = p - x1;
    if(x1 > x2) swap(x1, x2);
    return 0;
}

IL void polysqrt(int *B, int *A, int degA) { // use ntt_b, 
    if(degA == 0) { int x1, x2; cipolla(A[0], mod, x1, x2); B[0] = x1; B[1] = 0; return;}
    polysqrt(B, A, degA >> 1);
    polyinv(ntt_b, B, degA);

    int lim = 1, lglim = 0;
    while(lim <= (degA << 1)) { lim <<= 1; lglim++;}
    for(int i=0;i<lim;i++) rev[i] = (rev[i>>1] >> 1) | ((i&1) << (lglim-1));
    for(int i=0;i<=degA;i++) ntt_c[i] = A[i];
    for(int i=degA+1;i<lim;i++) B[i] = ntt_b[i] = ntt_c[i] = 0;

    dft(ntt_b, lim); dft(ntt_c, lim); dft(B, lim);
    for(int i=0;i<lim;i++) B[i] = (1ll * inv2 * B[i] % mod + 1ll * inv2 * ntt_b[i] % mod * ntt_c[i]) % mod;
    idft(B, lim);
    for(int i=degA+1;i<lim;i++) B[i] = 0;
}

int n, k, kk;
int f[N], ans[N];

int main() {
    initeps();
    read(n); n--;
    for(int i=0;i<=n;i++) read(f[i]);
    polysqrt(ans, f, n);
    for(int i=0;i<=n;i++) {write(ans[i]); putchar(" \n"[i==n]);}
    return 0;
}
\end{lstlisting}
\end{framed}

\subsubsection{多项式除法|取模}
\begin{framed}
\begin{lstlisting}
#include <bits/stdc++.h>
using namespace std;

#define IL inline
#define dbg1(x) cout << #x << " = " << x << ", "
#define dbg2(x) cout << #x << " = " << x << endl

template<typename Tp> IL void read(Tp &x) {
    x=0; int f=1; char ch=getchar();
    while(!isdigit(ch)) {if(ch == '-') f=-1; ch=getchar();}
    while(isdigit(ch)) {x =  x*10+ch-'0'; ch=getchar();}
    x *= f;
}
int buf[22];
template<typename Tp> IL void write(Tp x) {
    int p = 0;
    if(x < 0) {putchar('-'); x=-x;}
    if(x == 0) buf[++p] = 0;
    else while(x) {
        buf[++p] = x % 10;
        x /= 10;
    }
    for(int i=p;i;i--) putchar('0' + buf[i]);
}

const int N = 262144;
const int G = 3;
const int mod = 998244353;

IL int ksm(int a, int b, int p) {
    int ret = 1;
    while(b) {
        if(b&1) ret = 1ll * ret * a % p;
        a = 1ll * a * a % p;
        b >>= 1;
    }
    return ret;
}
IL int inv(int x) { return ksm(x, mod-2, mod);}

int rev[N], eps[N], ieps[N];

IL void initeps() {
    int g = ksm(G, (mod-1)/N, mod);
    eps[0] = ieps[0] = 1;
    for(int i=1;i<N;i++) eps[i] = 1ll * g * eps[i-1] % mod;
    for(int i=1;i<N;i++) ieps[i] = eps[N-i];
}
IL void trans(int *x, int *w, int n) {
    for(int i=0;i<n;i++) if(i < rev[i]) swap(x[i], x[rev[i]]);
    for(int i=2;i<=n;i<<=1) {
        int l = i >> 1, d = N / i;
        for(int j=0;j<n;j+=i) {
            for(int k=0;k<l;k++) {
                int t = 1ll * x[j+k+l] * w[d*k] % mod;
                x[j+k+l] = (x[j+k] - t + mod) % mod;
                x[j+k] = (x[j+k] + t) % mod;
            }
        }
    }
}
IL void dft(int *x, int n) { trans(x, eps, n);}
IL void idft(int *x, int n) {
    trans(x, ieps, n);
    int in = inv(n);
    for(int i=0;i<n;i++) x[i] = 1ll * x[i] * in % mod;
}

int ntt_a[N];
IL void polyinv(int *B, int *A, int degA) {
    if(degA == 0) {B[0] = inv(A[0]); B[1] = 0; return;}
    polyinv(B, A, degA >> 1);
    int lim = 1, lglim = 0;
    while(lim <= (degA<<1)) { lim <<= 1; lglim++;}
    for(int i=0;i<lim;i++) rev[i] = (rev[i>>1]>>1) | ((i&1) << (lglim-1));
    for(int i=0;i<=degA;i++) ntt_a[i] = A[i];
    for(int i=degA+1;i<lim;i++) ntt_a[i] = 0;
    dft(ntt_a, lim); dft(B, lim);
    for(int i=0;i<lim;i++) B[i] = 1ll*(2-1ll*ntt_a[i]*B[i]%mod+mod)%mod*B[i]%mod;
    idft(B, lim);
    for(int i=degA+1;i<lim;i++) B[i] = 0;
}

int ntt_b[N], ntt_c[N];
IL void polymul(int *C, int *A, int *B, int degA, int degB) {
    int lim = 1, lglim = 0;
    while(lim <= (degA + degB)) {lim <<= 1; lglim++;}
    for(int i=0;i<lim;i++) rev[i] = (rev[i>>1]>>1) | ((i&1) << (lglim-1));
    for(int i=0;i<=degA;i++) ntt_b[i] = A[i];
    for(int i=degA+1;i<lim;i++) ntt_b[i] = 0;
    for(int i=0;i<=degB;i++) ntt_c[i] = B[i];
    for(int i=degB+1;i<lim;i++) ntt_c[i] = 0;
    dft(ntt_b, lim); dft(ntt_c, lim);
    for(int i=0;i<lim;i++) ntt_b[i] = 1ll * ntt_b[i] * ntt_c[i] % mod;
    idft(ntt_b, lim);
    for(int i=0;i<=degA+degB;i++) C[i] = ntt_b[i];
    for(int i=degA+degB+1;i<lim;i++) C[i] = 0;
}

// C = A / B
// D = A % B
int ntt_d[N], ntt_e[N], ntt_f[N];
IL void polydiv(int *C, int *D, int *A, int *B, int degA, int degB) {
    int degC = degA - degB, degD = max(degB-1, 0);
    for(int i=0;i<=degA;i++) ntt_e[i] = A[degA-i];
    for(int i=0;i<=degB;i++) ntt_f[i] = B[degB-i];
    for(int i=degA-degB+1;i<=degB;i++) ntt_f[i] = 0;
    polyinv(ntt_d, ntt_f, degA-degB);
    polymul(ntt_f, ntt_d, ntt_e, degA-degB, degA);
    for(int i=0;i<=degA-degB;i++) C[i] = ntt_f[degA-degB-i];
    polymul(D, C, B, degA-degB, degB);
    for(int i=degB;i<=degA;i++) D[i] = 0;
    for(int i=0;i<=degB-1;i++) D[i] = (A[i] - D[i] + mod) % mod;
}

int n, m;
int Q[N], R[N], f[N], g[N];

int main() {
    initeps();
    read(n); read(m);
    for(int i=0;i<=n;i++) read(f[i]);
    for(int i=0;i<=m;i++) read(g[i]);
    polydiv(Q, R, f, g, n, m);
    for(int i=0;i<=n-m;i++) { write(Q[i]); putchar(" \n"[i==n-m]);}
    for(int i=0;i<=m-1;i++) { write(R[i]); putchar(" \n"[i==m-1]);}
    return 0;
}
\end{lstlisting}
\end{framed}

\subsubsection{下降幂多项式乘法}
\begin{framed}
\begin{lstlisting}
#include <bits/stdc++.h>
using namespace std;

#define IL inline
#define ri register int 
#define dbg1(x) cout << #x << " = " << x << ", "
#define dbg2(x) cout << #x << " = " << x << endl

template<typename Tp> IL void read(Tp &x) {
    x=0; int f=1; char ch=getchar();
    while(!isdigit(ch)) {if(ch == '-') f=-1; ch=getchar();}
    while(isdigit(ch)) {x =  x*10+ch-'0'; ch=getchar();}
    x *= f;
}
int buf[22];
template<typename Tp> IL void write(Tp x) {
    int p = 0;
    if(x < 0) {putchar('-'); x=-x;}
    if(x == 0) buf[++p] = 0;
    else while(x) {
        buf[++p] = x % 10;
        x /= 10;
    }
    for(int i=p;i;i--) putchar('0' + buf[i]);
}

const int mod = 998244353;
const int G = 3;
const int N = 524288;

IL int ksm(int a, int b, int m) {
    int ret = 1;
    while(b) {
        if(b&1) ret = 1ll * ret * a % m;
        a = 1ll * a * a % m;
        b >>= 1;
    }
    return ret;
}
IL int inv(int x) {return ksm(x, mod-2, mod);}
const int inv2 = inv(2);

int rev[N], eps[N], ieps[N];
IL void initeps() {
    int g = ksm(G, (mod-1) / N, mod);
    ieps[0] = eps[0] = 1;
    for(int i=1;i<N;i++) eps[i] = 1ll * eps[i-1] * g % mod;
    for(int i=1;i<N;i++) ieps[i] = eps[N-i];
}

IL void trans(int *x, int *w, int n) {
    for(int i=0;i<n;i++) if(i < rev[i]) swap(x[i], x[rev[i]]);
    for(int i=2;i<=n;i<<=1) {
        int l = i >> 1, d = N / i;
        for(int j=0;j<n;j+=i) {
            for(int k=0;k<l;k++) {
                int t = 1ll * x[j+k+l] * w[d*k] % mod;
                x[j+k+l] = (x[j+k] - t + mod) % mod;
                x[j+k] = (x[j+k] + t) % mod;
            }
        }
    }
}

IL void dft(int *x, int n) { trans(x, eps, n);}
IL void idft(int *x, int n) { 
    trans(x, ieps, n);
    int in = inv(n);
    for(int i=0;i<n;i++) x[i] = 1ll * x[i] * in % mod;
}

int jc[N], ijc[N];
int ntt_a[N], ntt_b[N], ntt_c[N];
void FFPmul(int *C, int *A, int *B, int degA, int degB) {
    int lim = 1, lglim = 0;
    while(lim <= (degA+degB)*2) {lim <<= 1; lglim++;}
    for(int i=0;i<lim;i++) rev[i] = (rev[i>>1]>>1) | ((i&1) << (lglim-1));

    for(int i=0;i<=degA;i++) ntt_a[i] = A[i];
    for(int i=degA+1;i<lim;i++) ntt_a[i] = 0;
    for(int i=0;i<=degB;i++) ntt_b[i] = B[i];
    for(int i=degB+1;i<lim;i++) ntt_b[i] = 0;
    for(int i=0;i<=degA+degB;i++) ntt_c[i] = ijc[i];
    for(int i=degA+degB+1;i<lim;i++) ntt_c[i] = 0;
    dft(ntt_a, lim); dft(ntt_b, lim); dft(ntt_c, lim);
    for(int i=0;i<lim;i++) {
        ntt_a[i] = 1ll * ntt_a[i] * ntt_c[i] % mod;
        ntt_b[i] = 1ll * ntt_b[i] * ntt_c[i] % mod;
    }
    idft(ntt_a, lim); idft(ntt_b, lim);
    for(int i=0;i<=degA+degB;i++) ntt_a[i] = 1ll * ntt_a[i] * ntt_b[i] % mod * jc[i] % mod;
    for(int i=degA+degB+1;i<lim;i++) ntt_a[i] = 0;
    for(int i=0;i<=degA+degB;i++) ntt_c[i] = (i&1) ? mod - ijc[i] : ijc[i];
    for(int i=degA+degB+1;i<lim;i++) ntt_c[i] = 0;
    dft(ntt_a, lim); dft(ntt_c, lim);
    for(int i=0;i<lim;i++) ntt_a[i] = 1ll * ntt_a[i] * ntt_c[i] % mod;
    idft(ntt_a, lim);
    for(int i=0;i<=degA+degB;i++) C[i] = ntt_a[i];
}

int n, m;
int f[N], g[N], ans[N];

int main() {
    initeps();
    jc[0] = jc[1] = ijc[0] = ijc[1] = 1;
    for(int i=2;i<N;i++) jc[i] = 1ll * jc[i-1] * i % mod;
    ijc[N-1] = inv(jc[N-1]);
    for(int i=N-2;i;i--) ijc[i] = 1ll * ijc[i+1] * (i+1) % mod;

    read(n); read(m);
    for(int i=0;i<=n;i++) read(f[i]);
    for(int i=0;i<=m;i++) read(g[i]);
    FFPmul(ans, f, g, n, m);
    for(int i=0;i<=n+m;i++) { write(ans[i]); putchar(" \n"[i==n+m]);}
    return 0;
}
\end{lstlisting}
\end{framed}

\subsubsection{BM 算法}

给出一个数列 P 从 0 开始的前 n 项,求序列 P 在 $\bmod 998244353$  下的最短线性递推式,并在 $\bmod 998244353$ 下输出 $P_m$.

输入

第一行共两个数 n, m,表示将会给出序列 P 的前 n 项,要求 $P_m$

第二行 n 个数,表示 $P_0, P_1, P_2, ..., P_{n-1}$

输出

第一行输出该最短线性递推式。

第二行输出 $P_m$​ 的值。

时间复杂度 $O(n^2 \log m)$

\begin{framed}
\begin{lstlisting}
#include<bits/stdc++.h>
#define ll long long
#define lld long ll
using namespace std;
template<typename tn> void read(tn &a){
    tn x=0,f=1; char c=' ';
    for(;!isdigit(c);c=getchar()) if(c=='-') f=-1;
    for(;isdigit(c);c=getchar()) x=x*10+c-'0';
    a=x*f;
}
const int N = 10010,mod = 998244353;
ll h[N];
int n,m;
ll fp(ll a,ll k){
    ll ans=1;
    for(;k;k>>=1,a=a*a%mod)
        if(k&1) ans=a*ans%mod;
    return ans;
}
void BM(ll *a,int n,vector<ll>&ans){
    ans.clear();
    vector<ll> lst;
    int w=0;ll delta=0;
    for(int i=1;i<=n;i++){
        ll tmp=0;
        for(int j=0;j<ans.size();j++)
            tmp=(tmp+a[i-1-j]*ans[j])%mod;
        if((a[i]-tmp)%mod==0) continue;
        if(!w){
            w=i;delta=a[i]-tmp;
            for(int j=i;j;j--) ans.push_back(0);
            continue;
        }
        vector<ll> now=ans;
        ll mul=(a[i]-tmp)*fp(delta,mod-2)%mod;
        if(ans.size()<lst.size()+i-w) ans.resize(lst.size()+i-w);
        ans[i-w-1]=(ans[i-w-1]+mul)%mod;
        for(int j=0;j<lst.size();j++) ans[i-w+j]=(ans[i-w+j]-mul*lst[j])%mod;
        if(now.size()-i<lst.size()-w){
            lst=now;w=i;delta=a[i]-tmp;
        }
    }
}
ll calc(int m,vector<ll>&coef,ll*h){
    if(m<=coef.size()) return h[m];
    int k=coef.size();
    static ll f[N],g[N],res[N],p[N];
    p[0]=-1;
    for(int i=1;i<=k;i++) p[i]=coef[i-1];
    for(int i=0;i<=2*k;i++) f[i]=g[i]=0;
    f[0]=1;
    if(k>1) g[1]=1;
    else g[0]=p[0];
    auto mul = [&](ll *a,ll *b,ll *c){
        for(int i=0;i<=2*k;i++) res[i]=0;
        for(int i=0;i<k;i++)
            for(int j=0;j<k;j++)
                res[i+j]=(res[i+j]+a[i]*b[j])%mod;
        for(int i=2*k;i>=k;i--)
            if(res[i]%mod)
                for(int j=k;~j;j--)
                    res[i-j]=(res[i-j]+res[i]*p[j])%mod;
        for(int i=0;i<2*k;i++) c[i]=res[i];
        return 0;
    };
    for(;m;m>>=1,mul(g,g,g))
        if(m&1) mul(f,g,f);
    ll ans=0;
    for(int i=0;i<k;i++)
        ans=(ans+h[i+1]*f[i])%mod;
    return ans;
}
int main(){
    read(n);read(m);
    for(int i=1;i<=n;i++)
        read(h[i]);
    vector<ll> ans;
    BM(h,n,ans);
    for(auto x:ans) cout<<(x+mod)%mod<<' ';
    cout<<'\n';
    cout<<(calc(m,ans,h)+mod)%mod<<'\n';
    return 0;
}
\end{lstlisting}
\end{framed}


\newpage
\section{数据结构}

\subsection{反悔堆}

在一条链上共 n 个坑里种树,最多种 m 棵树,不能在连续的两个坑里种树,每个坑有一个收益(可能为负),求最大收益。

\begin{framed}
\begin{lstlisting}
#include<cstdio>
#include<cstring>
#include<vector>
#include<queue>
#include<algorithm>
using namespace std;
typedef long long LL;

const int N = 500000 + 3;

struct HN {
    LL v; int a;
    HN(LL v,int a):v(v),a(a){}
    bool operator < (const HN& rhs) const {
        return v < rhs.v;
    }
};
int n,m;
int l[N],r[N],vis[N];
LL ans = 0;
LL a[N];
priority_queue<HN> q;

int main() {
    ans = 0LL;
    scanf("%d%d",&n,&m);
    for(int i=1;i<=n;i++) {
        scanf("%lld",&a[i]); q.push(HN(a[i],i));
        l[i] = i-1; r[i] = i+1; vis[i] = 0;
    }
    r[0] = 1; l[n+1] = n;
    while(m--) {
        while(vis[q.top().a]) q.pop();
        HN s = q.top(); q.pop();
        if(s.v < 0) break;
        ans += s.v;
        a[s.a] = a[l[s.a]] + a[r[s.a]] - a[s.a];
        vis[l[s.a]] = vis[r[s.a]] = true; a[l[s.a]] = a[r[s.a]] = 0;
        l[s.a] = l[l[s.a]]; r[l[s.a]] = s.a;
        r[s.a] = r[r[s.a]]; l[r[s.a]] = s.a;
        q.push(HN(a[s.a],s.a));
    }
    printf("%lld\n",ans);
    return 0;
}
\end{lstlisting}
\end{framed}

\subsection{平衡树}
\subsubsection{Treap}
1.插入 x 数

2.删除 x 数(若有多个相同的数,只删除一个)

3.查询 x 数的排名(排名定义为比当前数小的数的个数 +1)

4.查询排名为 x 的数

5.求 x 的前驱(前驱定义为小于 x,且最大的数)

6.求 x 的后继(后继定义为大于 x,且最小的数)

所有指针都要初始化为null,不初始化为null无法insert进去.

\begin{framed}
\begin{lstlisting}
#include<cstdio>
#include<cstring>
#include<cstdlib>
#include<algorithm>
using namespace std;
#define IL inline 
typedef long long LL;
const int N = 1e5 + 3;

struct Node *null;
struct Node {
    Node *ch[2];
    int v,sz,r,cnt;
    IL Node() {}
    IL Node(int v):v(v) {ch[0]=ch[1]=null; r=rand(); sz=cnt=1;}
    IL bool operator < (const Node& rhs) const { return r < rhs.r;}
    IL int cmp(int x) const {
        if(x == v) return -1;
        return v < x;
    }
    IL void upd() { sz = ch[0] -> sz + ch[1] -> sz + cnt;}
};

IL void initnull() {null = new Node(); null->v=null->sz=null->r=null->cnt=0;}
IL void rotate(Node* &o,int d) {
    Node *k = o->ch[d^1]; o->ch[d^1] = k->ch[d]; k->ch[d] = o;
    o->upd(); k->upd(); o = k;
}
void insert(Node* &o,int x) {
    if(o == null) {o = new Node(x); return;}
    o->sz++;
    int d = o->cmp(x);
    if(d == -1) {o->cnt++; return;}
    insert(o->ch[d],x);
    if(o->r < o->ch[d]->r) rotate(o,d^1);
    o -> upd();
}
void erase(Node* &o,int x) {
    if(o == null) return;
    int d = o->cmp(x);
    if(d == -1) {
        Node* u = o;
        if(o->cnt > 1) {o->cnt--; o->sz--; return;}
        if(o->ch[0] != null && o->ch[1] != null) {
            int d2 = o->ch[0]->r > o->ch[1]->r;
            rotate(o,d2); erase(o->ch[d2],x);
        }
        else {
            if(o->ch[0] == null) o = o->ch[1]; else o = o->ch[0];
            delete u;
        }
    }
    else erase(o->ch[d],x);
    if(o != null) o->upd();
}
int query_rank(Node *o, int x) {
    if(o == null) return 1;
    if(o->v == x) return o->ch[0]->sz + 1;
    else if(o->v < x) return o->ch[0]->sz + o->cnt + query_rank(o->ch[1],x);
    else return query_rank(o->ch[0],x);
}
int query_kth(Node* o, int k) {
    if(o == null) return 0;
    if(k <= o->ch[0]->sz) return query_kth(o->ch[0],k);
    else if(k > o->ch[0]->sz + o->cnt)
        return query_kth(o->ch[1],k - o->ch[0]->sz - o->cnt);
    return o->v;
}

Node *ans,*root;
void query_pre(Node* o,int x) {
    if(o == null) return;
    if(o->v < x) { ans = o; query_pre(o->ch[1],x);}
    else query_pre(o->ch[0],x);
}
void query_sub(Node* o,int x) {
    if(o == null) return;
    if(x < o->v) { ans = o; query_sub(o->ch[0],x);}
    else query_sub(o->ch[1],x);
}
IL void ins(int x) {insert(root,x);}
IL void del(int x) {erase(root,x);}
IL int rank(int x) {return query_rank(root,x);}
IL int kth(int x) {return query_kth(root,x);}
IL int pre(int x) {ans=null;query_pre(root,x);return ans->v;}
IL int sub(int x) {ans=null;query_sub(root,x);return ans->v;}

IL void init() { initnull(); ans = root = null;}

int main() {
    int n; scanf("%d",&n);
    int op,x;
    init();
    for(int i=0;i<n;i++) {
        scanf("%d%d",&op,&x);
        switch(op) {
            case 1: ins(x); break;
            case 2: del(x); break;
            case 3: printf("%d\n",rank(x)); break;
            case 4: printf("%d\n",kth(x)); break;
            case 5: printf("%d\n",pre(x)); break;
            case 6: printf("%d\n",sub(x)); break;
        }
    }
    return 0;
} 
\end{lstlisting}
\end{framed}

\subsubsection{Splay}
这个 splay 有两个哨兵节点,分别是1和n+2。

本题代码是维护可以翻转的区间序列。

void splay(Node* \&o,int x) 把左数第 x 个元素移动到根节点。

Node* merge(Node* left,Node* right) 把left树与right树合并。

void split(Node *o,int k,Node *\&left,Node *\&right) 把o这棵树前k个元素分裂到left树中,剩下的分裂到right树中。
\begin{framed}
\begin{lstlisting}
#include<cstdio>
#include<cstring>
#include<vector>
#include<algorithm>
using namespace std;
#define IL inline 
#define ri register int
typedef long long LL;

struct Node *null;
struct Node {
    Node *ch[2];
    int v,sz,cnt,flip;
    IL Node() {}
    IL Node(int v):v(v){ch[0]=ch[1]=null;flip=0;sz=cnt=1;}
    IL int cmp(int k) const {
        int d = k - ch[0]->sz;
        if(d == 1) return -1;
        return d > 0;
    }
    IL void upd() { sz = cnt + ch[0]->sz + ch[1]->sz;}
    IL void pushdown() {
        if(flip) {
            flip = 0;
            swap(ch[0],ch[1]);
            ch[0]->flip ^= 1;
            ch[1]->flip ^= 1;
        }
    }
};
IL void initnull() { null = new Node(); null->sz = null->v = null->cnt = null->flip = 0;}
IL void rotate(Node *&o,int d) {
    Node* k = o->ch[d^1]; o->ch[d^1] = k->ch[d]; k->ch[d] = o;
    o->upd(); k->upd(); o = k;
}
void splay(Node* &o,int k) {
    o->pushdown();
    int d = o->cmp(k);
    if(d == 1) k -= o->ch[0]->sz + o->cnt;
    if(d != -1) {
        Node* p = o->ch[d];
        p->pushdown(); 
        int d2 = p->cmp(k);
        int k2 = (d2 == 0 ? k : k - p->ch[0]->sz - p->cnt);
        if(d2 != -1) {
            splay(p->ch[d2],k2);
            if(d == d2) rotate(o,d^1); else rotate(o->ch[d],d);
        }
        rotate(o,d^1);
    }
}
IL Node* merge(Node* left,Node* right) {
    splay(left,left->sz);
    left->ch[1] = right;
    left->upd();
    return left;
}
IL void split(Node* o,int k,Node *&left, Node *&right) {
    splay(o,k);
    left = o;
    right = o->ch[1];
    o->ch[1] = null;
    left->upd(); 
}

const int N = 1e5 + 9;
int n,m,valcnt;
int val[N];
Node *root;

IL Node* build(int sz) {
    if(!sz) return null;
    Node *l = build(sz/2);
    Node *o = new Node(val[++valcnt]);
    o->ch[0] = l; o->ch[1] = build(sz-sz/2-1);
    o->upd();
    return o;
}

IL void init(int sz) {
    initnull(); root = null;
    for(int i=0;i<=n+2;i++) val[i] = i;
    valcnt = 0;
    root = build(sz);
}

vector<int> ans;
void print(Node *o) {
    if(o == null) return;
    o->pushdown();
    print(o->ch[0]);
    ans.push_back(o->v);
    print(o->ch[1]);
}

int main() {
    scanf("%d%d",&n,&m);
    init(n+2);
    while(m--) {
        int a,b; scanf("%d%d",&a,&b);
        Node *left,*mid,*right,*o;
        split(root,a,left,o);
        split(o,b-a+1,mid,right);
        mid->flip ^= 1;
        root = merge(merge(left,mid),right);
    }
    ans.clear();
    print(root);
    for(int i=1;i<ans.size()-1;i++) printf("%d ",ans[i]-1);
    return 0;
}
\end{lstlisting}
\end{framed}

\subsection{线段树}
\subsubsection{普通线段树}

区间乘、区间加、以及求出区间和。
\begin{framed}
\begin{lstlisting}
#include<cstdio>
#include<cstring>
#include<cctype>
#include<algorithm>
using namespace std;
typedef long long LL;
const int N = 1e5 + 3;

template<typename T>
inline void read(T& x) {
    x=0;T f=1;char ch = getchar();
    while(!isdigit(ch)){if(ch=='-')f=-1;ch=getchar();}
    while(isdigit(ch)){x=(x<<3)+(x<<1)+ch-'0';ch=getchar();}
    x *= f;
}

LL MOD;

struct Intervaltree {
    int n; LL* val;
    int ql,qr,v;
    LL sumv[N<<2],addv[N<<2],mulv[N<<2];
    void build(int o,int L,int R) {
        mulv[o]=1;
        if(L == R) {sumv[o]=val[L]%MOD;return;}//sumv[o]=addv[o]=val[L]%MOD;
        int M = L + (R-L) / 2;
        build(o<<1,L,M); build(o<<1|1,M+1,R);
        sumv[o] = (sumv[o<<1]+sumv[o<<1|1]) % MOD;
    }
    inline void build(int n,LL* val) {
        this->n = n; this->val = val;
        build(1,1,n);
    }
    inline void pushdown(int o,int L,int R) {
        int M = L + (R-L) / 2;
        sumv[o<<1]=(mulv[o]*sumv[o<<1]%MOD+(M-L+1)*addv[o]%MOD)%MOD;
        sumv[o<<1|1]=(mulv[o]*sumv[o<<1|1]%MOD+(R-M)*addv[o]%MOD)%MOD;
        mulv[o<<1]=mulv[o<<1]*mulv[o]%MOD;
        mulv[o<<1|1]=mulv[o<<1|1]*mulv[o]%MOD;
        addv[o<<1]=(addv[o<<1]*mulv[o]+addv[o])%MOD;
        addv[o<<1|1]=(addv[o<<1|1]*mulv[o]+addv[o])%MOD;
        mulv[o]=1; addv[o] = 0;
    }
    void upd_add(int o,int L,int R) {
        if(ql<=L&&R<=qr) {
            addv[o] = (addv[o]+v) % MOD;
            sumv[o] = (sumv[o]+v*(R-L+1)) % MOD;
            return ;
        }
        pushdown(o,L,R);
        int M = L + (R-L) / 2;
        if(ql <= M) upd_add(o<<1,L,M);
        if(M < qr) upd_add(o<<1|1,M+1,R);
        sumv[o] = (sumv[o<<1]+sumv[o<<1|1]) % MOD;
    }
    void upd_mul(int o,int L,int R) {
        if(ql<=L&&R<=qr) {
            mulv[o] = (mulv[o]*v) % MOD;
            addv[o] = (addv[o]*v) % MOD;
            sumv[o] = (sumv[o]*v) % MOD;
            return ;
        }
        pushdown(o,L,R);
        int M = L + (R-L) / 2;
        if(ql <= M) upd_mul(o<<1,L,M);
        if(M < qr) upd_mul(o<<1|1,M+1,R);
        sumv[o] = (sumv[o<<1]+sumv[o<<1|1]) % MOD;
    }
    LL que(int o,int L,int R) {
        if(ql <= L && R <= qr) return sumv[o];
        pushdown(o,L,R);
        LL ans = 0;
        int M = L + (R-L) / 2;
        if(ql <= M) ans = (ans + que(o<<1,L,M)) % MOD;
        if(M < qr) ans = (ans + que(o<<1|1,M+1,R)) % MOD;
        return ans;
    }
};

Intervaltree solver;

int n,m;
LL a[N];

int main() {
    read(n); read(m); read(MOD);
    for(int i=1;i<=n;i++) read(a[i]);
    for(int i=1;i<=n;i++) a[i] %= MOD;
    solver.build(n,a);
    while(m--) {
        int op,x,y,v; read(op); read(x); read(y);
        solver.ql=x,solver.qr=y;
        if(op!=3) read(v),solver.v=v;
        switch(op) {
            case 1: solver.upd_mul(1,1,n); break;
            case 2: solver.upd_add(1,1,n); break;
            case 3: printf("%lld\n",solver.que(1,1,n)); break;
        }
    }
    return 0;
}
\end{lstlisting}
\end{framed}

下面的代码中 

query(1,1,n,L,R) 求出 $\sum_{i=L}^R a_i$

query2(1,1,n,L,R) 求出 $\sum_{i=L}^R a^2_i$

\begin{framed}
\begin{lstlisting}
struct SegmentTree {
    int n, MOD; LL *val;
    LL sumv[N<<2], sumv2[N<<2], addv[N<<2], mulv[N<<2];
    IL void pushup(int o,int L,int R) {
        sumv[o] = (sumv[o<<1] + sumv[o<<1|1]) % MOD;
        sumv2[o] = (sumv2[o<<1] + sumv2[o<<1|1]) % MOD;
    }
    void build(int o,int L,int R) {
        mulv[o] = 1; 
        sumv[o] = sumv2[o] = addv[o] = 0;
        if(L == R) {
            sumv[o] = val[L] % MOD;
            sumv2[o] = val[L] * val[L] % MOD;
            return;
        }
        int M = L+R >> 1;
        build(o<<1,L,M); build(o<<1|1,M+1,R);
        pushup(o,L,R);
    }
    IL void build(int n, int MOD, LL* val) {
        this->n = n; this->MOD = MOD; this->val = val;
        build(1,1,n);
    }
    IL void pushdown(int o,int L,int R) {
        int M = L+R >> 1;
        sumv2[o<<1] = mulv[o]*mulv[o]%MOD*sumv2[o<<1]%MOD + 2*mulv[o]*addv[o]%MOD*sumv[o<<1]%MOD
            + addv[o]*addv[o]%MOD*(M-L+1)%MOD;
        sumv2[o<<1|1] = mulv[o]*mulv[o]%MOD*sumv2[o<<1|1]%MOD + 2*mulv[o]*addv[o]%MOD*sumv[o<<1|1]%MOD
            + addv[o]*addv[o]%MOD*(R-M)%MOD;
        sumv[o<<1] = (mulv[o]*sumv[o<<1]%MOD + addv[o]*(M-L+1)%MOD) % MOD;
        sumv[o<<1|1] = (mulv[o]*sumv[o<<1|1]%MOD + addv[o]*(R-M)%MOD) % MOD;
        mulv[o<<1] = mulv[o<<1] * mulv[o] % MOD;
        mulv[o<<1|1] = mulv[o<<1|1] * mulv[o] % MOD;
        addv[o<<1] = (addv[o<<1]*mulv[o]+addv[o]) % MOD;
        addv[o<<1|1] = (addv[o<<1|1]*mulv[o]+addv[o]) % MOD;
        mulv[o] = 1; addv[o] = 0;
    }
    void upd_mul(int o,int L,int R,int qL,int qR,int v) {
        if(qL <= L && R <= qR) {
            sumv2[o] = sumv2[o] * v % MOD * v % MOD;
            sumv[o] = sumv[o] * v % MOD;
            addv[o] = addv[o] * v % MOD;
            mulv[o] = mulv[o] * v % MOD; 
            return;
        }
        pushdown(o,L,R);
        int M = L+R >> 1;
        if(qL <= M) upd_mul(o<<1,L,M,qL,qR,v);
        if(M < qR) upd_mul(o<<1|1,M+1,R,qL,qR,v);
        pushup(o,L,R);
    } 
    void upd_add(int o,int L,int R,int qL,int qR,int v) {
        if(qL <= L && R <= qR) {
            sumv2[o] = (sumv2[o] + 2 * v * sumv[o] % MOD + 1LL * v * v % MOD * (R-L+1) % MOD) % MOD;
            sumv[o] = (sumv[o] + 1LL * v * (R-L+1)) % MOD;
            addv[o] = (addv[o] + v) % MOD;
            return;
        }
        pushdown(o,L,R);
        int M = L+R >> 1;
        if(qL <= M) upd_add(o<<1,L,M,qL,qR,v);
        if(M < qR) upd_add(o<<1|1,M+1,R,qL,qR,v);
        pushup(o,L,R);
    }
    LL query(int o,int L,int R,int qL,int qR) {
        if(qL <= L && R <= qR) return sumv[o];
        pushdown(o,L,R);
        int M = L+R >> 1;
        LL ans = 0;
        if(qL <= M) ans = (ans + query(o<<1,L,M,qL,qR)) % MOD;
        if(M < qR) ans = (ans + query(o<<1|1,M+1,R,qL,qR)) % MOD;
        return ans;
    }
    LL query2(int o,int L,int R,int qL,int qR) {
        if(qL <= L && R <= qR) return sumv2[o];
        pushdown(o,L,R);
        int M = L+R >> 1;
        LL ans = 0;
        if(qL <= M) ans = (ans + query2(o<<1,L,M,qL,qR)) % MOD;
        if(M < qR) ans = (ans + query2(o<<1|1,M+1,R,qL,qR)) % MOD;
        return ans;
    }
}; 
\end{lstlisting}
\end{framed}

\subsubsection{李超线段树}
本题代码强制在线。

要求在平面直角坐标系下维护两个操作:

1. 在平面上加入一条线段。记第 i 条被插入的线段的标号为 i。

2. 给定一个数 k,询问与直线 x=k 相交的线段中,交点纵坐标最大的线段的编号。
\begin{framed}
\begin{lstlisting}
#include<cstdio>
#include<cstring>
#include<algorithm>
using namespace std;
typedef long long LL;
#define IL inline
#define ri register int

const int N = 1e5 + 3;
const int M1 = 39989;
const int M2 = 1e9;

int n,lastans;

struct LINE {
    double k,b;
    LINE(double k=0.0,double b=0.0):k(k),b(b) {}
}l[N];

IL double f(int i,int x) { return l[i].k * x + l[i].b;}
IL double inter(int a,int b) { return (l[b].b-l[a].b) / (l[a].k-l[b].k);}
IL int cmp(int x,int a,int b) { return f(a,x) > f(b,x) ? a : b;}

struct LiChaoSegmentTree {
    int lazy[M1<<2];
    void build(int o,int L,int R) {
        lazy[o] = 0;
        if(L == R) return;
        int M = L + (R-L) / 2;
        build(o<<1,L,M); build(o<<1|1,M+1,R);
    }
    
    int qL,qR;
    void upd(int o,int L,int R,int v) {
        if(qL <= L && R <= qR) {
            if(!lazy[o]) {lazy[o] = v; return;}
            double ly0=f(v,L),ry0=f(v,R),ly1=f(lazy[o],L),ry1=f(lazy[o],R);
            if(ly0<=ly1 && ry0<=ry1) return;
            if(ly0>=ly1 && ry0>=ry1) {lazy[o]=v; return;}
            double x = inter(lazy[o],v);
            int M = L + (R-L) / 2;
            if(ly0 >= ly1) {
                if(x <= M) upd(o<<1,L,M,v);
                else upd(o<<1|1,M+1,R,lazy[o]), lazy[o] = v;
            }
            else {
                if(x > M) upd(o<<1|1,M+1,R,v);
                else upd(o<<1,L,M,lazy[o]), lazy[o] = v;
            }
            return;
        }
        int M = L + (R-L) / 2;
        if(qL <= M) upd(o<<1,L,M,v);
        if(M < qR) upd(o<<1|1,M+1,R,v);
    }
    
    int pos;
    int query(int o,int L,int R) {
        int ans = 0;
        if(lazy[o]) ans = lazy[o];
        if(L == R) return ans;
        int M = L + (R-L) / 2;
        if(pos <= M) ans = cmp(pos,ans,query(o<<1,L,M));
        if(M < pos) ans = cmp(pos,ans,query(o<<1|1,M+1,R));
        return ans;
    }
};

LiChaoSegmentTree lpr;

int main() {
    lastans = 0;
    scanf("%d",&n);
    int cnt = 0;
    for(int i=1;i<=n;i++) {
        int op; scanf("%d",&op);
        if(op == 0) {
            int x1; scanf("%d",&x1);
            x1 = (x1 + lastans - 1) % M1 + 1; lpr.pos = x1;
            printf("%d\n",lastans=lpr.query(1,1,M1));
        }
        else {
            int x0,y0,x1,y1; scanf("%d%d%d%d",&x0,&y0,&x1,&y1);
            x0 = (x0+lastans-1)%M1+1; x1 = (x1+lastans-1)%M1+1;
            y0 = (y0+lastans-1)%M2+1; y1 = (y1+lastans-1)%M2+1;
            if(x0 == x1) l[++cnt] = LINE(0.0,max(y0,y1));
            else l[++cnt] = LINE((double)(y1-y0)/(x1-x0),y0-(double)(y1-y0)/(x1-x0)*x0);
            lpr.qL = min(x0,x1); lpr.qR = max(x0,x1);
            lpr.upd(1,1,M1,cnt);
        }
    }
    return 0;
}
\end{lstlisting}
\end{framed}

\subsubsection{主席树}
静态区间第 k 小。

\begin{framed}
\begin{lstlisting}
#include<cstdio>
#include<cctype>
#include<cstring>
#include<algorithm>
using namespace std;
typedef long long LL;
#define IL inline
#define ri register int

void read(int &x) {
    x = 0;
    int f=1; char ch = getchar();
    while(!isdigit(ch)) {if(ch=='-')f=-1;ch=getchar();}
    while(isdigit(ch)) {x=x*10+ch-'0';ch=getchar();}
    x *= f;
}
const int N = 2e5 + 3;

int n,q,m;
int a[N],b[N];

struct PresidentSegmentTree {
    int tot;
    int rt[N];
    int numv[N<<5],ls[N<<5],rs[N<<5];
    
    int build(int L,int R) {
        int root = ++tot;
        numv[root] = 0;
        if(L == R) return root;
        int M = L + R >> 1;
        ls[root] = build(L,M);
        rs[root] = build(M+1,R);
        return root;
    }
    
    int pos;
    int upd(int pre,int L,int R) {
        int root = ++tot;
        ls[root]=ls[pre]; rs[root]=rs[pre]; numv[root]=numv[pre]+1;
        if(L == R) return root;
        int M = L+R >> 1;
        if(pos <= M) ls[root] = upd(ls[pre],L,M);
        else rs[root] = upd(rs[pre],M+1,R);
        return root;
    }
    
    int query(int u,int v,int L,int R,int k) {
        if(L == R) return L;
        int x = numv[ls[v]] - numv[ls[u]];
        int M = L+R >> 1;
        if(x >= k) return query(ls[u],ls[v],L,M,k);
        else return query(rs[u],rs[v],M+1,R,k-x);
    }
}lpr;

int main() {
    read(n); read(q);
    for(ri i=1;i<=n;i++) { read(a[i]); b[i] = a[i];}
    sort(b+1,b+1+n);
    m = unique(b+1,b+1+n) - b - 1;
    lpr.tot = 0;
    lpr.rt[0] = lpr.build(1,m);
    for(ri i=1;i<=n;i++) {
        lpr.pos = lower_bound(b+1,b+1+m,a[i]) - b;
        lpr.rt[i] = lpr.upd(lpr.rt[i-1],1,m);
    }
    while(q--) {
        int l,r,k; read(l); read(r); read(k);
        printf("%d\n",b[lpr.query(lpr.rt[l-1],lpr.rt[r],1,m,k)]);
    }
    return 0;
}
\end{lstlisting}
\end{framed}

\subsubsection{线段树分裂|合并}
给出一个可重集 a(编号为 1),它支持以下操作:

0 p x y:将可重集 p 中大于等于 x 且小于等于 y 的值放入一个新的可重集中(新可重集编号为从 2 开始的正整数,是上一次产生的新可重集的编号+1)。

1 p t:将可重集 t 中的数放入可重集 p,且清空可重集 t(数据保证在此后的操作中不会出现可重集 t)。

2 p x q:在 p 这个可重集中加入 x 个数字 q。

3 p x y:查询可重集 p 中大于等于 x 且小于等于 y 的值的个数。

4 p k:查询在 p 这个可重集中第 k 小的数,不存在时输出 -1。

\begin{framed}
\begin{lstlisting}
#include<cstdio>
#include<cstring>
#include<cctype>
#include<algorithm>
using namespace std;
typedef long long LL;
#define IL inline
#define ri register int

IL void read(int &x) {
    int f=1; x=0; char ch = getchar();
    while(!isdigit(ch)) {if(ch=='-')f=-1;ch=getchar();}
    while(isdigit(ch)) {x=x*10+ch-'0'; ch=getchar();}
    x *= f;
}

const int N = 2e5 + 3;

int n,m,totrt = 0, totrab = 0, totnode = 0;
int rt[N];
int rab[N<<5],ls[N<<5],rs[N<<5];
LL numv[N<<5];

IL int newnode() { return totrab ? rab[totrab--] : ++totnode;}
IL void del(int o) { rab[++totrab] = o; ls[o]=rs[o]=numv[o]=0; return;}

int upd(int o,int L,int R,int pos,int v) {
    if(!o) o = newnode();
    numv[o] += v;
    if(L == R) { return o;}
    int M = L+R >> 1;
    if(pos <= M) ls[o] = upd(ls[o],L,M,pos,v);
    else rs[o] = upd(rs[o],M+1,R,pos,v);
    return o;
}

LL query(int o,int L,int R,int qL,int qR) {
    if(!o) return 0;
    if(qL <= L && R <= qR) return numv[o];
    int M = L+R >> 1;
    LL ans = 0;
    if(qL <= M) ans += query(ls[o],L,M,qL,qR);
    if(M < qR) ans += query(rs[o],M+1,R,qL,qR);
    return ans;
}

int merge(int u,int v,int L,int R) {
    if(!u||!v) return u|v;
    numv[u] += numv[v];
    if(L == R) return u;
    int M = L+R >> 1;
    ls[u] = merge(ls[u],ls[v],L,M);
    rs[u] = merge(rs[u],rs[v],M+1,R);
    del(v);
    return u;
}

int kth(int o,int L,int R,int k) {
    if(L == R) return L;
    int M = L+R >> 1;
    if(numv[ls[o]] >= k) return kth(ls[o],L,M,k);
    else return kth(rs[o],M+1,R,k-numv[ls[o]]);
}

int split(int u,LL k,int L,int R) {
    int root = newnode();
    int M = L+R >> 1;
    if(k > numv[ls[u]]) rs[root] = split(rs[u],k-numv[ls[u]],L,M);
    else swap(rs[u],rs[root]);
    if(k < numv[ls[u]]) ls[root] = split(ls[u],k,M+1,R);
    numv[root] = numv[u] - k;
    numv[u] = k;
    return root;
} // <= k -> the tree of u, > k -> the tree of root.

int main() {
    totnode = totrab = 0; totrt = 1;
    read(n); read(m);
    for(ri i=1;i<=n;i++) {
        int x; read(x);
        rt[1] = upd(rt[1],1,n,i,x);
    }
    while(m--) {
        int op,x,y,z;
        read(op);
        if(op == 0) {
            read(x); read(y); read(z);
            LL k1 = query(rt[x],1,n,1,z), k2 = query(rt[x],1,n,y,z);
            int tmp;
            rt[++totrt] = split(rt[x],k1-k2,1,n);
            tmp = split(rt[totrt],k2,1,n);
            rt[x] = merge(rt[x],tmp,1,n);
        }
        else if(op == 1) {
            read(x); read(y);
            rt[x] = merge(rt[x],rt[y],1,n);
        }
        else if(op == 2) {
            read(x); read(y); read(z);
            rt[x] = upd(rt[x],1,n,z,y);
        }
        else if(op == 3) {
            read(x); read(y); read(z);
            printf("%lld\n",query(rt[x],1,n,y,z));
        }
        else if(op == 4) {
            read(x); read(y);
            if(numv[rt[x]] < y) printf("-1\n"); 
            else printf("%d\n",kth(rt[x],1,n,y));
        }
    }
    return 0;
} 
\end{lstlisting}
\end{framed}

\subsubsection{扫描线+线段树 统计矩形周长}
输入文件的第一行是一个整数 N(0<=N<5000),表示有多少个矩形。接下来N行给出了每一个矩形左下角坐标和右上角坐标(所有坐标的数值范围都在-10000到10000之间)。

输出所有矩形周长
\begin{framed}
\begin{lstlisting}
#include<cstdio>
#include<cstring>
#include<algorithm>
using namespace std;

const int maxn = 20000 + 5;

int mx=-999999999,mn=999999999;
int tot = 0, ans = 0,n,last;

struct line{
    int y,x1,x2,dir;
    line(int y=0,int x1=0,int x2=0,int dir=0):y(y),x1(x1),x2(x2),dir(dir) {}
}l[maxn];

bool cmp(line a,line b){
    return a.y < b.y || a.y == b.y && a.dir > b.dir;
}

struct Intervaltree{
    int sumv[maxn<<2],num[maxn<<2],len[maxn<<2];
    bool lflag[maxn<<2],rflag[maxn<<2];
    int y1,y2,v;
    int _sum,_num,_len;
    
    void upd(int o,int L,int R){
        if(y1 <= L && R <= y2) sumv[o] += v;
        else {
            int M = L + (R-L) / 2;
            if(y1 <= M) upd(o<<1,L,M);
            if(M < y2) upd(o<<1|1,M+1,R);
        }
        if(sumv[o]){num[o] = 1; len[o] = R-L+1; lflag[o]=rflag[o]=1;}
        else if(L == R){num[o] = 0; len[o] = 0; lflag[o]=rflag[o]=0;}
        else {
            len[o] = len[o<<1]+len[o<<1|1];
            num[o] = num[o<<1]+num[o<<1|1];
            if(rflag[o<<1]&&lflag[o<<1|1]) num[o]--;
            lflag[o] = lflag[o<<1];
            rflag[o] = rflag[o<<1|1];
        }
    }
}tree;

int main(){
    scanf("%d",&n);
    for(int i=0;i<n;i++){
        int x1,x2,y1,y2;
        scanf("%d%d%d%d",&x1,&y1,&x2,&y2);
        mx = max(mx,x2);
        mn = min(mn,x1);
        l[tot++] = line(y1,x1,x2,1); 
        l[tot++] = line(y2,x1,x2,-1);
    }
    if(mn <= 0){
        for(int i=0;i<tot;i++){
            l[i].x1 += -mn + 1;
            l[i].x2 += -mn + 1;
        }
        mx += -mn + 1;
    }
    sort(l,l+tot,cmp);
    for(int i=0;i<tot;i++){
        tree.y1 = l[i].x1; tree.y2 = l[i].x2-1; 
        tree.v = l[i].dir; tree.upd(1,1,mx);
        while(l[i].y==l[i+1].y && l[i].dir==l[i+1].dir){
            i++; tree.y1=l[i].x1; tree.y2=l[i].x2-1;
            tree.v=l[i].dir; tree.upd(1,1,mx);
        }
        ans += abs(tree.len[1]-last);
        last = tree.len[1];
        ans += tree.num[1]*2*(l[i+1].y-l[i].y);
    }
    printf("%d\n",ans);
    return 0;
}
\end{lstlisting}
\end{framed}

\subsection{动态树}
\subsubsection{Link Cut Tree}
接下来要用到的变量声明

- ch[2] 左右儿子

- pa 父亲指向

- sum 路径权值和

- v 点权

- flip 翻转标记

- Other\_Vars

函数声明

一般数据结构函数(字面意思)

1. PushUp(x)

2. PushDown(x)

Splay 系函数(不会多做解释)

1. Get(x) 获取 $x$ 是父亲的哪个儿子。

2. Splay(x) 通过和 Rotate 操作联动实现把 $x$ 旋转到当前 Splay 的根。

3. Rotate(x) 将 $x$ 向上旋转一层的操作。

新操作

1. Access(x) 把从根到 $x$ 的所有点放在一条实链里,使根到 $x$ 成为一条实路径,并且在同一棵 Splay 里。只有此操作是必须实现的,其他操作视题目而实现。

2. isroot(x) 判断 $x$ 是否是所在树的根。

3. makeroot(x) 使 $x$ 点成为其所在树的根。

4. link(x, y) 在 $x, y$ 两点间连一条边。

5. cut(x, y) 把 $x, y$ 两点间边删掉。

6. findroot(x) 找到 $x$ 所在树的根节点编号。

7. change(x, v) 修改 $x$ 的点权为 $v$ 。

8. Split(x, y) 提取出 $x, y$ 间的路径,方便做区间操作。

题目描述

给定 n 个点以及每个点的权值,要你处理接下来的 m 个操作。
操作有四种,操作从 0 到 3 编号。点从 1 到 n 编号。

- 0 x y 代表询问从 x 到 y 的路径上的点的权值的 $\text{xor}$ 和。保证 x 到 y 是联通的。

- 1 x y 代表连接 x 到 y,若 x 到 y 已经联通则无需连接。

- 2 x y 代表删除边 (x,y),不保证边 (x,y) 存在。

- 3 x y 代表将点 x 上的权值变成 y。

输入格式

第一行两个整数,分别为 n 和 m,代表点数和操作数。

接下来 n 行,每行一个整数,第 (i + 1) 行的整数 $a_i$ 表示节点 i 的权值。

接下来 m 行,每行三个整数,分别代表操作类型和操作所需的量。

输出格式

对于每一个 0 号操作,你须输出一行一个整数,表示 x 到 y 的路径上点权的 $\text{xor}$ 和。

\begin{framed}
\begin{lstlisting}
#include<bits/stdc++.h>
using namespace std;
#define IL inline
template<typename Tp> IL void read(Tp& x) {
    x=0; int f=1; char ch=getchar();
    while(!isdigit(ch)) {if(ch=='-')f=-1; ch=getchar();}
    while(isdigit(ch)) {x = x*10+ch-'0'; ch=getchar();}
    x *= f;
}
int buf[22];
template<typename Tp> IL void write(Tp x) {
    int p = 0;
    if(x < 0) {putchar('-'); x=-x;}
    if(x == 0) buf[++p] = 0;
    else while(x) {
        buf[++p] = x % 10;
        x /= 10;
    }
    for(int i=p;i;i--) putchar(buf[i] + '0');
}

struct Node *null;
struct Node {
    Node *ch[2], *pa;
    int v, sum, flip;
    IL Node() {}
    IL Node(int v):v(v),sum(v) {flip=0; pa = ch[0] = ch[1] = null;}
    IL bool Get() { return pa -> ch[1] == this;}
    IL bool isroot() { return pa == null || (pa -> ch[0] != this && pa -> ch[1] != this);}
    IL void pushup() {
        sum = v;
        if(ch[0] != null) sum ^= ch[0] -> sum;
        if(ch[1] != null) sum ^= ch[1] -> sum;
    }
    IL void pushdown() {
        if(!flip) return;
        swap(ch[0], ch[1]);
        if(ch[0] != null) ch[0]->flip ^= 1;
        if(ch[1] != null) ch[1]->flip ^= 1;
        flip = 0;
    }
};

IL void initnull() {
    null = new Node(); null->v = null->sum = null->flip = 0;
    null -> pa = null -> ch[0] = null -> ch[1] = null;
}
IL void rotate(Node *o) {
    int k = o -> Get();
    Node *x = o -> pa, *y = x -> pa, *z = o -> ch[k^1];
    if(!(x -> isroot())) y -> ch[x -> Get()] = o;
    o -> ch[k^1] = x; x -> ch[k] = z;
    x -> pa = o; o -> pa = y;
    if(z != null) z -> pa = x;
    x -> pushup(); o -> pushup();
}
IL void splay(Node *o) {
    stack<Node*> sta;
    sta.push(o);
    while(! sta.top() -> isroot()) sta.push(sta.top() -> pa);
    while(!sta.empty()) { sta.top() -> pushdown(); sta.pop();}
    while(! o -> isroot()) {
        if(! o -> pa -> isroot()) rotate(o -> Get() != o -> pa -> Get() ? o : o -> pa);
        rotate(o);
    }
}
IL Node* access(Node *o) {
    Node *x;
    for(x=null; o!=null; o = (x=o) -> pa) {
        splay(o);
        o -> ch[1] = x; 
        o -> pushup();
    }
    return x;
}
IL void makeroot(Node *o) {
    access(o); splay(o);
    o->flip ^= 1;
}
IL Node* findroot(Node *o) {
    access(o); splay(o);
    while(o -> pushdown(), o -> ch[0] != null) o = o -> ch[0];
    return o;
}
IL void link(Node* x, Node* y) {
    if(findroot(x) == findroot(y)) return;
    makeroot(x); x -> pa = y;
}
IL void cut(Node *x, Node* y) {
    makeroot(x); access(y); splay(y);
    if(y -> ch[0] == x) y -> ch[0] = x -> pa = null;
}
IL void upd(Node *o, int v) {
    splay(o); o->v = v; o->pushup();
}
IL int query(Node *x, Node* y) {
    makeroot(x); access(y); splay(y);
    return y -> sum;
}

const int N = 100000 + 5;

int n, m;
Node *root[N];

int main() {
    initnull(); 
    read(n); read(m);
    for(int i=1;i<=n;i++) {
        int v; read(v);
        root[i] = new Node(v);
    }
    while(m--) {
        int op, x, y;
        read(op); read(x); read(y);
        if(op == 0) { write(query(root[x], root[y])); putchar(10);}
        if(op == 1) link(root[x], root[y]);
        if(op == 2) cut(root[x], root[y]);
        if(op == 3) upd(root[x], y);
    }
    return 0;
} 
\end{lstlisting}
\end{framed}

\subsection{树套树}

P3380 树套树

您需要写一种数据结构(可参考题目标题),来维护一个有序数列,其中需要提供以下操作:

1. 查询 k 在区间内的排名

2. 查询区间内排名为 k 的值

3. 修改某一位值上的数值

4. 查询 k 在区间内的前驱(前驱定义为严格小于 x,且最大的数,若不存在输出 -2147483647)

5. 查询 k 在区间内的后继(后继定义为严格大于 x,且最小的数,若不存在输出 2147483647)

输入

第一行两个数 n,m 表示长度为n的有序序列和m个操作

第二行有n个数,表示有序序列

下面有m行,opt表示操作标号

若 opt=1 则为操作1,之后有三个数l,r,k 表示查询k在区间[l,r]的排名

若 opt=2 则为操作2,之后有三个数l,r,k 表示查询区间[l,r]内排名为k的数

若 opt=3 则为操作3,之后有两个数pos,k 表示将pos位置的数修改为k

若 opt=4 则为操作4,之后有三个数l,r,k 表示查询区间[l,r]内k的前驱

若 opt=5 则为操作5,之后有三个数l,r,k 表示查询区间[l,r]内k的后继


\subsubsection{离散化树状数组套动态开点权值线段树}

\begin{framed}
\begin{lstlisting}
#include <bits/stdc++.h>
using namespace std;

#define IL inline
#define pb push_back
#define fi first
#define se second
#define mk make_pair

template<typename Tp> IL void read(Tp& x) {
    x=0; int f=1; char ch=getchar();
    while(!isdigit(ch)) { if(ch == '-') f=-1; ch=getchar();}
    while(isdigit(ch)) {x = x * 10 + ch-'0'; ch=getchar();}
    x *= f;
}
int buf[22];
template<typename Tp> IL void write(Tp x) {
    int p = 0;
    if(x < 0) { putchar('-'); x=-x;}
    if(x == 0) { putchar('0'); return;}
    while(x) {
        buf[++p] = x % 10;
        x /= 10;
    }
    for(int i=p;i;i--) putchar('0' + buf[i]);
}

const int N = 50000 + 5;
const int inf = 2147483647;

struct SegNode {
    int v, ls, rs;
}t[N * 100]; // (n + m) * 50

struct ASK {
    int op, l, r, k;
    ASK(int op=0, int l=0, int r=0, int k=0):op(op), l(l), r(r), k(k) {}
}ask[N];

int n, m, totrub = 0;
int a[N], rub[N * 100];

int totLSH = 0, totnode = 0;
int LSH[N << 1];
int rt[N];

int newnode() { return totrub == 0 ? ++totnode : rub[totrub--];}
void pushup(int o, int L, int R) {
    t[o].v = t[t[o].ls].v + t[t[o].rs].v;
}
void Seg_upd(int &o, int L, int R, int pos, int v) {
    if(!o) o = newnode();
    if(L == R) { t[o].v += v; return;}
    int M = L+R >> 1;
    if(pos <= M) Seg_upd(t[o].ls, L, M, pos, v);
    else Seg_upd(t[o].rs, M+1, R, pos, v);
    pushup(o, L, R);
    if(t[o].ls != 0 && t[t[o].ls].v == 0) {
        int lson = t[o].ls; t[o].ls = 0;
        t[lson].ls = t[lson].rs = t[lson].v = 0;
        rub[++totrub] = lson;
    }
    if(t[o].rs != 0 && t[t[o].rs].v == 0) {
        int rson = t[o].rs; t[o].rs = 0;
        t[rson].ls = t[rson].rs = t[rson].v = 0;
        rub[++totrub] = rson;
    }
    return;
}

IL int lb(int x) { return x & (-x);}
IL void BIT_add(int pos, int v) {
    for(int i=pos; i<=n; i+=lb(i)) {
        Seg_upd(rt[i], 1, totLSH, a[pos], v);
    }
}

int cnttl = 0, cnttr = 0;
int tl[N], tr[N];

int query_kth(int L, int R, int k) {
    if(L == R) return L;
    int M = L+R >> 1, x = 0;
    for(int i=1;i<=cnttl;i++) x -= t[t[tl[i]].ls].v;
    for(int i=1;i<=cnttr;i++) x += t[t[tr[i]].ls].v;
    if(k <= x) {
        for(int i=1;i<=cnttl;i++) tl[i] = t[tl[i]].ls;
        for(int i=1;i<=cnttr;i++) tr[i] = t[tr[i]].ls;
        return query_kth(L, M, k);
    }
    else {
        for(int i=1;i<=cnttl;i++) tl[i] = t[tl[i]].rs;
        for(int i=1;i<=cnttr;i++) tr[i] = t[tr[i]].rs;
        return query_kth(M+1, R, k - x);
    }
}
int kth(int L, int R, int k) {
    cnttl = cnttr = 0;
    for(int i=L-1;i;i-=lb(i)) {
        tl[++cnttl] = rt[i];
    }
    for(int i=R;i;i-=lb(i)) {
        tr[++cnttr] = rt[i];
    }
    return query_kth(1, totLSH, k);
}

int query_rank(int L, int R, int val) {
    if(L == R) return 0;
    int M = L+R >> 1, x = 0;
    if(val <= M) {
        for(int i=1;i<=cnttl;i++) tl[i] = t[tl[i]].ls;
        for(int i=1;i<=cnttr;i++) tr[i] = t[tr[i]].ls;
        return query_rank(L, M, val);
    }
    else {
        for(int i=1;i<=cnttl;i++) x -= t[t[tl[i]].ls].v, tl[i] = t[tl[i]].rs;
        for(int i=1;i<=cnttr;i++) x += t[t[tr[i]].ls].v, tr[i] = t[tr[i]].rs;
        return x + query_rank(M+1, R, val);
    }
}
int rnk(int L, int R, int val) {
    cnttl = cnttr = 0;
    for(int i=L-1;i;i-=lb(i)) tl[++cnttl] = rt[i];
    for(int i=R;i;i-=lb(i)) tr[++cnttr] = rt[i];
    return query_rank(1, totLSH, val) + 1;
}

int pre(int L, int R, int val) {
    int rk = rnk(L, R, val) - 1;
    if(rk == 0) return 0;
    return kth(L, R, rk);
}
int sub(int L, int R, int val) {
    if(val == totLSH) return totLSH + 1;
    int rk = rnk(L, R, val+1);
    if(rk == R - L + 2) return totLSH + 1;
    return kth(L, R, rk);
}

int main() {
    // freopen("P3380_2.in", "r", stdin);
    // freopen("3380.out", "w", stdout);
    read(n); read(m);
    for(int i=1;i<=n;i++) read(a[i]);
    for(int i=1;i<=n;i++) LSH[i] = a[i];
    totLSH = n;
    for(int i=1;i<=m;i++) {
        int op, l, r, k; read(op);
        if(op == 3) {
            read(l); r = l;
            read(k);
        }
        else {
            read(l); read(r); read(k);
        }
        ask[i] = ASK(op, l, r, k);
        if(op == 1 || op == 2) continue;
        LSH[++totLSH] = k;
    }
    sort(LSH+1, LSH+1+totLSH);
    totLSH = unique(LSH+1, LSH+1+totLSH) - LSH - 1;
    for(int i=1; i<=n; i++) {
        a[i] = lower_bound(LSH+1, LSH+1+totLSH, a[i]) - LSH;
        BIT_add(i, 1);
    }

    // printf("After the sequence build, the amount of nodes: %d\n", totnode);

    LSH[0] = -inf; LSH[totLSH + 1] = inf;
    for(int i=1;i<=m;i++) {
        if(ask[i].op == 1) {
            ask[i].k = lower_bound(LSH+1, LSH+1+totLSH, ask[i].k) - LSH;
            write(rnk(ask[i].l, ask[i].r, ask[i].k)); putchar(10);
        }
        else if(ask[i].op == 2) {
            write(LSH[ kth(ask[i].l, ask[i].r, ask[i].k) ]); putchar(10);
        }
        else if(ask[i].op == 3) {
            BIT_add(ask[i].l, -1);
            a[ask[i].l] = lower_bound(LSH+1, LSH+1+totLSH, ask[i].k) - LSH;
            BIT_add(ask[i].l, +1);
        }
        else if(ask[i].op == 4) {
            ask[i].k = lower_bound(LSH+1, LSH+1+totLSH, ask[i].k) - LSH;
            write(LSH[ pre(ask[i].l, ask[i].r, ask[i].k) ]); putchar(10);
        }
        else if(ask[i].op == 5) {
            ask[i].k = lower_bound(LSH+1, LSH+1+totLSH, ask[i].k) - LSH;
            write(LSH[ sub(ask[i].l, ask[i].r, ask[i].k) ]); putchar(10);
        }
    }
    // printf("the total amount of nodes: %d\n", totnode);
    return 0;
}
\end{lstlisting}
\end{framed}

\subsubsection{不离散化树状数组套动态开点线段树}

\begin{framed}
\begin{lstlisting}
#include <bits/stdc++.h>
using namespace std;

#define IL inline
#define pb push_back
#define fi first
#define se second
#define mk make_pair

template<typename Tp> IL void read(Tp& x) {
    x=0; int f=1; char ch=getchar();
    while(!isdigit(ch)) { if(ch == '-') f=-1; ch=getchar();}
    while(isdigit(ch)) {x = x * 10 + ch-'0'; ch=getchar();}
    x *= f;
}
int buf[22];
template<typename Tp> IL void write(Tp x) {
    int p = 0;
    if(x < 0) { putchar('-'); x=-x;}
    if(x == 0) { putchar('0'); return;}
    while(x) {
        buf[++p] = x % 10;
        x /= 10;
    }
    for(int i=p;i;i--) putchar('0' + buf[i]);
}

const int N = 50000 + 5;
const int V = 100000000;
const int inf = 2147483647;

struct SegNode {
    int v, ls, rs;
}t[N * 200]; // (n + m) * 100

struct ASK {
    int op, l, r, k;
    ASK(int op=0, int l=0, int r=0, int k=0):op(op), l(l), r(r), k(k) {}
}ask[N];

int n, m, totrub = 0;
int a[N], rub[N*200];

int totLSH = 0, totnode = 0;
int LSH[N << 1];
int rt[N];

int newnode() { return totrub == 0 ? ++totnode : rub[totrub--];}
void pushup(int o, int L, int R) {
    t[o].v = t[t[o].ls].v + t[t[o].rs].v;
}
void Seg_upd(int &o, int L, int R, int pos, int v) {
    if(!o) o = newnode();
    if(L == R) { t[o].v += v; return;}
    int M = L+R >> 1;
    if(pos <= M) Seg_upd(t[o].ls, L, M, pos, v);
    else Seg_upd(t[o].rs, M+1, R, pos, v);
    pushup(o, L, R);
    if(t[o].ls != 0 && t[t[o].ls].v == 0) {
        int lson = t[o].ls; t[o].ls = 0;
        t[lson].ls = t[lson].rs = t[lson].v = 0;
        rub[++totrub] = lson;
    }
    if(t[o].rs != 0 && t[t[o].rs].v == 0) {
        int rson = t[o].rs; t[o].rs = 0;
        t[rson].ls = t[rson].rs = t[rson].v = 0;
        rub[++totrub] = rson;
    }
    return;
}

IL int lb(int x) { return x & (-x);}
IL void BIT_add(int x, int v) {
    for(int i=x; i<=n; i+=lb(i)) {
        Seg_upd(rt[i], 0, V, a[x], v);
    }
}

int cnttl = 0, cnttr = 0;
int tl[N], tr[N];

int query_kth(int L, int R, int k) {
    if(L == R) return L;
    int M = L+R >> 1, x = 0;
    for(int i=1;i<=cnttl;i++) x -= t[t[tl[i]].ls].v;
    for(int i=1;i<=cnttr;i++) x += t[t[tr[i]].ls].v;
    if(k <= x) {
        for(int i=1;i<=cnttl;i++) tl[i] = t[tl[i]].ls;
        for(int i=1;i<=cnttr;i++) tr[i] = t[tr[i]].ls;
        return query_kth(L, M, k);
    }
    else {
        for(int i=1;i<=cnttl;i++) tl[i] = t[tl[i]].rs;
        for(int i=1;i<=cnttr;i++) tr[i] = t[tr[i]].rs;
        return query_kth(M+1, R, k - x);
    }
}
int kth(int L, int R, int k) {
    cnttl = cnttr = 0;
    for(int i=L-1;i;i-=lb(i)) {
        tl[++cnttl] = rt[i];
    }
    for(int i=R;i;i-=lb(i)) {
        tr[++cnttr] = rt[i];
    }
    return query_kth(0, V, k);
}

int query_rank(int L, int R, int val) {
    if(L == R) return 0;
    int M = L+R >> 1, x = 0;
    if(val <= M) {
        for(int i=1;i<=cnttl;i++) tl[i] = t[tl[i]].ls;
        for(int i=1;i<=cnttr;i++) tr[i] = t[tr[i]].ls;
        return query_rank(L, M, val);
    }
    else {
        for(int i=1;i<=cnttl;i++) x -= t[t[tl[i]].ls].v, tl[i] = t[tl[i]].rs;
        for(int i=1;i<=cnttr;i++) x += t[t[tr[i]].ls].v, tr[i] = t[tr[i]].rs;
        return x + query_rank(M+1, R, val);
    }
}
int rnk(int L, int R, int val) {
    cnttl = cnttr = 0;
    for(int i=L-1;i;i-=lb(i)) tl[++cnttl] = rt[i];
    for(int i=R;i;i-=lb(i)) tr[++cnttr] = rt[i];
    return query_rank(0, V, val) + 1;
}

int pre(int L, int R, int val) {
    int rk = rnk(L, R, val) - 1;
    if(rk == 0) return -inf;
    return kth(L, R, rk);
}
int suc(int L, int R, int val) {
    int rk = rnk(L, R, val+1);
    if(rk == R - L + 2) return inf;
    return kth(L, R, rk);
}

int main() {
    read(n); read(m);
    for(int i=1;i<=n;i++) {
        read(a[i]);
        BIT_add(i, 1);
    }

    for(int i=1;i<=m;i++) {
        int op, l, r, k; read(op);
        if(op == 3) {
            read(l); r = l;
            read(k);
        }
        else {
            read(l); read(r); read(k);
        }
        ask[i] = ASK(op, l, r, k);
        if(ask[i].op == 1) {
            write(rnk(ask[i].l, ask[i].r, ask[i].k)); putchar(10);
        }
        else if(ask[i].op == 2) {
            write(kth(ask[i].l, ask[i].r, ask[i].k)); putchar(10);
        }
        else if(ask[i].op == 3) {
            BIT_add(ask[i].l, -1);
            a[ask[i].l] = ask[i].k;
            BIT_add(ask[i].l, +1);
        }
        else if(ask[i].op == 4) {
            write(pre(ask[i].l, ask[i].r, ask[i].k)); putchar(10);
        }
        else if(ask[i].op == 5) {
            write(suc(ask[i].l, ask[i].r, ask[i].k)); putchar(10);
        }
    }
    // printf("the total amount of nodes: %d\n", totnode);
    return 0;
}
\end{lstlisting}
\end{framed}

\subsubsection{线段树套平衡树}

\begin{framed}
\begin{lstlisting}
#include <bits/stdc++.h>
using namespace std;

#define IL inline
#define pb push_back
#define fi first
#define se second
#define mk make_pair
#define dbg1(x) cout << #x << " = " << x << ", "
#define dbg2(x) cout << #x << " = " << x << endl

template<typename Tp> IL void read(Tp& x) {
    x=0; int f=1; char ch=getchar();
    while(!isdigit(ch)) { if(ch == '-') f=-1; ch=getchar();}
    while(isdigit(ch)) {x = x * 10 + ch-'0'; ch=getchar();}
    x *= f;
}
int buf[22];
template<typename Tp> IL void write(Tp x) {
    int p = 0;
    if(x < 0) { putchar('-'); x=-x;}
    if(x == 0) { putchar('0'); return;}
    while(x) {
        buf[++p] = x % 10;
        x /= 10;
    }
    for(int i=p;i;i--) putchar('0' + buf[i]);
}

const int N = 50000 + 5;
const int V = 100000000;
const int inf = 2147483647;

struct Node* null;
struct Node {
    Node *ch[2];
    int v, sz, r, cnt;
    IL Node() {}
    IL Node(int v):v(v) {ch[0] = ch[1] = null; r=rand(); sz=cnt=1;}
    IL bool operator < (const Node& rhs) const { return r < rhs.r;}
    IL int cmp(int x) const {
        if(x == v) return -1;
        return v < x;
    }
    IL void upd() { sz = ch[0] -> sz + ch[1] -> sz + cnt;}
};

Node *Treapans;
IL void initnull() { null = new Node(); null->v = null->sz = null->r = null->cnt = 0; Treapans = null;}



struct Treap {
    Node *root;
    IL Treap() { root = null;}
    IL void init() { root = null;}
    IL void rotate(Node* &o, int d) {
        Node *k = o->ch[d^1]; o->ch[d^1] = k->ch[d]; k->ch[d] = o;
        o->upd(); k->upd(); o = k;
    }
    void insert(Node *&o, int x) {
        if(o == null) { o = new Node(x); return;}
        o->sz++;
        int d = o->cmp(x);
        if(d == -1) { o->cnt++; return;}
        insert(o->ch[d], x);
        if(o->r < o->ch[d]->r) rotate(o, d^1);
        o -> upd();
    }
    void erase(Node *&o, int x) {
        if(o == null) return;
        int d = o->cmp(x);
        if(d == -1) {
            Node *u = o;
            if(o->cnt > 1) { o->cnt--; o->sz--; return;}
            if(o->ch[0] != null && o->ch[1] != null) {
                int d2 = o->ch[0]->r > o->ch[1]->r;
                rotate(o, d2); erase(o->ch[d2], x);
            }
            else {
                if(o->ch[0] == null) o = o->ch[1]; else o = o->ch[0];
                delete u;
            }
        }
        else erase(o->ch[d], x);
        if(o != null) o->upd();
    }
    int query_rank(Node *o, int x) {
        if(o == null) return 1;
        if(o -> v == x) return o -> ch[0] -> sz + 1;
        else if(o -> v < x) return o -> ch[0] -> sz + o -> cnt + query_rank(o -> ch[1], x);
        else return query_rank(o -> ch[0], x);
    }
    int query_kth(Node *o, int k) {
        if(o == null) return 0;
        if(k <= o -> ch[0] ->sz) return query_kth(o -> ch[0], k);
        else if(k > o -> ch[0] -> sz + o -> cnt) 
            return query_kth(o -> ch[1], k - o->ch[0]->sz - o->cnt);
        return o -> v;
    }
    void query_pre(Node *o, int x) {
        if(o == null) return;
        if(o->v < x) { Treapans = o; query_pre(o->ch[1], x);}
        else query_pre(o->ch[0], x);
    }
    void query_suc(Node *o, int x) {
        if(o == null) return;
        if(x < o->v) { Treapans = o; query_suc(o->ch[0], x);}
        else query_suc(o->ch[1], x);
    }

    IL void insert(int x) { insert(root, x);}
    IL void erase(int x) { erase(root, x);}
    IL int rnk(int x) { return query_rank(root, x);}
    IL int kth(int x) { return query_kth(root, x);}
    IL int pre(int x) { Treapans = null; query_pre(root, x); return Treapans == null ? -inf : Treapans->v;}
    IL int suc(int x) { Treapans = null; query_suc(root, x); return Treapans == null ? inf : Treapans->v;}

    // addition
    int count_less_than_x(Node *o, int x) {
        if(o == null) return 0;
        if(o->v == x) return o->ch[0]->sz;
        else if(o->v < x) return o->ch[0]->sz + o->cnt + count_less_than_x(o->ch[1], x);
        else return count_less_than_x(o->ch[0], x);
    }
    IL int query_less_than_x(int x) { return count_less_than_x(root, x);}

    // void print(Node *o) {
    //     if(o -> ch[0] != null) print(o -> ch[0]);
    //     printf("the number %d is counted %d time(s)\n", o -> v, o -> cnt);
    //     if(o -> ch[1] != null) print(o -> ch[1]);
    // }
    // void print() { print(root);}
};


struct Treap_In_SegmentTree {
    int n;
    Treap t[N << 2];
    void build(int o, int L, int R, int *a) {
        // dbg1(o); dbg1(L); dbg2(R);
        t[o].init();
        for(int i=L;i<=R;i++) t[o].insert(a[i]);
        if(L == R) return;
        int M = L+R >> 1;
        build(o << 1, L, M, a); build(o<<1|1, M+1, R, a);
    }
    IL void init(int n, int *a) {
        this -> n = n;
        build(1, 1, n, a);
    }
    void upd(int o, int L, int R, int pos, int v, int ori_v) {
        t[o].erase(ori_v);
        t[o].insert(v);
        if(L == R) return;
        int M = L+R >> 1;
        if(pos <= M) upd(o<<1, L, M, pos, v, ori_v);
        else upd(o<<1|1, M+1, R, pos, v, ori_v);
    }
    int query_less_than_v(int o, int L, int R, int qL, int qR, int v) {
        if(qL <= L && R <= qR) { return t[o].query_less_than_x(v);}
        int M = L+R >> 1;
        int ret = 0;
        if(qL <= M) ret += query_less_than_v(o << 1, L, M, qL, qR, v);
        if(M < qR) ret += query_less_than_v(o<<1|1, M+1, R, qL, qR, v);
        return ret;
    }
    int query_kth(int qL, int qR, int k) {
        int L = 0, R = V, ans = 0;
        while(L <= R) {
            int M = L+R >> 1;
            int q = query_less_than_v(1, 1, n, qL, qR, M);
            if(q + 1 <= k) { ans = M; L = M + 1;}
            else if(q + 1 > k) { R = M - 1;}
            // dbg1(L); dbg1(R); dbg1(M); dbg1(q + 1); dbg2(ans);
        }
        return ans;
    }
    int query_pre(int o, int L, int R, int qL, int qR, int v) {
        if(qL <= L && R <= qR) {
            return t[o].pre(v);
        }
        int M = L+R >> 1;
        int ret = -inf;
        if(qL <= M) ret = max(ret, query_pre(o << 1, L, M, qL, qR, v));
        if(M < qR) ret = max(ret, query_pre(o << 1 | 1, M+1, R, qL, qR, v));
        return ret;
    }
    int query_suc(int o, int L, int R, int qL, int qR, int v) {
        if(qL <= L && R <= qR) {
            // dbg1(o); dbg1(L); dbg1(R); dbg2(t[o].suc(v));
            return t[o].suc(v);
        }
        int M = L+R >> 1;
        int ret = inf;
        if(qL <= M) ret = min(ret, query_suc(o << 1, L, M, qL, qR, v));
        if(M < qR) ret = min(ret, query_suc(o << 1 | 1, M+1, R, qL, qR, v));
        return ret;
    }

    // void print(int o, int L, int R) {
    //     dbg1(o); dbg1(L); dbg2(R);
    //     t[o].print(); putchar(10);
    //     if(L == R) return;
    //     int M = L+R >> 1;
    //     print(o << 1, L, M); print(o << 1 | 1, M + 1, R);
    // }
}lpr;

int n, m;
int a[N];

int main() {
    srand(time(0));
    initnull();
    read(n); read(m);
    for(int i=1;i<=n;i++) read(a[i]);
    lpr.init(n, a);
    // lpr.print(1, 1, n);


    while(m--) {
        int op;read(op);
        if(op == 1) {
            int l, r, k; read(l); read(r); read(k);
            write(lpr.query_less_than_v(1, 1, n, l, r, k) + 1); putchar(10);
        }
        else if(op == 2) {
            int l, r, k; read(l); read(r); read(k);
            write(lpr.query_kth(l, r, k)); putchar(10);
        }
        else if(op == 3) {
            int pos, k; read(pos); read(k);
            lpr.upd(1, 1, n, pos, k, a[pos]);
            a[pos] = k;
        }
        else if(op == 4) {
            int l, r, k; read(l); read(r); read(k);
            write(lpr.query_pre(1, 1, n, l, r, k)); putchar(10);
        }
        else if(op == 5) {
            int l, r, k; read(l); read(r); read(k);
            write(lpr.query_suc(1, 1, n, l, r, k)); putchar(10);
        }
    }
    return 0;
}
\end{lstlisting}
\end{framed}

\newpage
\section{图论}
\subsection{树上问题}
\subsubsection{树链剖分}
已知一棵包含 N 个结点的树(连通且无环),每个节点上包含一个数值,需要支持以下操作:

操作 1: 格式: 1 x y z 表示将树从 x 到 y 结点最短路径上所有节点的值都加上 z。

操作 2: 格式: 2 x y 表示求树从 x 到 y 结点最短路径上所有节点的值之和。

操作 3: 格式: 3 x z 表示将以 x 为根节点的子树内所有节点值都加上 z。

操作 4: 格式: 4 x 表示求以 x 为根节点的子树内所有节点值之和
\begin{framed}
\begin{lstlisting}
#include<cstdio>
#include<cstring>
#include<vector>
#include<algorithm>
using namespace std;
typedef long long LL;
#define IL inline
#define ri register int

const int N = 1e5 + 3;

int n,m,root;
LL MOD;
LL a[N],b[N];

struct SegmentTree {
    LL sumv[N<<2],addv[N<<2];
    void build(int o,int L,int R) {
        addv[o] = sumv[o] = 0;
        if(L == R) { sumv[o] = addv[o] = a[L]; return;}
        int M = L + (R-L) / 2;
        build(o<<1,L,M); build(o<<1|1,M+1,R);
        sumv[o] = sumv[o<<1] + sumv[o<<1|1];
    }
    
    IL void pushdown(int o,int L,int R) {
        int M = L + (R-L) / 2;
        addv[o<<1] = (addv[o<<1] + addv[o]) % MOD;
        addv[o<<1|1] = (addv[o<<1|1] + addv[o]) % MOD;
        sumv[o<<1] = (sumv[o<<1] + addv[o]*(M-L+1)) % MOD;
        sumv[o<<1|1] = (sumv[o<<1|1] + addv[o]*(R-M)) % MOD;
        addv[o] = 0;
    }
    
    int qL,qR;
    LL v;
    void upd(int o,int L,int R) {
        if(qL <= L && R <= qR) { addv[o] += v; sumv[o] = (sumv[o] + v*(R-L+1)) % MOD; return;}
        pushdown(o,L,R);
        int M = L + (R-L) / 2;
        if(qL <= M) upd(o<<1,L,M);
        if(M < qR) upd(o<<1|1,M+1,R);
        sumv[o] = sumv[o<<1] + sumv[o<<1|1];
    }
    
    LL query(int o,int L,int R) {
        if(qL <= L && R <= qR) { return sumv[o];}
        pushdown(o,L,R);
        int M = L + (R-L) / 2;
        LL ans = 0;
        if(qL <= M) ans = (ans + query(o<<1,L,M)) % MOD;
        if(M < qR) ans = (ans + query(o<<1|1,M+1,R)) % MOD;
        return ans;
    }
}lpr;

int dfs_clock=0;
int dep[N],sz[N],son[N],pa[N];
int dfn[N],top[N];
vector<int> G[N];

void dfs1(int u,int fa,int nowd) {
    dep[u] = nowd;
    sz[u] = 1;
    son[u] = 0;
    pa[u] = fa;
    for(ri i=0;i<G[u].size();i++) {
        int v = G[u][i];
        if(v == fa) continue;
        dfs1(v,u,nowd+1);
        sz[u] += sz[v];
        if(sz[v] > sz[son[u]]) son[u] = v;
    }
}
// get dep,sz,max_son,pa;

void dfs2(int u,int nowtop) {
    dfn[u] = ++dfs_clock;
    a[dfs_clock] = b[u];
    top[u] = nowtop;
    if(son[u]) dfs2(son[u],nowtop);
    for(ri i=0;i<G[u].size();i++) {
        int v = G[u][i];
        if(v == pa[u] || v == son[u]) continue;
        dfs2(v,v);
    }
}
//get dfn,top

IL void _1() {
    int x,y; scanf("%d%d%lld",&x,&y,&lpr.v);
    lpr.v %= MOD;
    while(top[x] != top[y]) {
        if(dep[top[x]] < dep[top[y]]) swap(x,y);
        lpr.qL = dfn[top[x]], lpr.qR = dfn[x];
        lpr.upd(1,1,n);
        x = pa[top[x]];
    }
    if(dep[x] > dep[y]) swap(x,y);
    lpr.qL = dfn[x], lpr.qR = dfn[y];
    lpr.upd(1,1,n);
}

IL LL _2() {
    LL ans = 0;
    int x,y; scanf("%d%d",&x,&y);
    while(top[x] != top[y]) {
        if(dep[top[x]] < dep[top[y]]) swap(x,y);
        lpr.qL = dfn[top[x]], lpr.qR = dfn[x];
        ans = (ans + lpr.query(1,1,n)) % MOD;
        x = pa[top[x]];
    }
    if(dep[x] > dep[y]) swap(x,y);
    lpr.qL = dfn[x], lpr.qR = dfn[y];
    ans = (ans + lpr.query(1,1,n)) % MOD;
    return ans;
}

IL void _3() {
    int x; scanf("%d%lld",&x,&lpr.v);
    lpr.qL = dfn[x]; lpr.qR = dfn[x] + sz[x] - 1;
    lpr.v %= MOD;
    lpr.upd(1,1,n);
}

IL LL _4() {
    int x; scanf("%d",&x);
    lpr.qL = dfn[x]; lpr.qR = dfn[x] + sz[x] - 1;
    return lpr.query(1,1,n);
}

int main() {
    dfs_clock = 0;
    scanf("%d%d%d%lld",&n,&m,&root,&MOD);
    for(ri i=1;i<=n;i++) scanf("%lld",&b[i]);
    for(ri i=1;i<n;i++) {
        int u,v; scanf("%d%d",&u,&v);
        G[u].push_back(v);
        G[v].push_back(u);
    }
    dfs1(root,0,1);
    dfs2(root,root);
    lpr.build(1,1,n);
    while(m--) {
        int op; scanf("%d",&op);
        if(op == 1) _1();
        if(op == 2) printf("%lld\n",_2());
        if(op == 3) _3();
        if(op == 4) printf("%lld\n",_4());
    }
    return 0;
}
\end{lstlisting}
\end{framed}

\subsubsection{虚树}
建虚树需要先树链剖分再建虚树。

虚树中只有关键点和 dfs 序相邻的关键点的 LCA 。

虚树的边权需要仔细考虑清楚。

\begin{framed}
\begin{lstlisting}
#include<cstdio>
#include<cstring>
#include<algorithm>
#include<vector>
using namespace std;
typedef long long LL;
typedef pair<int,int> pii;
#define se second
#define fi first
#define pb push_back
#define IL inline
#define ri register int

struct Edge {
    int u,v,w;
    Edge(int u=0,int v=0,int w=0):u(u),v(v),w(w) {}
};

const int N = 250000 + 3;
const LL INFL = 1e18 + 7;

int n,m;

int dfs_clock = 0;
int dep[N],sz[N],pa[N],son[N];
int dfn[N],top[N];
LL minv[N];

vector<Edge> G[N];

void dfs1(int u,int f,int nowd,LL noww) {
    dep[u] = nowd;
    sz[u] = 1;
    pa[u] = f;
    son[u] = 0;
    minv[u] = noww;
    for(auto e : G[u]) {
        int v = e.v, w = e.w;
        if(v == f) continue;
        dfs1(v,u,nowd+1,min(noww,(LL)w));
        sz[u] += sz[v];
        if(sz[v] > sz[son[u]]) son[u] = v;
    }
}
void dfs2(int u,int nowtop) {
    dfn[u] = ++dfs_clock;
    top[u] = nowtop;
    if(son[u]) dfs2(son[u],nowtop);
    for(auto e : G[u]) {
        int v = e.v;
        if(v == pa[u] || v == son[u]) continue;
        dfs2(v,v);
    }
}

int LCA(int u,int v) {
    int t1 = top[u], t2 = top[v];
    while(top[u] != top[v]) {
        if(dep[top[u]] < dep[top[v]]) {
            swap(u,v);
        }
        u = pa[top[u]];
    }
    return dep[u] > dep[v] ? v : u;
}

int h[N];
int sta[N];

IL bool cmp(int a,int b) { return dfn[a] < dfn[b];} 
IL void build_virtual_tree() {
    int statop;
    sort(h+1,h+1+m,cmp);
    sta[statop = 1] = 1; G[1].clear();
    for(ri i=1,l;i<=m;i++) {
        if(h[i] != 1) {
            l = LCA(h[i],sta[statop]);
            if(l != sta[statop]) {
                while(dfn[l] < dfn[sta[statop-1]]) {
                    G[sta[statop-1]].pb(Edge(sta[statop-1],sta[statop]));
                    statop--;
                }
                if(dfn[l] > dfn[sta[statop-1]]) {
                    G[l].clear();
                    G[l].pb(Edge(l,sta[statop]));
                    sta[statop] = l;
                }
                else G[l].pb(Edge(l,sta[statop--]));
            }
            G[h[i]].clear(); sta[++statop] = h[i];
        }
    }
    for(ri i=1;i<statop;i++) G[sta[i]].pb(Edge(sta[i],sta[i+1]));
    return;
}

bool vip[N];
LL dp(int u) {
    LL ans = 0, sum = 0;
    for(auto e : G[u]) {
        int v = e.v;
        sum += dp(v);
    }
    if(vip[u]) ans = minv[u];
    else ans = min(minv[u],sum);
    return ans;
}

int main() {
    int n; scanf("%d",&n);
    for(ri i=1;i<n;i++) {
        int u,v,w; scanf("%d%d%d",&u,&v,&w);
        G[u].pb(Edge(u,v,w));
        G[v].pb(Edge(v,u,w));
    } 
    dfs1(1,0,1,INFL); dfs2(1,1);
    
    int T; scanf("%d",&T);
    while(T--) {
        scanf("%d",&m);
        for(ri i=1;i<=m;i++) scanf("%d",h+i);
        for(ri i=1;i<=m;i++) vip[h[i]] = true;
        build_virtual_tree();
        printf("%lld\n",dp(1));
        for(ri i=1;i<=m;i++) vip[h[i]] = false;
    }
    return 0;
}
\end{lstlisting}
\end{framed}

\subsubsection{求出每个子树重心}
从这棵树的叶子节点开始往上推出各个子树的重心。关于重心,我们有一条性质:以树的重心为根时,所有子树的大小都不超过整棵树大小的一半。假设我们当前要开始找以 u 节点为根的子树重心,已经把 u 的儿子们的重心都找出来了,那么以 u 节点为根的子树重心一定在重儿子子树重心到 u 的路径上。重心可能有两个。
\begin{framed}
\begin{lstlisting}
#include<cstdio>
#include<cstring>
#include<vector>
#include<algorithm>
using namespace std;

const int N = 2e5 + 3;
vector<int> G[N];

int pa[N],ans[N],sz[N];

void dfs(int u,int fa) {
    pa[u] = fa; ans[u] = u; sz[u] = 1;
    for(int i=0;i<G[u].size();i++) {
        int v = G[u][i];
        if(v == fa) continue;
        dfs(v,u);
        sz[u] += sz[v];
    }
    for(int i=0;i<G[u].size();i++) {
        int v = G[u][i], ansv = ans[v];
        if(v == fa) continue;
        while(sz[ansv]*2 < sz[u]) ansv = pa[ansv];
        if(sz[ansv] < sz[ans[u]]) ans[u] = ansv;
    }
}

int main() {
    int n; scanf("%d",&n);
    for(int i=0,u,v;i<n-1;i++) {
        scanf("%d%d",&u,&v); u--; v--;
        G[u].push_back(v); G[v].push_back(u);
    }
    dfs(0,-1);
    for(int i=0;i<n;i++) {
        if(sz[ans[i]]*2 == sz[i]) {
            int a=ans[i], b=pa[ans[i]];
            if(a > b) swap(a,b);
            printf("%d %d\n",a+1,b+1);
        }
        else printf("%d\n",ans[i]+1);
    }
    return 0;
}
\end{lstlisting}
\end{framed}

\subsubsection{点分治}
时间复杂度 $O(n \log^2 n)$ ,点分治本身一个 log ,套了树状数组一个 log。

下面的代码是求出给定一棵 n 个节点的树,每条边有边权,求出树上两点距离小于等于 k 的点对数量。
\begin{framed}
\begin{lstlisting}
#include<cstdio>
#include<cstring>
#include<algorithm>
#include<vector>
#include<queue>
using namespace std;
typedef long long LL;
#define IL inline
#define ri register int
#define mk make_pair
#define pb push_back
#define fi first
#define se second

const int N = 4e4 + 5;

int n,K;

int BIT[N];
IL int lb(int x) {return x & (-x);}
IL int sum(int x) {
    int ans = 0;
    for(;x;x-=lb(x)) ans += BIT[x];
    return ans;
}
IL void add(int x,int d) {
    for(;x<=K;x+=lb(x)) BIT[x] += d;
}

struct Edge {
    int u,v,w;
    Edge(int u=0,int v=0,int w=0):u(u),v(v),w(w){}
};

vector<Edge> G[N];

int sz[N];
bool vis[N];
void dfs_cen(int u,int f,int tot,int &cen) {
    sz[u] = 1;
    int now = 0;
    for(auto e : G[u]) {
        int v = e.v;
        if(v == f || vis[v]) continue;
        dfs_cen(v,u,tot,cen);
        sz[u] += sz[v];
        now = max(now,sz[v]);
    }
    now = max(now,tot - sz[u]);
    if(now * 2 <= tot) cen = u;
}
IL int get_cen(int u,int tot) { int cen; dfs_cen(u,0,tot,cen); return cen;}

void dfs_sz(int u,int f) {
    sz[u] = 1;
    for(auto e : G[u]) {
        int v = e.v, w = e.w;
        if(v == f || vis[v]) continue;
        dfs_sz(v,u);
        sz[u] += sz[v];
    }
}

int d[N];
void dfs_dis(int u,int f,int dis,int &cnt) {
    d[++cnt] = dis;
    for(auto e : G[u]) {
        int v = e.v, w = e.w;
        if(v == f || vis[v]) continue;
        dfs_dis(v,u,dis+w,cnt);
    }
}

queue<int> tag;
int work(int u,int tot) {
    u = get_cen(u,tot);
    dfs_sz(u,0);
    vis[u] = true;
    //printf("work : %d %d\n",u,tot);
    
    //solve
    int ans = 0;
    for(auto e : G[u]) {
        int v = e.v, w = e.w;
        if(vis[v]) continue;

        int cnt = 0;
        dfs_dis(v,u,w,cnt);
        for(ri i=1;i<=cnt;i++) {
            if(d[i] > K) continue;
            ans += sum(K-d[i]) + 1;
        }
        for(ri i=1;i<=cnt;i++) {
            add(d[i],+1);
            tag.push(d[i]);
        }
    }
    
    while(!tag.empty()) {
        int x = tag.front(); tag.pop();
        add(x,-1);
    }
    for(auto e : G[u]) {
        int v = e.v;
        if(vis[v]) continue;
        ans += work(v,sz[v]);
    }
    return ans;
}

int main() {
    scanf("%d",&n);
    for(ri i=1;i<n;i++) {
        int u,v,w; scanf("%d%d%d",&u,&v,&w);
        G[u].pb(Edge(u,v,w));
        G[v].pb(Edge(v,u,w));
    }
    scanf("%d",&K);
    printf("%d\n",work(1,n));
    return 0;
}
\end{lstlisting}
\end{framed}

\subsubsection{dsu on tree}
\paragraph{问题一}

给定一棵 n 个点的树 T。点有点权,第 i 个点的点权为 $val_i$。求出每个点的子树中的权值种数。 $1 \leq n \leq 10^5, 0 \leq val_i \leq 10^9$。

由于颜色具体时多少不重要,所以可以离散化。用数组 col[] 表示每种颜色的出现次数,每次向上合并时,父节点直接把重儿 子的信息拿过来用,再将轻儿子的信息暴力合并上去,边合并边 维护答案即可。 根据实现的不同,时间复杂度为 $O(n \log nf(n))$。
\begin{framed}
\begin{lstlisting}
//树上启发式合并 
set<int>* S[MAXN];
void dfs3(int cur) {
  for (auto nx: G[cur]) {
    if (nx == fa[cur]) continue;
    dfs3(nx);
  }
  // S[son1], S[son2], ...
  if (!son[cur]) {
    S[cur] = new set<int>();
    S[cur]->insert(v[cur]);
  } else {
    S[cur] = S[son[cur]];
    for (auto nx: G[cur]) {
      if (nx == fa[cur] || nx == son[cur]) continue;
      for (auto val: *S[nx]) {
        S[cur]->insert(val);
      }
    }
  }
  // S[cur]
  ans[cur] = S[cur]->size();
}
\end{lstlisting}
\end{framed}

\paragraph{问题二}

给定一棵 n 个点的树 T。点有点权,第 i 个点的点权为 $val_i$。求出每个点的子树中的众数。

\begin{framed}
\begin{lstlisting}
struct DS {
  map<int, int> M;
  int ans;

  DS(): ans(-1) {}

  void insert(int x, int v) {
    M[x] += v;
    if (ans == -1 || M[x] > M[ans]) {
      ans = x;
    }
  }
};

DS* D[MAXN];

void dfs4(int cur) {
  for (auto nx: G[cur]) {
    if (nx == fa[cur]) continue;
    dfs4(nx);
  }
  if (!son[cur]) {
    D[cur] = new DS();
    D[cur]->insert(v[cur], 1);
  } else {
    D[cur] = D[son[cur]];
    for (auto nx: G[cur]) {
      if (nx == fa[cur] || nx == son[cur]) continue;
      for (auto t: D[nx]->M) {
        D[cur]->insert(t.first, t.second);
      }
    }
  }
  // cur 的子树里面的众数为 D[cur]->ans
}
\end{lstlisting}
\end{framed}

\subsubsection{最小斯坦纳树}

给定一个包含 n 个结点和 m 条带权边的无向连通图 G=(V,E)。

再给定包含 k 个结点的点集 S,选出 G 的子图 G'=(V',E'),使得:

1. S $\subseteq$ V′;

2. G′ 为连通图;

3. E' 中所有边的权值和最小。

求出 E' 中所有边的权值和。

对于 100\% 的数据,$1\leq n\leq 100,\ \ 1\leq m\leq 500,\ \ 1\leq k\leq 10,\ \ 1\leq u,v\leq n,\ \ 1\leq w\leq 10^6。$

保证给出的无向图连通,但可能存在重边和自环。

\begin{framed}
\begin{lstlisting}
#include<bits/stdc++.h>
using namespace std;
typedef pair<int, int> pii;
#define mk make_pair
#define pb push_back
#define fi first
#define se second

const int N = 500 + 5;
const int INF = 0x3f3f3f3f;

struct Edge {
    int u, v, w;
    Edge(int u=0,int v=0, int w=0):u(u),v(v),w(w) {}
};

int n, m, k;
int f[N][1030];
int vip[N];
int done[N];
vector<Edge> G[N];

priority_queue<pii, vector<pii>, greater<pii> > q;
void dijkstra(int s) {
    fill(done+1, done+1+n, 0);
    while(!q.empty()) {
        pii u = q.top(); q.pop();
        if(done[u.se]) continue;
        done[u.se] = true;
        for(auto &e : G[u.se]) {
            int v = e.v, w = e.w;
            if(f[v][s] > f[u.se][s] + w) {
                f[v][s] = f[u.se][s] + w;
                q.push(mk(f[v][s], v));
            }
        }
    }
}

int main() {
    memset(f, 0x3f, sizeof(f));
    scanf("%d%d%d",&n,&m,&k);
    for(int i=1;i<=m;i++) {
        int u, v, w; scanf("%d%d%d",&u,&v,&w);
        G[u].pb(Edge(u,v,w));
        G[v].pb(Edge(v,u,w));
    }
    for(int i=1;i<=k;i++) {
        scanf("%d",vip+i);
        f[vip[i]][1 << (i-1)] = 0;
    }
    for(int s=1; s < (1<<k); s++) {
        for(int i=1;i<=n;i++) {
            for(int subs = s&(s-1); subs; subs = s&(subs-1)) 
                f[i][s] = min(f[i][s], f[i][subs] + f[i][s^subs]);
            if(f[i][s] != INF) q.push(mk(f[i][s], i));
        }
        dijkstra(s);
    }
    printf("%d\n", f[vip[1]][(1<<k) - 1]);
    return 0;
}
\end{lstlisting}
\end{framed}

\subsection{图的匹配问题}
\subsubsection{二分图最大匹配}
\paragraph{使用邻接矩阵实现的匈牙利算法}

时间复杂度最坏 $O(|V|^3)$

\begin{framed}
\begin{lstlisting}
#include<cstdio>
#include<cstring>
#include<cmath>
#include<algorithm>
using namespace std;

const int N = 1000 + 3;

struct Hungary {
    int n,m;
    int Left[N];
    bool S[N],T[N],G[N][N];
    
    void init(int n,int m) {
        this->n = n; this->m = m;
        fill(Left,Left+m+1,0);
        fill(S,S+n+1,0);
        fill(T,T+m+1,0);
        for(int i=1;i<=n;i++) fill(G[i],G[i]+m+1,0);
    }
    
    void AddEdge(int u,int v) { G[u][v] = 1;}
    
    bool match(int i) {
        S[i] = true;
        for(int j=1;j<=m;j++) if(G[i][j] && !T[j]) {
            T[j] = true;
            if(Left[j] == 0 || match(Left[j])) {
                Left[j] = i;
                return true;
            }
        }
        return false;
    }
    
    int solve() {
        int ans = 0;
        fill(Left,Left+m+1,0);
        for(int i=1;i<=n;i++) {
            fill(T,T+m+1,0);
            if(match(i)) ans++;
        }
        return ans;
    }
};

Hungary solver;

int main() {
    int n,m,e; scanf("%d%d%d",&n,&m,&e);
    solver.init(n,m);
    while(e--) {
        int u,v; scanf("%d%d",&u,&v);
        if(u > n || u<1 || v > m || v < 1) continue;
        solver.AddEdge(u,v);
    }
    printf("%d\n",solver.solve());
    return 0;
}
\end{lstlisting}
\end{framed}

\paragraph{使用邻接表实现的匈牙利算法}
时间复杂度 $O(VE)$

\begin{framed}
\begin{lstlisting}
#include<cstdio>
#include<cstring>
#include<cmath>
#include<algorithm>
#include<vector>
using namespace std;
typedef vector<int> vi;
#define IL inline
#define ri register int
#define pb push_back
#define mk make_pair
#define fi first
#define se second

const int N = 1000 + 3;

struct Hungary {
    int n,m;
    int Left[N];
    bool S[N],T[N];
    vector<int> G[N];
    
    void init(int n,int m) {
        this->n = n; this->m = m;
        fill(Left,Left+m+1,0);
        fill(S,S+n+1,0);
        fill(T,T+m+1,0);
        for(int i=1;i<=n;i++) G[i].clear();
    }
    
    void AddEdge(int u,int v) { G[u].pb(v);}
    
    bool match(int u) {
        S[u] = true;
        for(auto v : G[u]) if(!T[v]) {
            T[v] = true;
            if(Left[v] == 0 || match(Left[v])) {
                Left[v] = u;
                return true;
            }
        }
        return false;
    }
    
    int solve() {
        int ans = 0;
        fill(Left,Left+m+1,0);
        for(int i=1;i<=n;i++) {
            fill(T,T+m+1,0);
            if(match(i)) ans++;
        }
        return ans;
    }
};

Hungary solver;

int main() {
    int n,m,e; scanf("%d%d%d",&n,&m,&e);
    solver.init(n,m);
    while(e--) {
        int u,v; scanf("%d%d",&u,&v);
        if(u > n || u<1 || v > m || v < 1) continue;
        solver.AddEdge(u,v);
    }
    printf("%d\n",solver.solve());
    return 0;
}
\end{lstlisting}
\end{framed}

\paragraph{使用 Dinic 算法的建模方法}
将左边所有点接上源点,右边所有点接上汇点,容量皆为 1 。原来的每条边从左往右连边,容量也皆为 1 ,最大流即最大匹配。如果使用 Dinic 算法求该网络的最大流,可在 $O(\sqrt n m)$ 求出。

\paragraph{二分图最大独立集}
选最多的点,满足两两之间没有边相连。二分图中,最大独立集 = n - 最大匹配。

\paragraph{二分图最小点覆盖}
选最少的点,满足每条边至少有一个端点被选,不难发现补集是独立集。二分图中,最小点覆盖 = n - 最大独立集 = 最大匹配。


\subsubsection{二分图最大权匹配}
匈牙利算法又称为 KM 算法,可以在 $O(n^3)$ 时间内求出二分图的 最大权完美匹配。

考虑到二分图中两个集合中的点并不总是相同,为了能应用 KM 算法解决二分图的最大权匹配,需要先作如下处理:将两个集合中点数比较少的补点,使得两边点数相同,再将不存在的边权重设为 $0$,这种情况下,问题就转换成求 最大权完美匹配问题,从而能应用 KM 算法求解。

uoj \#80 二分图最大权匹配代码:

\begin{framed}
\begin{lstlisting}
#include <bits/stdc++.h>
using namespace std;

template <typename T>
struct hungarian {  // km
  int n;
  vector<int> matchx;
  vector<int> matchy;
  vector<int> pre;
  vector<bool> visx;
  vector<bool> visy;
  vector<T> lx;
  vector<T> ly;
  vector<vector<T> > g;
  vector<T> slack;
  T inf;
  T res;
  queue<int> q;
  int org_n;
  int org_m;

  hungarian(int _n, int _m) {
    org_n = _n;
    org_m = _m;
    n = max(_n, _m);
    inf = numeric_limits<T>::max();
    res = 0;
    g = vector<vector<T> >(n, vector<T>(n));
    matchx = vector<int>(n, -1);
    matchy = vector<int>(n, -1);
    pre = vector<int>(n);
    visx = vector<bool>(n);
    visy = vector<bool>(n);
    lx = vector<T>(n, -inf);
    ly = vector<T>(n);
    slack = vector<T>(n);
  }

  void addEdge(int u, int v, T w) {
    g[u][v] = max(w, 0);  // 负值还不如不匹配 因此设为0不影响
  }

  bool check(int v) {
    visy[v] = true;
    if (matchy[v] != -1) {
      q.push(matchy[v]);
      visx[matchy[v]] = true;
      return false;
    }
    while (v != -1) {
      matchy[v] = pre[v];
      swap(v, matchx[pre[v]]);
    }
    return true;
  }

  void bfs(int i) {
    while (!q.empty()) {
      q.pop();
    }
    q.push(i);
    visx[i] = true;
    while (true) {
      while (!q.empty()) {
        int u = q.front();
        q.pop();
        for (int v = 0; v < n; v++) {
          if (!visy[v]) {
            T delta = lx[u] + ly[v] - g[u][v];
            if (slack[v] >= delta) {
              pre[v] = u;
              if (delta) {
                slack[v] = delta;
              } else if (check(v)) {
                return;
              }
            }
          }
        }
      }
      // 没有增广路 修改顶标
      T a = inf;
      for (int j = 0; j < n; j++) {
        if (!visy[j]) {
          a = min(a, slack[j]);
        }
      }
      for (int j = 0; j < n; j++) {
        if (visx[j]) {  // S
          lx[j] -= a;
        }
        if (visy[j]) {  // T
          ly[j] += a;
        } else {  // T'
          slack[j] -= a;
        }
      }
      for (int j = 0; j < n; j++) {
        if (!visy[j] && slack[j] == 0 && check(j)) {
          return;
        }
      }
    }
  }

  void solve() {
    // 初始顶标
    for (int i = 0; i < n; i++) {
      for (int j = 0; j < n; j++) {
        lx[i] = max(lx[i], g[i][j]);
      }
    }

    for (int i = 0; i < n; i++) {
      fill(slack.begin(), slack.end(), inf);
      fill(visx.begin(), visx.end(), false);
      fill(visy.begin(), visy.end(), false);
      bfs(i);
    }

    // custom
    for (int i = 0; i < n; i++) {
      if (g[i][matchx[i]] > 0) {
        res += g[i][matchx[i]];
      } else {
        matchx[i] = -1;
      }
    }
    cout << res << "\n";
    for (int i = 0; i < org_n; i++) {
      cout << matchx[i] + 1 << " ";
    }
    cout << "\n";
  }
};

int main() {
  ios::sync_with_stdio(0), cin.tie(0);
  int n, m, e;
  cin >> n >> m >> e;

  hungarian<long long> solver(n, m);

  int u, v, w;
  for (int i = 0; i < e; i++) {
    cin >> u >> v >> w;
    u--, v--;
    solver.addEdge(u, v, w);
  }
  solver.solve();
}

\end{lstlisting}
\end{framed}

\subsubsection{一般图最大匹配}
带花树算法 $2 \leq|V| \leq 500, 1 \leq |E| \leq 124750$ 无法分析复杂度。

输出一般图最大匹配的边数。

输出一般图匹配中每个点匹配后对应的另一个点。

\begin{framed}
\begin{lstlisting}
#include <cstdio>
#include <cstring>
#include<cstdlib>
#include <algorithm>
#define M 250010

using namespace std;

char inp[33554432], *inpch = inp;

int Head[M], Next[M], Go[M], Pre[510], Nxt[510], F[510], S[510], Q[510], Vis[510], *Top = Q, Cnt = 0, Tim = 0, n, m, x, y;

inline void addedge(int x, int y)
{
    Go[++Cnt] = y;
    Next[Cnt] = Head[x];
    Head[x] = Cnt;
}

int find(int x)
{
    return x == F[x] ? x : F[x] = find(F[x]);
}

int lca(int x, int y)
{
    for(Tim++, x = find(x), y = find(y); ; x ^= y ^= x ^= y)
        if(x)
        {
            if(Vis[x] == Tim) return x;
            Vis[x] = Tim;
            x = find(Pre[Nxt[x]]);
        }
}

void blossom(int x, int y, int l)
{
    while(find(x) != l)
    {
        Pre[x] = y, y = Nxt[x];
if (S[y]==-1)exit(233);
        S[*Top = y] = 0, *Top++;
        if(F[x] == x) F[x] = l;
        if(F[y] == y) F[y] = l;
        x = Pre[y];
    }
}

int Match(int x)
{
    for(int i = 1; i <= n; i++) F[i] = i;
    memset(S, -1, sizeof S);
    S[*(Top = Q) = x] = 0, Top++;
    for(int *i = Q; i != Top; *i++)
        for(int T = Head[*i]; T; T = Next[T])
        {
            int g = Go[T];
            if(S[g] == -1)
            {
                Pre[g] = *i, S[g] = 1;
                if(!Nxt[g])
                {
                    for(int u = g, v = *i, lst; v; u = lst, v = Pre[u])
                        lst = Nxt[v], Nxt[v] = u, Nxt[u] = v;
                    return 1;
                }
                S[*Top = Nxt[g]] = 0, Top++;
            }
            else if(!S[g] && find(g) != find(*i))
            {
                int l = lca(g, *i);
                blossom(g, *i, l);
                blossom(*i, g, l);
            }
        }
    return 0;
}

inline void Read(int& x)
{
    x = 0;
    while(*inpch < '0') *inpch++;
    while(*inpch >= '0') x = x * 10 + *inpch++ - '0';
}

int main()
{
    fread(inp, 1, 33554432, stdin);
    Read(n), Read(m);
    for(int i = 1; i <= m; i++)
    {
        Read(x), Read(y);
        addedge(x, y);
        addedge(y, x);
    }
    int ans = 0;
    for(int i = n; i >= 1; i--)
        if(!Nxt[i]) ans += Match(i);
    printf("%d\n", ans);
    for(int i = 1; i <= n; i++) printf("%d ", Nxt[i]);
    putchar('\n');
    return 0;
}
\end{lstlisting}
\end{framed}

\subsubsection{一般图最大权匹配}
带花树算法 $2 \leq|V| \leq 400, 1 \leq |E| \leq 79800$ 无法分析复杂度。

第一行输出总共的权值和。

接下来输出一般图最大权匹配中每个点匹配后对应的另一个点。

\begin{framed}
\begin{lstlisting}
#include <bits/stdc++.h>
#define pb emplace_back
using namespace std;
bool chkmax(int &a,int b){ return a<b?a=b,true:false; }
bool chkmin(int &a,int b){ return b<a?a=b,true:false; }
const int N=605,inf=1<<30,NUL=1,IN=2,OUT=3; // 1.5n,2w
struct solver {
    struct edge_t {
        int u,v,w;
        edge_t(){ u=v=w=0; }
        edge_t(int u,int v,int w):u(u),v(v),w(w){}
    } e[N][N];
    int nc,n,m,h[N],lk[N],fl[N],fc,match[N],in[N],mpos[N],col[N],from[N];
    vector<int> flower[N];
    void init(int n0){
        nc=n0,n=m=n0+(n0&1);
        for(int i=1;i<=n;++i) for(int j=1;j<=n;++j) e[i][j]=edge_t(i,j,0);
    }
    void add(int u,int v,int c){ if(chkmax(e[u][v].w,c)) e[v][u]=e[u][v]; }
    int len(edge_t e){ return h[e.u]+h[e.v]-e.w*2; }
    int other(edge_t e,int x){ return in[e.u]+in[e.v]-x; }
    int slack(int x){ return mpos[x]?len(e[x][in[mpos[x]]])/(1+(col[x]==OUT)):inf; }
    void upd(int v,int u){ if(!mpos[u]||len(e[u][mpos[u]])>len(e[u][v])) mpos[u]=v; }
    void set(int x,int b){ in[x]=b; if(x>n) for(int y:flower[x]) set(y,b); }
    int lca(int u,int v){
        for(u=in[u],v=in[v],++fc;u||v;u=in[from[u]],v=in[from[v]]){
            if(fl[u]==fc) return u; if(u) fl[u]=fc;
            if(fl[v]==fc) return v; if(v) fl[v]=fc;
        }
        return 0;
    }
    void color(int x,int c){
        col[x]=c,mpos[x]=0;
        if(c!=IN){
            for(int y=1;y<=m;++y) if(y!=x&&in[y]==y&&col[y]!=IN){
                if(col[y]==OUT) upd(y,x);
                if(c==OUT) upd(x,y);
            }
        }
    }
    void contract(int u,int v,int p){
        int x=n+1; while(in[x]) ++x; chkmax(m,x);
        match[from[x]=match[x]=match[p]]=in[x]=x;
        vector<int> &c=flower[x];
        for(u=in[u];u!=p;u=in[from[u]]) c.pb(u);
        c.pb(p),reverse(c.begin(),c.end());
        for(v=in[v];v!=p;v=in[from[v]]) c.pb(v);
        for(int y:c) for(int k=1;k<=m;++k) if(!e[x][k].u||len(e[y][k])<len(e[x][k])) e[x][k]=e[k][x]=e[y][k];
        set(x,x); for(int y:c) mpos[y]=from[y]=match[y]=col[y]=0; color(x,OUT);
    }
    void decompose(int b){
        vector<int> &c=flower[b];
        for(int x:c) set(x,x);
        int len=c.size(),p,q; vector<int> cyc(len);
        int fr=other(e[in[from[b]]][b],in[from[b]]),bk=other(e[b][match[b]],match[b]);
        for(p=0;c[p]!=fr;++p); for(q=0;q<len;++q) cyc[q]=c[(p+q)%len]; for(p=0;cyc[p]!=bk;++p);
        if(p&1) reverse(cyc.begin()+1,cyc.end()),p=len-p; from[cyc[0]]=from[b];
        for(q=0;q<p;++q) color(cyc[q],(q&1)?OUT:IN),match[cyc[q]]=cyc[q^1],from[cyc[q+1]]=cyc[q];
        for(q=p+1;q<len;++q) color(cyc[q],NUL),match[cyc[((q-1)^1)+1]]=cyc[q];
        from[match[b]]=match[match[cyc[p]]=match[b]]=cyc[p],color(cyc[p],IN);
        match[b]=in[b]=mpos[b]=from[b]=match[b]=0; c.clear();
        for(int k=1;k<=m;++k) e[b][k]=e[k][b]=edge_t();
    }
    bool detect(int u,int v){
        u=in[u],v=in[v];
        vector<int> path;
        if(col[v]==OUT){
            int p=lca(u,v);
            if(!p){
                while(u) path.pb(u),u=in[from[u]];
                reverse(path.begin(),path.end());
                while(v) path.pb(v),v=in[from[v]];
            }else return contract(u,v,p),false;
        }else if(match[v]){
            return from[v]=u,color(v,IN),from[match[v]]=v,color(match[v],OUT),false;
        }else for(path.pb(v);u;u=in[from[u]]) path.pb(u);
        for(int i=0;i<path.size();++i) match[path[i]]=path[i^1];
        return true;
    }
    bool augment(){
        for(int x=1;x<=m;++x) mpos[x]=from[x]=0,col[x]=in[x]?NUL:0;
        for(int x=1;x<=m;++x) if(in[x]==x) color(x,match[x]?NUL:OUT);
        while(true){
            for(int x=1,c=1;c<=m;x=x%m+1,++c) if(in[x]==x&&col[x]!=IN&&!slack(x)) if(c=detect(mpos[x],x)) return true;
            for(int x=n+1,c=1;c<=m-n;x=(x-n)%(m-n)+n+1,++c) if(in[x]==x&&col[x]==IN&&!h[x]) decompose(x),c=0;
            int d=inf;
            for(int x=1;x<=m;++x) if(in[x]==x) {
                if(col[x]!=IN) chkmin(d,slack(x));
                else if(x>n) chkmin(d,h[x]/2);
            }
            if(d==inf) return false;
            for(int x=1;x<=n;++x){
                if(col[in[x]]==OUT) h[x]-=d;
                if(col[in[x]]==IN) h[x]+=d;
            }
            for(int x=n+1;x<=m;++x) if(in[x]==x) {
                if(col[x]==OUT) h[x]+=d*2;
                if(col[x]==IN) h[x]-=d*2;
            }
        }
        return false;
    }
    void untie(vector<int>&c,int v){
        int p=0,l=c.size();
        for(int x:c) set(x,x);
        while(c[p]!=in[v]) ++p;
        if(in[v]>n) untie(flower[in[v]],v);
        for(int i=1;i<l;i+=2) untie(c[(p+i)%l],c[(p+i+1)%l]);
        c.clear();
    }
    void untie(int x,int y){
        int u=e[x][y].u,v=e[x][y].v;
        if(in[u]!=x) swap(u,v);
        if(e[x][y].w) lk[lk[v]=u]=v;
        if(x>n) untie(flower[x],u);
        if(y>n) untie(flower[y],v);
    }
    void mwpm(){
        for(int u=1;u<=n;++u){
            for(int v=1;v<=n;++v) chkmax(h[u],e[u][v].w);
            h[u]+=h[u]&1,in[u]=u;
        }
        while(augment());
        for(int x=1;x<=m;++x) if(in[x]==x&&match[x]<x) untie(match[x],x);
        long long ret=0;
        for(int x=1;x<=n;++x) if(x<lk[x]) ret+=e[x][lk[x]].w;
        cout<<ret<<'\n';
        for(int x=1;x<=nc;++x) cout<<lk[x]<<" \n"[x==nc];
    }
} ;

int main(){
    ios::sync_with_stdio(false);
    cin.tie(0), cout.tie(0);
    solver a;
    int n, m;
    cin >> n >> m;
    a.init(n);
    for (int i = 1; i <= m; ++i) {
        int u, v, c;
        cin >> u >> v >> c;
        a.add(u, v, c);
    }
    a.mwpm();
}
\end{lstlisting}
\end{framed}

\subsection{网络流}
\subsubsection{Dinic}
\begin{framed}
\begin{lstlisting}
#include<cstdio>
#include<cstring>
#include<algorithm>
#include<vector>
#include<queue>
using namespace std;
#define IL inline
#define ri register int 
#define pb push_back
#define mk make_pair

const int N = 200 + 5;
const int INF = 1e9 + 7;

struct Edge {
    int u, v, c, f;
    Edge(int u=0,int v=0,int c=0,int f=0):
        u(u),v(v),c(c),f(f) {}
};

struct Dinic {
    int n, m, s, t;
    vector<Edge> edges;
    vector<int> G[N];
    bool vis[N];
    int d[N];
    int cur[N];
    
    IL void init(int n) {
        this->n = n;
        for(ri i=1;i<=n;i++) G[i].clear();
        edges.clear();
    }
    
    IL void AddEdge(int u, int v, int c) {
        edges.pb(Edge(u,v,c,0));
        edges.pb(Edge(v,u,0,0));
        m = edges.size();
        G[u].pb(m-2);
        G[v].pb(m-1);
    }
    bool BFS() {
        fill(vis+1,vis+1+n,0);
        queue<int> q;
        q.push(s);
        d[s] = 0;
        vis[s] = true;
        while(!q.empty()) {
            int u = q.front(); q.pop();
            for(ri i=0;i<G[u].size();i++) {
                Edge& e = edges[G[u][i]];
                if(!vis[e.v] && e.c > e.f) {
                    vis[e.v] = true;
                    d[e.v] = d[u] + 1;
                    q.push(e.v);
                }
            }
        }
        return vis[t];
    }
    int DFS(int u, int a) {
        if(u == t || a == 0) return a;
        int flow = 0, f;
        for(ri& i=cur[u];i<G[u].size();i++) {
            Edge& e = edges[G[u][i]];
            if(d[u]+1 == d[e.v] && (f=DFS(e.v, min(a,e.c-e.f))) > 0) {
                e.f += f;
                edges[G[u][i]^1].f -= f;
                flow += f;
                a -= f;
                if(a == 0) break;
            }
        }
        return flow;
    }
    int Maxflow(int s, int t) {
        this->s = s; this->t = t;
        int flow = 0;
        while(BFS()) {
            fill(cur+1, cur+1+n, 0);
            flow += DFS(s, INF);
        }
        return flow;
    }
}solver;
int main(){
    int n,m,s,t;
    scanf("%d%d%d%d",&n,&m,&s,&t);
    solver.init(n);
    for(int i=0;i<m;i++){
        int u,v,w;
        scanf("%d%d%d",&u,&v,&w);
        solver.AddEdge(u,v,w);
    }
    printf("%d\n",solver.Maxflow(s,t));
    return 0;
}
\end{lstlisting}
\end{framed}

\subsubsection{基于 Capacity Scaling 的弱多项式复杂度 dijkstra 最小费用流}

$O(m^2 \log U \log m)$, $U$ 为边的最大容量。

\begin{framed}
\begin{lstlisting}
#include <iostream>
#include <queue>
#include <vector>
#include <algorithm>

using namespace std;

typedef long long ll;
typedef pair<ll, int> pli;

const ll INF = 1e18;
const ll LARGE = 1e12;

int n, m;
vector<bool> vis;
vector<int> head, nxt, from, to, pre;
vector<ll> raw_cap, cap, cost, p, dis;
priority_queue<pli, vector<pli>, greater<pli> > q;

void add(int u, int v, ll f, ll w)
{
    nxt.push_back(head[u]);
    head[u] = to.size();
    from.push_back(u);
    to.push_back(v);
    raw_cap.push_back(f);
    cap.push_back(0);
    cost.push_back(w);
}

void add_edge(int u, int v, ll f, ll w)
{
    add(u, v, f, w);
    add(v, u, 0, -w);
}

ll c(int id)
{
    return p[from[id]] + cost[id] - p[to[id]];
}

void dijkstra(int s)
{
    vis.assign(n + 2, false);
    dis.assign(n + 2, INF);
    pre.assign(n + 2, -1);
    dis[s] = 0;
    q.push(pli(0, s));

    while (!q.empty())
    {
        int u = q.top().second;
        ll w = q.top().first;
        q.pop();
        if (vis[u]) continue;
        vis[u] = true;
        for (int i = head[u]; ~i; i = nxt[i])
        {
            int v = to[i];
            if (cap[i] && dis[v] > w + c(i))
            {
                dis[v] = w + c(i);
                pre[v] = i;
                q.push(pli(dis[v], v));
            }
        }
    }
}

void add_one_cap(int id)
{
    int u = from[id];
    int v = to[id];
    if (cap[id])
    {
        ++cap[id];
        return;
    }
    dijkstra(v);
    if (dis[u] < INF && dis[u] + c(id) < 0)
    {
        ++cap[id ^ 1];
        while (u != v)
        {
            int x = pre[u];
            --cap[x];
            ++cap[x ^ 1];
            u = from[x];
        }
    }
    else ++cap[id];
    ll max_dis = 0;
    ll cur_len = c(id);
    for (int i = 1; i <= n; ++i) if (dis[i] < INF) max_dis = max(max_dis, dis[i]);
    for (int i = 1; i <= n; ++i) p[i] += dis[i] < INF ? dis[i] : max_dis + max(0ll, -cur_len);

    dijkstra(n + 1);
    for (int i = 1; i <= n; ++i) p[i] += dis[i];
}

int main()
{
    int s, t;

    cin >> n >> m >> s >> t;

    head.resize(n + 2, -1);
    p.resize(n + 2, 0);

    for (int i = 1; i <= m; ++i)
    {
        ll u, v, f, w;
        cin >> u >> v >> f >> w;
        add_edge(u, v, f, w);
    }

    add_edge(t, s, LARGE, -LARGE);

    for (int i = 1; i <= n; ++i)
    {
        add_edge(n + 1, i, 0, 0);
        cap[to.size() - 2] = 1;
    }

    for (int i = 40; i >= 0; --i)
    {
        for (int j = 0; j <= m * 2 + 1; ++j) cap[j] <<= 1;
        for (int j = 0; j <= m * 2; j += 2)
        {
            if ((raw_cap[j] >> i) & 1)
            {
                add_one_cap(j);
            }
        }
    }
    ll min_cost = 0;
    for (int i = 0; i < m; ++i) min_cost += cap[i << 1 | 1] * cost[i << 1];
    cout << cap[m << 1 | 1] << ' ' << min_cost;
    return 0;
}
\end{lstlisting}
\end{framed}

\subsubsection{基于 Capacity Scaling 的弱多项式复杂度 SPFA 最小费用流}

复杂度为 $O(nm^2 \log U)$ ,但很难卡满。

\begin{framed}
\begin{lstlisting}
#include <iostream>
#include <queue>
#include <vector>
#include <algorithm>

using namespace std;

typedef long long ll;
typedef pair<ll, int> pli;

const ll INF = 1e18;
const ll LARGE = 1e12;

int n, m;
queue<int> q;
vector<bool> inq;
vector<ll> raw_cap, cap, cost, dis;
vector<int> head, nxt, from, to, pre;

void add(int u, int v, ll f, ll w)
{
    nxt.push_back(head[u]);
    head[u] = to.size();
    from.push_back(u);
    to.push_back(v);
    raw_cap.push_back(f);
    cap.push_back(0);
    cost.push_back(w);
}

void add_edge(int u, int v, ll f, ll w)
{
    add(u, v, f, w);
    add(v, u, 0, -w);
}

void spfa(int s)
{
    inq.assign(n + 1, false);
    dis.assign(n + 1, INF);
    pre.assign(n + 1, -1);
    dis[s] = 0;
    q.push(s);

    while (!q.empty())
    {
        int u = q.front();
        inq[u] = false;
        q.pop();

        for (int i = head[u]; ~i; i = nxt[i])
        {
            int v = to[i];
            ll w = cost[i];
            if (cap[i] && dis[v] > dis[u] + w)
            {
                dis[v] = dis[u] + w;
                pre[v] = i;
                if (!inq[v])
                {
                    inq[v] = true;
                    q.push(v);
                }
            }
        }
    }
}

void add_one_cap(int id)
{
    int u = from[id];
    int v = to[id];
    if (cap[id])
    {
        ++cap[id];
        return;
    }
    spfa(v);
    if (dis[u] < INF && dis[u] + cost[id] < 0)
    {
        ++cap[id ^ 1];
        while (u != v)
        {
            int x = pre[u];
            --cap[x];
            ++cap[x ^ 1];
            u = from[x];
        }
    }
    else ++cap[id];
}

int main()
{
    int s, t; 

    cin >> n >> m >> s >> t;

    head.resize(n + 1, -1);

    for (int i = 1; i <= m; ++i)
    {
        ll u, v, f, w;
        cin >> u >> v >> f >> w;
        add_edge(u, v, f, w);
    }

    add_edge(t, s, LARGE, -LARGE);

    for (int i = 40; i >= 0; --i)
    {
        for (int j = 0; j <= m * 2 + 1; ++j) cap[j] <<= 1;
        for (int j = 0; j <= m * 2; j += 2)
        {
            if ((raw_cap[j] >> i) & 1)
            {
                add_one_cap(j);
            }
        }
    }

    ll min_cost = 0;

    for (int i = 0; i < m; ++i) min_cost += cap[i << 1 | 1] * cost[i << 1];

    cout << cap[m << 1 | 1] << ' ' << min_cost;

    return 0;
}
\end{lstlisting}
\end{framed}

\newpage
\section{字符串}
\subsection{Manacher 算法}
$p[i]-1$ 是以 i 为中心的回文串长度。

\begin{framed}
\begin{lstlisting}
#include<algorithm>
#include<iostream>
#include<cstring>
#include<cstdio>
#include<cmath>
using namespace std;
const int N = 11000000;
char c[N+5],s[2*N+5];
int p[2*N+5];
int Manacher(char *s) {
    memset(p,0,sizeof(p));
    int max_len=0,mid=0,max_r=0,len=strlen(s);
    for(int i=0;i<len;i++) {
        p[i]=min(p[2*mid-i],max_r-i);
        while(s[i+p[i]]==s[i-p[i]]) p[i]++;
        if(max_r<i+p[i]) max_r=i+p[i],mid=i;
        max_len=max(max_len,p[i]-1);
    }
    return max_len;
}
int main() {
    while(scanf("%s",c)!=EOF) {
        int len=strlen(c);
        s[0]='$',s[1]='#';
        for(int i=0;i<len;i++)
            s[2*i+2]=c[i],s[2*i+3]='#';
        s[2*len+2]='\0';
        printf("%d\n",Manacher(s));
    }
    return 0;
}
\end{lstlisting}
\end{framed}

\subsection{后缀自动机}
$cnt[]$ 数组表示字符串出现次数。这段代码求出了:

给定一个只包含小写字母的字符串 S,

请你求出 S 的所有出现次数不为 1 的子串的出现次数乘上该子串长度的最大值。

\begin{framed}
\begin{lstlisting}
#include<bits/stdc++.h>
using namespace std;
typedef long long LL;
#define IL inline
#define pb push_back

const int N = 1000000 + 5;
const int CHAR_NUM = 30;

struct SAM {
    int n, last;
    int len[N<<1];
    int link[N<<1];
    int nxt[N<<1][CHAR_NUM];
    int cnt[N<<1];
    IL void clear(int u) {
        cnt[u] = 0;
        len[u] = 0;
        link[u] = 0;
        for(int i=0;i<CHAR_NUM;i++) nxt[u][i] = 0;
    }
    
    IL void init() {
        clear(0);
        link[0] = -1;
        n = 1;
        last = 0;
    }
    
    IL void sam_extend(int c) { // c = c - 'a'.
        int cur = n++;
        clear(cur);
        cnt[cur] = 1;
        len[cur] = len[last] + 1;
        int p = last;
        while(p != -1 && !nxt[p][c]) {
            nxt[p][c] = cur;
            p = link[p];
        }
        if(p == -1) {
            link[cur] = 0;
        }
        else {
            int q = nxt[p][c];
            if(len[p] + 1 == len[q]) {
                link[cur] = q;
            }
            else {
                int clone = n++;
                len[clone] = len[p] + 1;
                for(int i=0;i<CHAR_NUM;i++) 
                    nxt[clone][i] = nxt[q][i];
                link[clone] = link[q];
                while(p != -1 && nxt[p][c] == q) {
                    nxt[p][c] = clone;
                    p = link[p];
                }
                link[q] = link[cur] = clone;
            }
        }
        last = cur;
    }
    
    vector<int> inv_link[N<<1];
    int dfs(int u, LL& ans) {
        for(int v : inv_link[u]) {
            cnt[u] += dfs(v, ans);
        }
        if(cnt[u] > 1) ans = max(ans, 1LL * cnt[u] * len[u]); 
        return cnt[u];
    }
    LL solve(const char* s) {
        init();
        int lens = strlen(s);
        for(int i=0;i<lens;i++) sam_extend(s[i]-'a');
        for(int i=1;i<n;i++) inv_link[link[i]].pb(i);
        LL ans = 0;
        dfs(0,ans);
        return ans;
    }
};

SAM sam;

char s[N];

int main() {
    scanf("%s",s);
    printf("%d\n",sam.solve(s));
    return 0;
}
\end{lstlisting}
\end{framed}

\subsection{广义后缀自动机}

给 n 个字符串,求出其中本质不同的子串个数(不包含空串)

\begin{framed}
\begin{lstlisting}
#include<bits/stdc++.h>
using namespace std;
#define IL inline
#define mk make_pair
#define fi first
#define se second

const int N = 1000000 + 5;
const int CHAR_NUM = 30;

struct exSAM {
    int n;
    int len[N<<1];
    int link[N<<1];
    int nxt[N<<1][CHAR_NUM];
    
    IL void clear(int u) {
        len[u] = 0;
        link[u] = 0;
        for(int i=0;i<CHAR_NUM;i++) nxt[u][i] = 0;
    }
    
    IL void init() {
        clear(0);
        link[0] = -1;
        n = 1;
    }
    IL int insertSAM(int last,int c) {
        int cur = nxt[last][c];
        if(len[cur]) return cur;
        len[cur] = len[last] + 1;
        int p = link[last];
        while(p != -1 && !nxt[p][c]) {
            nxt[p][c] = cur;
            p = link[p];
        }
        if(p == -1) {
            link[cur] = 0;
            return cur;
        }
        int q = nxt[p][c];
        if(len[p] + 1 == len[q]) {
            link[cur] = q;
            return cur;
        }
        int clone = n++;
        len[clone] = len[p] + 1;
        for(int i=0;i<CHAR_NUM;i++) 
            nxt[clone][i] = len[nxt[q][i]] != 0 ? nxt[q][i] : 0;
        link[clone] = link[q];
        while(p != -1 && nxt[p][c] == q) {
            nxt[p][c] = clone;
            p = link[p];
        }
        link[q] = link[cur] = clone;
        return cur;
    }
    int insertTrie(int cur,int c) {
        if(nxt[cur][c]) return nxt[cur][c];
        return nxt[cur][c] = n++;
    }
    void insert(const string& s) {
        int root = 0;
        for(auto ch : s) root = insertTrie(root, ch-'a');
    }
    void insert(const char* s, int lens) {
        int root = 0;
        for(int i=0;i<lens;i++) root = insertTrie(root, s[i]-'a');
    }
    void build() {
        queue<pair<int,int> > q;
        for(int i=0;i<26;i++)
            if(nxt[0][i]) q.push(mk(i,0));
        while(!q.empty()) {
            pair<int, int> u = q.front(); q.pop();
            int last = insertSAM(u.se, u.fi);
            for(int i=0;i<26;i++)
                if(nxt[last][i]) q.push(mk(i,last));
        }
    }
};

exSAM sam;

int n;
char s[N];

int main() {
    scanf("%d",&n);
    sam.init();
    for(int i=1;i<=n;i++) {
        scanf("%s",s);
        int len = strlen(s);
        sam.insert(s, len);
    }
    
    sam.build();
    
    long long ans = 0;
    for(int i=1;i<sam.n;i++) {
        ans += sam.len[i] - sam.len[sam.link[i]];
    }
    printf("%lld\n",ans);
    return 0;
}
\end{lstlisting}
\end{framed}

\newpage
\section{计算几何}

\subsection{极角排序}

此题 SWERC2018 F.Paris by Night 需要计算一条过两个给定点的直线,使得直线分开的两边权值之和的差值最小。

这里的极角排序是利用判断象限后求叉积,以辐角 $[0,2 \pi )$ 从小到大排序。

\begin{framed}
\begin{lstlisting}
#include<bits/stdc++.h>
using namespace std;
typedef long long LL;
typedef double db;
#define IL inline
#define ri register int

const db eps = 1e-10;
IL int dcmp(db x) {if(fabs(x) < eps) return 0; else return x < 0 ? -1 : 1;}
struct P {
    db x,y;
    IL P(db x=0.0, db y=0.0, db w=0):x(x),y(y) {}
};
typedef P V;
IL V operator + (const V& a,const V& b) {return V(a.x+b.x, a.y+b.y);}
IL V operator - (const V& a,const V& b) {return V(a.x-b.x, a.y-b.y);}
IL V operator * (const V& a,db p) {return V(a.x*p,a.y*p);}
IL V operator / (const V& a,db p) {return V(a.x/p,a.y/p);}
IL bool operator < (const P& a, const P& b) {return a.x < b.x || (a.x == b.x && a.y < b.y);}
IL bool operator == (const P& a, const P& b) {return dcmp(a.x-b.x)==0 && dcmp(a.y-b.y)==0;}
IL db Dot(const V& a,const V& b) {return a.x*b.x+a.y*b.y;}
IL db Length(const V& a) {return sqrt(Dot(a,a));}
IL db Angle(const V& a,const V& b) {return acos(Dot(a,b) / Length(a) / Length(b));}
IL db Cross(const V& a,const V& b) {return a.x*b.y-a.y*b.x;}

IL int Quadrant(const P& a) {
    if(dcmp(a.x) > 0 && dcmp(a.y) >= 0) return 1;
    if(dcmp(a.x) <= 0 && dcmp(a.y) > 0) return 2;
    if(dcmp(a.x) < 0 && dcmp(a.y) <= 0) return 3;
    if(dcmp(a.x) >= 0 && dcmp(a.y) < 0) return 4;
    return 0;
}
IL bool cmp(const P& a, const P& b) {
    return Quadrant(a) < Quadrant(b) || (Quadrant(a) == Quadrant(b) && dcmp(Cross(a,b)) > 0);
}

const int N = 4000 + 5;
const LL INFL = 1e18 + 5;

struct PP {
    P p; int w;
    PP (P p=P(0,0), int w=0):p(p),w(w){}
};

PP p[N], tmp[N];

IL bool OnLeft(const P& a, const V& vec) { return dcmp(Cross(vec, a)) > 0;}

LL sum = 0, suml = 0, sumr = 0, ans = INFL;

int main() {
    sum = suml = sumr = 0;
    ans = INFL;
    int n; scanf("%d",&n);
    for(ri i=0;i<n;i++) {
        int x,y,w; scanf("%d%d%d",&x,&y,&w);
        p[i] = PP(P(x,y),w);
        sum += w;
    }
    for(ri i=0;i<n;i++) {
        int m = 0;
        suml = sumr = 0;
        for(ri j=0;j<n;j++) {
            if(i == j) continue;
            tmp[m++] = PP(p[j].p-p[i].p, p[j].w);
        }
        sort(tmp, tmp+m, [](PP a, PP b) {return cmp(a.p, b.p);});

        for(ri l=0,r=1;l<m;l++) {
            for( ;OnLeft(tmp[r].p, tmp[l].p); ) { // the point of (l,r) are on left.
                suml += tmp[r].w;
                r = (r+1) % m;
            }
            sumr = sum - p[i].w - tmp[l].w - suml;
            ans = min(ans, abs(sumr-suml));
            if(l != m-1) suml -= tmp[l+1].w;
        }
    }
    printf("%lld\n",ans);
    return 0;
}
\end{lstlisting}
\end{framed}

\subsection{平面最近点对}

\subsubsection{分治做法}
给出 n 个点求出平面最近的点对。分治做法 $O(n \log n)$

\begin{framed}
\begin{lstlisting}
#include<bits/stdc++.h>
using namespace std;

const int N = 200000 + 5;

struct P {
    double x, y;
    int id;
    P(double x=0,double y=0,int id=0):x(x),y(y),id(id) {}
};

int n;
P p[N];

struct MinDistInPlane {
    int n;
    P* a;
    double mindist = 1e20;
    int ansa, ansb;
    void upd_ans(const P& a,const P& b) {
        double dist = sqrt((a.x-b.x)*(a.x-b.x)+(a.y-b.y)*(a.y-b.y));
        if(dist < mindist) mindist = dist, ansa = a.id, ansb = b.id;
    }
    
    void solve(int L, int R) {
        if(R-L <= 3) {
            for(int i=L;i<=R;i++)
                for(int j=i+1;j<=R;j++) 
                    upd_ans(a[i], a[j]);
            sort(a+L, a+R+1, [](P x, P y) {return x.y < y.y;});
            return;
        }
        int M = L+R >> 1;
        int MX = a[M].x;
        solve(L,M); solve(M+1,R);
        static P t[N];
        merge(a+L, a+M+1, a+M+1, a+R+1, t, [](P x, P y) {return x.y < y.y;});
        copy(t, t+R-L+1, a+L);
        
        int tsz = 0;
        for(int i=L;i<=R;i++) {
            if(abs(a[i].x-MX) < mindist) {
                for(int j=tsz;j && a[i].y-t[j].y < mindist; j--) {
                    upd_ans(a[i], t[j]);
                }
                t[++tsz] = a[i];
            }
        }
    }
    
    double MinDist(P* a, int n) {
        this->a = a; this->n = n;
        sort(a, a+n, [](P x, P y) {return x.x < y.x || (x.x==y.x && x.y < y.y);});
        solve(0, n-1);
        return mindist;
    }
};

MinDistInPlane solver;

int main() {
    scanf("%d", &n);
    for(int i=0;i<n;i++) {
        scanf("%lf%lf", &p[i].x, &p[i].y);
        p[i].id = i;
    }
    printf("%.4lf\n", solver.MinDist(p, n));
    return 0;
}
\end{lstlisting}
\end{framed}

\subsubsection{非分治做法}

时间复杂度 $O(n \log n)$

\begin{framed}
\begin{lstlisting}
#include<bits/stdc++.h>
using namespace std;

typedef double db;

const int N = 200000 + 5;

struct P {
    db x,y;
    int id;
    P(db x=0, db y=0, int id=0):x(x), y(y), id(id) {}
};

struct MinDistInPlane {
    double mindist=1e20;
    int ansa, ansb;
    struct cmp_x {
        bool operator() (const P& a, const P& b) const {
            return a.x < b.x || (a.x == b.x && a.y < b.y);
        }
    };
    struct cmp_y {
        bool operator() (const P& a, const P& b) const {
            return a.y < b.y;
        }
    };

    multiset<P, cmp_y> s;
    
    void upd_ans(const P& a, const P& b) {
        double dist = sqrt((a.x-b.x)*(a.x-b.x)+(a.y-b.y)*(a.y-b.y));
        if(mindist > dist) mindist = dist, ansa = a.id, ansb = b.id;
    }
    
    double MinDist(P* a, int n) {
        sort(a, a+n, cmp_x());
        for(int i=0,l=0;i<n;i++) {
            while(l<i && a[i].x-a[l].x >= mindist) s.erase(s.find(a[l++]));
            for(auto it=s.lower_bound(P(a[i].x,a[i].y-mindist));
                it != s.end() && it->y - a[i].y < mindist; it++) upd_ans(*it, a[i]);
            s.insert(a[i]);
        }
        return mindist;
    }
};

MinDistInPlane solver;

int n;
P p[N];

int main() {
    scanf("%d",&n);
    for(int i=0;i<n;i++) {
        scanf("%lf%lf",&p[i].x,&p[i].y);
        p[i].id = i;
    }
    printf("%.4lf\n", solver.MinDist(p,n));
    
    return 0;
}
\end{lstlisting}
\end{framed}

\newpage
\section{动态规划}

\subsection{优化方法}

\subsubsection{斜率优化}

设当 $i$ 选 $j$ 转移优于 $l$ 时:($j<l$)

假如推出一个式子,其中 $s[i]$ 单调递增。
$$
\frac{f[j]-s[j]^2-(f[l]-s[l]^2)}{-s[j]-(-s[l])} > s[i]
$$
那么维护一个斜率递增的式子,即下凸包。

\begin{framed}
\begin{lstlisting}
#include<cstdio>
#include<cstring>
#include<algorithm>
using namespace std;
#define IL inline
#define ri register int
typedef double db;
typedef long long LL;

const int N = 1e5 + 3, M = 203;
int n,k,a[N],q[N],pa[N][M];
LL s[N],f[N],g[N];

IL db slope(int i,int j) {
    if(s[i] == s[j]) return -1e18;
    return 1.0*((g[i]-s[i]*s[i])-(g[j]-s[j]*s[j])) / (s[j]-s[i]);
}

int main() {
    scanf("%d%d",&n,&k);
    for(ri i=1;i<=n;i++) scanf("%d",&a[i]), s[i]=s[i-1]+a[i];
    for(ri j=1;j<=k;j++) {
        int head,tail;
        q[head=tail=1] = 0;
        for(ri i=1;i<=n;i++) {
            while(head<tail && slope(q[head],q[head+1]) <= s[i]) ++head;
            f[i] = g[q[head]] + s[q[head]]*(s[i] - s[q[head]]);
            pa[i][j] = q[head];
            while(head<tail && slope(q[tail-1],q[tail]) >= slope(q[tail],i)) --tail;
            q[++tail] = i;
        }
        memcpy(g,f,sizeof(f));
    }
    printf("%lld\n",f[n]);
    for(ri x=n,i=k;i>=1;i--) x=pa[x][i], printf("%d%c",x," \n"[i==1]);
    return 0;
} 
\end{lstlisting}
\end{framed}

\newpage
\section{杂项}

\subsection{卡常火车头}

\begin{framed}
\begin{lstlisting}
#define fastcall __attribute__((optimize("-O3")))
#pragma GCC optimize(2)
#pragma GCC optimize(3)
#pragma GCC optimize("Ofast")
#pragma GCC optimize("inline")
#pragma GCC optimize("-fgcse")
#pragma GCC optimize("-fgcse-lm")
#pragma GCC optimize("-fipa-sra")
#pragma GCC optimize("-ftree-pre")
#pragma GCC optimize("-ftree-vrp")
#pragma GCC optimize("-fpeephole2")
#pragma GCC optimize("-ffast-math")
#pragma GCC optimize("-fsched-spec")
#pragma GCC optimize("unroll-loops")
#pragma GCC optimize("-falign-jumps")
#pragma GCC optimize("-falign-loops")
#pragma GCC optimize("-falign-labels")
#pragma GCC optimize("-fdevirtualize")
#pragma GCC optimize("-fcaller-saves")
#pragma GCC optimize("-fcrossjumping")
#pragma GCC optimize("-fthread-jumps")
#pragma GCC optimize("-funroll-loops")
#pragma GCC optimize("-freorder-blocks")
#pragma GCC optimize("-fschedule-insns")
#pragma GCC optimize("inline-functions")
#pragma GCC optimize("-ftree-tail-merge")
#pragma GCC optimize("-fschedule-insns2")
#pragma GCC optimize("-fstrict-aliasing")
#pragma GCC optimize("-falign-functions")
#pragma GCC optimize("-fcse-follow-jumps")
#pragma GCC optimize("-fsched-interblock")
#pragma GCC optimize("-fpartial-inlining")
#pragma GCC optimize("no-stack-protector")
#pragma GCC optimize("-freorder-functions")
#pragma GCC optimize("-findirect-inlining")
#pragma GCC optimize("-fhoist-adjacent-loads")
#pragma GCC optimize("-frerun-cse-after-loop")
#pragma GCC optimize("inline-small-functions")
#pragma GCC optimize("-finline-small-functions")
#pragma GCC optimize("-ftree-switch-conversion")
#pragma GCC optimize("-foptimize-sibling-calls")
#pragma GCC optimize("-fexpensive-optimizations")
#pragma GCC optimize("inline-functions-called-once")
#pragma GCC optimize("-fdelete-null-pointer-checks")
\end{lstlisting}
\end{framed}

\subsection{防止 unorderedmap 被卡方法}

\begin{framed}
\begin{lstlisting}
struct custom_hash {
    static uint64_t splitmix64(uint64_t x) {
        x += 0x9e3779b97f4a7c15;
        x = (x ^ (x >> 30)) * 0xbf58476d1ce4e5b9;
        x = (x ^ (x >> 27)) * 0x94d049bb133111eb;
        return x ^ (x >> 31);
    }

    size_t operator()(uint64_t x) const {
        static const uint64_t FIXED_RANDOM = chrono::steady_clock::now().time_since_epoch().count();
        return splitmix64(x + FIXED_RANDOM);
    }
};
unordered_map<long long, int, custom_hash> safe_map;
gp_hash_table<long long, int, custom_hash> safe_hash_table;
\end{lstlisting}
\end{framed}

\subsection{离线算法}

\subsubsection{CDQ 分治}

有 n 个元素,第 i 个元素有 $a_i,b_i,c_i$ 三个属性,设 $f(i)$ 表示满足 $a_j \leq a_i$ 且 $b_j \leq b_i$ 且 $c_j \leq c_i$ 且 $j \ne i$ 的 j 的数量。

最大的元素值为 K 。

三位偏序,ans[] 中存储 $f(i)$,anss[i] 中存储有多少个 $f(j) = i$。

注意如果是在 dp 过程中,必须要把 solve(M+1,R) 的过程放到中间过程的后面。

如果算法复杂度不超过 $O(M \log M)$ ,请把 sort 过程换为类似归并排序的 merge 过程,防止复杂度炸到 $O(M \log^2 M)$

\begin{framed}
\begin{lstlisting}
#include<cstdio>
#include<cstring>
#include<algorithm>
#include<vector>
using namespace std;
typedef long long LL;
#define IL inline
#define ri register int 

const int N = 400000 + 5;
const int INF = 1e9 + 7;

int fwn = 2e5;
int fw[N];
IL int lb(int x) {return x & (-x);}
IL void fw_add(int x,int d) {
    for(;x<=fwn;x+=lb(x)) {
        fw[x] += d;
    }
}
IL int fw_sum(int x) {
    int ans = 0;
    for(;x;x-=lb(x)) {
        ans += fw[x];
    }
    return ans;
}

struct Node {
    int op,a,b,c,id;
};

int n,K;
Node q[N], q0[N];
IL bool cmpa(const Node& a, const Node& b) {
    return a.a < b.a || (a.a==b.a && a.op < b.op);
}
IL bool cmpb(const Node& a, const Node& b) {
    return a.b < b.b || (a.b==b.b && a.op < b.op);
}

int ans[N], anss[N];

void solve(int L,int R) {
    if(L == R) return;
    int M = L+R >> 1;
    solve(L,M);
    solve(M+1,R);
    
    sort(q+L,q+1+R,cmpb); // O(M log^2 M) 
                          // if complexity is less than that, plz merge it.
    for(ri i=L;i<=R;i++) {
        if(q[i].op == 1 && q[i].a <= M) fw_add(q[i].c, +1);
        if(q[i].op == 2 && q[i].a > M) {
            ans[q[i].id] += fw_sum(q[i].c);
        } 
    }
    for(ri i=L;i<=R;i++) {
        if(q[i].op == 1 && q[i].a <= M) fw_add(q[i].c, -1);
    }
    for(ri i=L;i<=R;i++) q[i] = q0[i];
}

int a[N];

int main() {
    scanf("%d%d",&n,&K);
    for(ri i=1;i<=n;i++) {
        scanf("%d%d%d",&q[i].a,&q[i].b,&q[i].c);
        q[i+n] = q[i];
        q[i].op = 1;
        q[i+n].op = 2;
        q[i+n].id = i;
    }
    sort(q+1,q+1+2*n, cmpa);
    for(ri i=1;i<=2*n;i++) q[i].a = i;
    
    memcpy(q0,q,sizeof(q));
    solve(1,2*n);
    
    for(ri i=1;i<=n;i++) {
        --ans[i];
        ++anss[ans[i]];
    } 
    for(ri i=0;i<n;i++) printf("%d\n",anss[i]);
    return 0;
}
\end{lstlisting}
\end{framed}

\subsubsection{整体二分}

给定一个含有 $n$ 个数的序列 $a_1,a_2 \dots a_n$,需要支持两种操作:

- Q l r k 表示查询下标在区间 $[l,r]$ 中的第 $k$ 小的数

- C pos v 表示将 $a_{pos}$ 改为 $v$

\begin{framed}
\begin{lstlisting}
#include<cstdio>
#include<cstring>
#include<algorithm>
#include<cctype>
using namespace std;
typedef long long LL;
#define IL inline
#define ri register int 
template<typename Tp> IL void read(Tp& x) {
    x=0; int f=1; char ch=getchar();
    while(!isdigit(ch)) {if(ch=='-')f=-1;ch=getchar();}
    while(isdigit(ch)) {x = x*10+ch-'0';ch=getchar();}
    x *= f;
}
int buf[22];
template<typename Tp> IL void write(Tp x) {
    int p = 0;
    if(x < 0) {putchar('-'); x=-x;}
    if(x == 0) buf[++p] = 0;
    else while(x) {
        buf[++p] = x % 10;
        x /= 10;
    }
    for(ri i=p;i>=1;i--) putchar('0'+buf[i]);
}
const int N = 300000 + 5;
int n,m;
int a[N];
struct Node {
    // op == 1 -> insert, x -> a[y], y -> y
    // op == -1 -> remove, x -> a[y], y -> y
    // op == 2 -> query, x -> qL, y -> qR
    int op, x, y, k;
    int id; // id is the id of queries.
    Node(int op=0, int x=0,int y=0,int k=0,int id=0):
        op(op), x(x), y(y), k(k), id(id) {}
} q[N], lq[N], rq[N];

int fw[N];
IL int lb(int x) { return x & (-x);}
IL void fwadd(int x,int d) {
    for(;x<=n;x+=lb(x)) {
        fw[x] += d;
    } 
}
IL int fwsum(int x) {
    int ans = 0;
    for(;x;x-=lb(x)) {
        ans += fw[x];
    }
    return ans;
}

int ans[N]; 
void solve(int vL,int vR,int qL,int qR) {
    if(qL > qR) return;
    if(vL == vR) {
        for(ri i=qL;i<=qR;i++) {
            if(q[i].op == 2) ans[q[i].id] = vL;
        }
        return ;
    }
    
    //insert
    int vM = vL + vR >> 1;
    int nL = 0, nR = 0;
    for(ri i=qL;i<=qR;i++) {
        if(q[i].op != 2) {
            if(q[i].x <= vM) fwadd(q[i].y, q[i].op), lq[++nL] = q[i];
            else rq[++nR] = q[i];
        }
        else {
            int now = fwsum(q[i].y) - fwsum(q[i].x-1);
            if(q[i].k <= now) lq[++nL] = q[i];
            else {
                q[i].k -= now;
                rq[++nR] = q[i];
            }
        }
    }
    
    //remove
    for(ri i=qL;i<=qR;i++) {
        if(q[i].op != 2) {
            if(q[i].x <= vM) fwadd(q[i].y, -q[i].op);
        }
    }
    
    for(ri i=1;i<=nL;i++) q[qL+i-1] = lq[i];
    for(ri i=1;i<=nR;i++) q[qL+nL+i-1] = rq[i];
    solve(vL, vM, qL, qL+nL-1);
    solve(vM+1, vR, qL+nL, qR);
}

int main() {
    read(n); read(m);
    int cnt = 0, Q = 0;
    for(ri i=1;i<=n;i++) {
        read(a[i]);
        q[++cnt] = Node(1,a[i],i);
    } 
    for(ri i=1;i<=m;i++) {
        char op = '\0';
        while(!isupper(op)) op = getchar();
        if(op == 'Q') {
            int l,r,k; read(l); read(r); read(k);
            q[++cnt] = Node(2,l,r,k,++Q);
        }
        else {
            int pos,v; read(pos); read(v);
            q[++cnt] = Node(-1,a[pos],pos);
            q[++cnt] = Node(+1,v,pos);
            a[pos] = v;
        }
    }
    solve(-1e9,1e9,1,cnt);
    for(ri i=1;i<=Q;i++) {
        write(ans[i]); putchar(10);
    }
    return 0;
}
\end{lstlisting}
\end{framed}

\subsection{随机化算法}

\subsubsection{模拟退火}

\begin{framed}
\begin{lstlisting}
#include <bits/stdc++.h>
#define down 0.996//徐徐降温 

using namespace std;

int n;
struct node{
int x;
int y;
int w;
}object[2005];//存下物体的坐标 
double ansx,ansy,answ;//最终答案 
double energy(double x,double y)//根据物理学知识,能量总和越小越稳定 
{
   double r=0,dx,dy;
   for (int a=1;a<=n;a++)
   {
      dx=x-object[a].x;
      dy=y-object[a].y;
      r+=sqrt(dx*dx+dy*dy)*object[a].w;
   }
      return r;
}
void sa()//模拟退火 
{
   double t=3000;//温度要足够高 
   while (t>1e-15)//略大于0 
   {
      double ex=ansx+(rand()*2-RAND_MAX)*t;//随机产生新的答案 
      double ey=ansy+(rand()*2-RAND_MAX)*t;
      double ew=energy(ex,ey);
      double de=ew-answ;
      if (de<0)//如果此答案更优,就接受 
      {
         ansx=ex;
         ansy=ey;
         answ=ew;
      }
      else if(exp(-de/t)*RAND_MAX>rand())//否则根据多项式概率接受 
      {
         ansx=ex;
         ansy=ey;
      }
      t*=down;
   }
}
void solve()//多跑几遍退火,增加得到最优解的概率 
{
   sa();
   sa();
   sa();
   sa();
}
int main() {
    cin>>n;
    for (int a=1;a<=n;a++)
    {
           scanf("%d%d%d",&object[a].x,&object[a].y,&object[a].w);
           ansx+=object[a].x;
           ansy+=object[a].y;
    }
    ansx/=n;//以平均数作为初始答案 
    ansy/=n;
    answ=energy(ansx,ansy);
    solve();
    printf("%.3lf %.3lf\n",ansx,ansy);//华丽的输出 
    return 0;
}
\end{lstlisting}
\end{framed}

一些技巧

1.分块模拟退火

有时函数的峰很多,模拟退火难以跑出最优解。

此时可以把整个值域分成几段,每段跑一遍模拟退火,然后再取最优解。

2.卡时

有一个 clock() 函数,返回程序运行时间。

可以把主程序中的 simulateAnneal(); 换成 while ((double)clock()/CLOCKS\_PER\_SEC < MAX\_TIME) simulateAnneal(); 。这样子就会一直跑模拟退火,直到用时即将超过时间限制。

这里的 MAX\_TIME 是一个自定义的略小于时限的数。

\subsection{悬线法}

\subsubsection{求最大的由 1 组成的正方形的边长}

给一个 01 矩阵,求出最大由 1 组成的正方形的边长。

\begin{framed}
\begin{lstlisting}
#include<cstdio>
#include<cstring>
#include<algorithm>
using namespace std;

const int N = 100 + 5;

int n,m;
int a[N][N];
int up[N][N],left[N][N],right[N][N];

int main(){
    scanf("%d%d",&n,&m);
    for(int i=1;i<=n;i++)    {
        for(int j=1;j<=m;j++){
            scanf("%d",&a[i][j]);
        }
    }
    for(int i=0;i<=m;i++) right[0][i] = m;
    int lo,ro,ans=0;
    for(int i=1;i<=n;i++){
        lo = 0,ro = m + 1;
        for(int j=1;j<=m;j++){
            if(!a[i][j]) {
                up[i][j] = left[i][j] = 0; lo = j;
            }
            else {
                up[i][j] = up[i-1][j]+1;
                left[i][j] = max(left[i-1][j],lo+1);
            }
        }
        for(int j=m;j>=1;j--){
            if(!a[i][j]){right[i][j] = m+1;ro = j;}
            else {
                right[i][j] = min(right[i-1][j],ro-1);
                ans = max(ans,min(up[i][j],right[i][j]-left[i][j]+1));
            }
        }
    }
    printf("%d\n",ans);
    return 0;
}
\end{lstlisting}
\end{framed}

\subsubsection{悬线法+拉伸法填充矩形}

NWRRC 2019 K.King's Children

给你一个 n * m 的矩形地图,里面 . 表示空地,然后在矩形中给你 A-Z 表示需要以这些点扩展一个矩形出来。

你要使 A 扩展出来的矩形最大,并且让其他的大写字母扩展出来的矩形把整个矩形地图填充满。

先用悬线法把 A 扩展出来,剩下的利用拉伸法(即代码中的 split 函数)拉伸出来。

\begin{framed}
\begin{lstlisting}
#include<cstdio>
#include<cstring>
#include<algorithm>
#include<cctype>
using namespace std;
#define IL inline
#define ri register int 

const int N = 1000 + 5;

int n,m, ar, ac;
char G[N][N];
int up[N][N], left[N][N], right[N][N];
IL int getA(int &U,int &D,int &L,int &R) {
    for(ri i=0;i<=m;i++) right[0][i] = m;
    int lo, ro, ans = 0, ansr, ansc;
    for(ri i=1;i<=n;i++) {
        lo = 0, ro = m + 1;
        for(ri j=1;j<=m;j++) {
            if(G[i][j] != '.') {
                up[i][j] = left[i][j] = 0; lo = j;
            }
            else {
                up[i][j] = up[i-1][j] + 1;
                left[i][j] = max(left[i-1][j], lo+1);
            }
        }
        for(ri j=m;j>=1;j--) {
            if(G[i][j] != '.') { right[i][j] = m+1; ro=j;}
            else {
                right[i][j] = min(right[i-1][j], ro-1);
                if(i-up[i][j]+1 <= ar && ar <= i && left[i][j] <= ac && ac <= right[i][j]) {
                    if(up[i][j] * (right[i][j]-left[i][j]+1) > ans) {
                        ans = up[i][j] * (right[i][j]-left[i][j]+1);
                        ansr = i;
                        ansc = j;
                    }
                }
            }
        }
    }
    U = ansr - up[ansr][ansc] + 1;
    D = ansr;
    L = left[ansr][ansc];
    R = right[ansr][ansc];
    return ans; 
}

IL char lower(char ch) {
    if(isupper(ch)) return tolower(ch);
    else return ch;
}

IL void split(int U,int D,int L,int R) {
    if(U > D || L > R) return;
    for(ri i=U+1;i<=D;i++) 
        for(ri j=L;j<=R;j++) 
            if(G[i][j] == '.') 
                G[i][j] = lower(G[i-1][j]);
    for(ri i=D-1;i>=U;i--) 
        for(ri j=L;j<=R;j++) 
            if(G[i][j] == '.') 
                G[i][j] = lower(G[i+1][j]);
    for(ri i=U;i<=D;i++) 
        for(ri j=L+1;j<=R;j++) 
            if(G[i][j] == '.') 
                G[i][j] = lower(G[i][j-1]);
    for(ri i=U;i<=D;i++) 
        for(ri j=R-1;j>=L;j--) 
            if(G[i][j] == '.') 
                G[i][j] = lower(G[i][j+1]);
}

int main() {
    scanf("%d%d",&n,&m);
    for(ri i=1;i<=n;i++) {
        scanf("%s",G[i]+1);
        for(ri j=1;j<=m;j++) {
            if(G[i][j] == 'A') {
                ar = i; ac = j;
                G[i][j] = '.';
            }
        }
    }
    int bestU, bestD, bestL, bestR;
    getA(bestU,bestD,bestL,bestR);
    for(ri i=bestU;i<=bestD;i++) {
        for(ri j=bestL;j<=bestR;j++) {
            G[i][j] = 'a';
        }
    }
    G[ar][ac] = 'A';
    split(1,bestU-1,1,m);
    split(bestD+1,n,1,m);
    split(bestU,bestD,1,bestL-1);
    split(bestU,bestD,bestR+1,m);
    for(ri i=1;i<=n;i++) {
        printf("%s\n",G[i]+1);
    }
    return 0;
}
\end{lstlisting}
\end{framed}

\newpage
\section{其他}

\subsection{emacs配置}

\begin{framed}
\begin{lstlisting}
(global-linum-mode t) ;;line num 显示行号

(setq-default indent-tabs-mode nil)  ;; 将 tab 转变为空格
(setq c-basic-offset 4)
(setq default-tab-width 4) ;;tab width 设置缩进以及 tab 键

(electric-pair-mode t)
(electric-layout-mode t)
(electric-indent-mode t)
;;electric pair 括号补全

(show-paren-mode t) ;;show paren 括号配对

;;(global-hl-line-mode t) ;; high light line 高亮当前行

(global-set-key [f6] 'gdb) ;; 摁 F6 进入 gdb 调试

(global-set-key [f7] 'compile) ;; 摁 F7 编译
(global-set-key [f8] 'shell) ;; 摁 F8 进入 shell


;;windows 下的含中文的文件 emacs 过于慢的问题。
(dolist (charset '(kana han symbol cjk-misc bopomofo))
(set-fontset-font (frame-parameter nil 'font)
charset
(font-spec :family "Microsoft Yahei" :size 18)))

;;;;; 设置编译信息
(defun compile-file ()
  (interactive)
  (compile (format "g++ -o %s %s -g -lm -Wall -std=c++17"  (file-name-sans-extension (buffer-name))(buffer-name))))
(global-set-key (kbd "<f9>") 'compile-file)
;;;;; 设置 F9 一键调试
;;;;; 改变 emacs 标题栏的标题
(setq frame-title-format "LPR nb!")
;;;;; 允许 emacs 和外部其他程序的粘贴
(setq x-select-enable-clipboard t)
;; 显示列号
(setq column-number-mode t)
;;;;; 修改透明度
(set-frame-parameter (selected-frame) 'alpha (list 90 50))
(add-to-list 'default-frame-alist (cons 'alpha (list 90 50)))
\end{lstlisting}
\end{framed}

\subsection{对拍}
\subsubsection{Linux}

\begin{framed}
\begin{lstlisting}
#include<bits/stdc++.h>
using namespace std;
int main(){
    int i;
  for (i=1;;i++){
        printf("The result of No. %d Case is:  ",i);
        system("./rand");
        system("./std");
        system("./my");
        if (system("diff std.out my.out")){
            printf("Wrong Answer\n");
            return 0;
        }
        else printf("Accepted\n");
    }
    return 0;
}
\end{lstlisting}
\end{framed}

\subsubsection{windows}

\begin{framed}
\begin{lstlisting}
@echo off
:loop
data.exe
A.exe
std.exe
fc A.out std.out
if not errorlevel 1 goto loop
pause
:end
\end{lstlisting}
\end{framed}

\subsection{C++ 输出}

\subsubsection{printf}

\begin{framed}
\begin{lstlisting}
%d 十进制有符号整数
%u 十进制无符号整数
%f 浮点数
%s 字符串
%c 单个字符
%p 指针的值
%e 指数形式的浮点数
%x, %X 无符号以十六进制表示的整数
%o 无符号以八进制表示的整数
%g 把输出的值按照 %e 或者 %f 类型中输出长度较小的方式输出
%p 输出地址符
%lu 32 位无符号整数
%llu 64 位无符号整数
\end{lstlisting}
\end{framed}

\subsubsection{cout}

\begin{framed}
\begin{lstlisting}
 dec 设置整数为十进制
 hex 设置整数为十六进制
 oct 设置整数为八进制
 setbase(n) 设置整数为 n 进制 (n=8,10,16)
 setfill(n) 设置字符填充,c 可以是字符常或字符变量
 setprecision(n) 设置浮点数的有效数字为 n 位
 setw(n) 设置字段宽度为 n 位
 setiosflags(ios::fixed) 设置浮点数以固定的小数位数显示
 setiosflags(ios::scientific) 设置浮点数以科学计数法表示
 setiosflags(ios::left) 输出左对齐
 setiosflags(ios::right) 输出右对齐
 setiosflags(ios::skipws) 忽略前导空格
 setiosflags(ios::uppercase) 在以科学计数法输出 E 与十六进制输出 X 以大写输出,否则小写。
 setiosflags(ios::showpos) 输出正数时显示 "+" 号
 setiosflags(ios::showpoint) 强制显示小数点
 resetiosflags() 终止已经设置的输出格式状态,在括号中应指定内容
\end{lstlisting}
\end{framed}

\subsection{注意事项}

\subsubsection{热身赛}

1. windows 还是 linux,确定 rand() 范围 assert(RAND\_MAX == (1 << 15) - 1); assert(RAND\_MAX == (1ll << 31ll) - 1);

2. Lf 是否支持 printf("\%Lf", 0.5);

3. 测试 lld 和 I64d

4. 栈

5. RE 会不会变成 TLE

6. MLE 会不会变成 TLE

7. 测试评测时间以及最高运行次数。(狂炸评测机)。

\subsubsection{正赛}

1. 吃早饭

2. 队友严禁划水!

\subsubsection{WA 了怎么办}

1. 代码不长,先叫有别的题的队友上机/给自己设计一组数据,自己打印下去瞪眼,然后考虑边界情况的数据,最后可以尝试对拍。如果对拍不出问题,说明读错题/大数据出错了,重读题干并仔细检查空间和 long long/int128 是否开对了。

2. 代码其长无比的大模拟建议瞪眼,之后分段调试。

\end{document}